\chapter{Der Tod des Subjekts: Die Ära des Strukturalismus}

\section{Einleitung: Der kalte Wind der Struktur}
In den 1960er Jahren änderte sich das intellektuelle Klima in Paris radikal.
War die Ära Sartres noch geprägt von \enquote{Hitze}, Rauch, Jazzkellern und leidenschaftlichen Appellen an die Freiheit, so brachte das neue Jahrzehnt eine \textbf{technokratische Kälte}.
Der Existentialismus hatte den Menschen (das Subjekt) auf den Thron gesetzt. Er war heroisch, tragisch und moralisch.
Doch nun traten Denker auf den Plan, die diesen Heroismus für naiv hielten. Sie trugen keine Baskenmützen, sondern Anzüge. Sie sprachen nicht von \enquote{Angst} oder \enquote{Ekel}, sondern von \enquote{Codes}, \enquote{Verwandtschaftssystemen} und \enquote{Diskursen}.
Der \textbf{Strukturalismus} war geboren.

Sein Angriff galt dem Heiligtum der westlichen Philosophie: dem \textbf{Subjekt}.
Seit Descartes galt das \enquote{Ich denke} als das unerschütterliche Zentrum der Welt. Sartre hatte dieses \enquote{Ich} ins Absolute gesteigert: Ich bin der Schöpfer meiner Werte.
Die Strukturalisten drehten den Spieß um. Ihre These war ein Schock:
\textbf{Nicht wir erschaffen die Welt. Die Strukturen erschaffen uns.}
Wir glauben, frei zu sprechen, aber wir folgen nur den Regeln der Grammatik. Wir glauben, wen wir lieben, sei unsere Wahl, aber es folgt unsichtbaren Verwandtschaftsregeln. Wir glauben, unsere Gedanken seien unsere eigenen, aber sie sind das Produkt einer historischen Epoche.
Der Slogan lautete nicht mehr \enquote{Ich denke}, sondern \enquote{Es denkt in mir} (Ça pense).
Das Subjekt ist nicht der Herr im eigenen Haus. Es ist nur ein Schnittpunkt von Linien, ein Effekt von Systemen, die es weder kontrolliert noch versteht.
Wie ein Geologe, der die Schichten unter der Landschaft untersucht, wollten die Strukturalisten die unsichtbaren Gitter (Grilles) freilegen, die unser Denken, Fühlen und Handeln determinieren. Es war der Abschied vom Humanismus und der Aufbruch in eine \enquote{Kälte der Hellsichtigkeit}.
Treibende Kraft hinter diesem Umbruch war eine neue Faszination für die Sprache. Inspiriert vom Linguisten Ferdinand de Saussure, begannen diese Denker, die Gesellschaft nicht mehr historisch zu lesen, sondern wie einen Text. Die große Entdeckung lautete: Bedeutung entsteht nicht durch die Dinge selbst oder durch das Bewusstsein des Einzelnen, sondern allein durch die Differenz der Zeichen im System. Wer die Kultur verstehen wollte, durfte nicht mehr auf die Geschichte (die Diachronie) schauen, wie es Hegel und Marx getan hatten, sondern musste die starren, zeitlosen Regeln (die Synchronie) freilegen, die im Hintergrund operieren.

Dieser Ansatz versprach endlich das, was der Philosophie lange gefehlt hatte: wissenschaftliche Exaktheit. Man wollte weg vom schwammigen "Erleben" des Existentialismus hin zu einer mathematischen Präzision der Geisteswissenschaften.

\section{Die Revolution der Sprache: Saussure als Fundament}
Eigentlich hatte der Strukturalismus keinen philosophischen Vater, sondern einen linguistischen.
Alles begann mit \textbf{Ferdinand de Saussure} (1857–1913), einem Schweizer Sprachwissenschaftler, dessen Vorlesungen posthum als \textit{Grundfragen der allgemeinen Sprachwissenschaft} (1916) erschienen.
Saussure revolutionierte unser Verständnis von Sprache – und damit vom Denken selbst.
Seine Kernidee: Sprache ist keine Liste von Namen für Dinge (Nomenklatur). Sie ist ein geschlossenes System von Zeichen.

\begin{tcolorbox}[title=Das sprachliche Zeichen]
Saussure zerlegte das \enquote{Wort} in zwei untrennbare Seiten (wie Vorder- und Rückseite eines Blattes Papier):
\begin{itemize}
    \item \textbf{Das Signifikat (Signifié):} Die Vorstellung / das Konzept (z.B. die geistige Idee eines Baumes).
    \item \textbf{Der Signifikant (Signifiant):} Das Lautbild / das geschriebene Wort (z.B. die Buchstaben B-A-U-M).
\end{itemize}
Der Clou: Die Verbindung zwischen beiden ist absolut \textbf{willkürlich} (arbiträr).
Das bedeutet: Das Lautbild \enquote{B-A-U-M} hat keinerlei innere Verbindung zu dem hölzernen Ding in der Natur.
Der Beweis ist simpel: Wäre der Name naturnotwendig mit der Sache verbunden, gäbe es weltweit nur eine einzige Sprache. Dass die Engländer \textit{tree} und die Franzosen \textit{arbre} sagen, beweist, dass das Zeichen eine reine \textbf{Konvention} ist.
\end{tcolorbox}

Wenn die Zeichen nun aber willkürlich sind (also keinen inneren Wert haben), woher wissen wir dann, was sie bedeuten?
Saussures geniale Antwort: \textbf{Bedeutung entsteht nur aus der Differenz.}
Wir erkennen ein Wort nicht an seinem Inhalt, sondern daran, dass es sich von anderen Wörtern unterscheidet.
Ein Beispiel: Das Wort \enquote{Nacht} bedeutet nichts anderes als \enquote{Nicht-Tag}.
Denken wir an das Farbspektrum. In der Natur gehen Farben fließend ineinander über. Es gibt keine Linie, wo \enquote{Blau} aufhört und \enquote{Grün} anfängt.
Die Sprache zieht diese Linien willkürlich ein. Wenn eine Sprache das Wort \enquote{Türkis} nicht kennt, dann gehört diese Nuance eben noch zu \enquote{Blau}.
Das heißt: Die Bedeutung eines Begriffs hängt untrennbar von den Grenzen zu seinen Nachbarn ab. Verschiebt sich eine Grenze, ändert sich das ganze System.

\paragraph{Die Schach- und Zug-Metapher}
Um zu verstehen, warum die \textbf{Struktur} wichtiger ist als der Inhalt, nutzte Saussure zwei berühmte Bilder:
\begin{enumerate}
    \item \textbf{Der Zug von Genf nach Paris (8:20 Uhr):} Dieser Zug ist eine feste Einheit im Fahrplan. Aber physikalisch ist er jeden Tag anders: Andere Waggons, andere Lok, anderes Personal. Trotzdem ist es \enquote{derselbe} Zug. Warum? Weil er durch seine \textbf{Position im System} (Fahrplan) definiert ist und sich von dem Zug um 9:30 Uhr unterscheidet.
    \item \textbf{Das Schachspiel:} Ob der Springer aus Holz oder Elfenbein ist, ist egal (Substanz). Wichtig ist nur, wie er ziehen darf und wie er sich vom Läufer unterscheidet (Form/Struktur).
\end{enumerate}
\subsection*{Die philosophische Konsequenz: Das Subjekt als Funktion}
Die philosophische Tragweite dieser linguistischen These war verheerend für den klassischen Humanismus.
Wenn Bedeutung nicht in den Dingen selbst liegt und auch nicht im Kopf des Individuums entsteht, sondern ausschließlich im \textit{System der Unterschiede}, dann verliert das Subjekt seine Macht.
\begin{itemize}
    \item \textbf{Die Dezentrierung:} Wir sind nicht die \enquote{Autoren} unserer Sprache. Wir treten bei der Geburt in ein riesiges, vorgefertigtes Netz aus Differenzen ein. Dieses Netz (die Sprache) war vor uns da und wird nach uns da sein.
    \item \textbf{Das Raster der Welt:} Bevor die Sprache die Welt in Begriffe wie \enquote{Fluss}, \enquote{Bach} und \enquote{Strom} unterteilt, ist das Wasser nur eine formlose Masse. Die Sprache legt ein \textbf{Raster} über die Realität. Wir sehen die Welt nicht \enquote{an sich}, sondern wir sehen sie durch die Gitterstäbe unserer Grammatik und unseres Wortschatzes.
\end{itemize}
Der Existentialist Sartre rief: \enquote{Ich gebe den Dingen ihren Sinn!}
Der Strukturalist antwortet nüchtern: \enquote{Nein. Du bedienst nur die Apparatur einer Sprache, die dich längst definiert hat.} Das \enquote{Ich} ist keine freie Schöpferkraft mehr, sondern eine \textbf{Funktion der Struktur}.

\paragraph{Der Funke springt über: Von der Sprache zur Kultur}
Hier geschah die entscheidende historische Wende.
In den 1940er und 50er Jahren erkannte ein junger Ethnologe das explosive Potenzial dieser These: \textbf{Claude Lévi-Strauss}.
Er stellte eine radikale Frage:
Wenn die Sprache ein System aus unbewussten Regeln ist, das durch binäre Oppositionen (Hell/Dunkel, Singular/Plural) funktioniert – gilt das dann nicht auch für den Rest der Kultur?
Funktionieren unsere Heiratsregeln, unsere Tischsitten und unsere Mythen nicht genau wie eine Sprache?
Lévi-Strauss wagte den Versuch, die Gesellschaft nicht mehr als Historiker zu lesen (Wer tat was?), sondern als Linguist (Welche unbewusste Grammatik steuert das Ganze?).
Damit verließ der Strukturalismus den Hörsaal der Linguisten und eroberte die Welt der Ethnologie.

\section{Claude Lévi-Strauss: Die wilden Mythen}
Der Mann, der den Strukturalismus zur Weltmacht führte, war kein Philosoph, sondern ein Ethnologe: \textbf{Claude Lévi-Strauss} (1908–2009).
Sein Buch \textit{Traurige Tropen} (1955) ist eines der großen melancholischen Meisterwerke des 20. Jahrhunderts. Es beginnt mit dem berühmten Satz: \enquote{Ich hasse Reisen und Forschungsreisende.}
Lévi-Strauss war fasziniert von den indigenen Völkern im Amazonas. Doch er suchte dort nicht nach Abenteuern, sondern nach der \textbf{universalen Grammatik der Menschheit}.
Er wandte Saussures Methode auf die Kultur an. Seine These: Kochen, Heiraten und Sagen sind wie eine Sprache. Sie folgen unbewussten Regeln.

\subsection{Das Inzest-Tabu als soziale Syntax}
Wo beginnt die Kultur? Für Lévi-Strauss beginnt sie mit dem \textbf{Inzest-Verbot}.
Lange dachte man, Inzest sei verboten, weil es biologisch schädlich sei. Lévi-Strauss zeigte: Das ist Unsinn (Völker ohne Genetik-Wissen verbieten es auch).
Der wahre Grund ist \textbf{strukturell}:
Das Verbot zwingt die Männer, ihre Schwestern und Töchter nicht für sich zu behalten, sondern sie an \textit{andere} Gruppen wegzugeben.
Das Ziel ist der \textbf{Tausch}.
Dadurch entstehen Allianzen zwischen Clans. Die Frauen fungieren hier als \enquote{Zeichen}, die zwischen Gruppen zirkulieren, wie Wörter in einem Satz. Das Inzest-Verbot ist also die erste \enquote{Grammatik} der Gesellschaft. Es verwandelt die Natur (biologische Familie) in Kultur (soziales Netz).

\subsection{Das wilde Denken: Mythen denken sich selbst}
Lévi-Strauss' radikalste Erkenntnis betraf die Mythen.
Man hielt die Geschichten der Ureinwohner oft für irrationales Gebrabbel. Lévi-Strauss bewies in seinem Buch \textit{Das wilde Denken} (1962), dass sie einer strengen Logik folgen – oft präziser als unsere Wissenschaft.
Mythen sind Codes. Sie arbeiten mit binären Oppositionen: Roh vs. Gekocht, Natur vs. Kultur, Leben vs. Tod.
Der Mythos versucht, diese Widersprüche logisch zu lösen.
Doch das Unheimliche dabei ist: Der Ureinwohner, der die Geschichte am Lagerfeuer erzählt, \textit{kennt} die Struktur gar nicht. Er hält sich für den Autor. Aber laut Lévi-Strauss folgt er nur einem unbewussten Algorithmus.
Lévi-Strauss sagte dazu den berühmten Satz:
\begin{quote}
    \textit{\enquote{Ich behaupte nicht, dass die Menschen die Mythen denken, sondern dass die Mythen sich in den Menschen denken, ohne dass sie es wissen.}}
\end{quote}
Hier war er vollzogen: der Tod des Subjekts. Nicht das Individuum denkt. Die Struktur denkt durch das Individuum hindurch. Wir sind nur die Resonanzkörper einer uralten Logik.

\section{Roland Barthes: Der Tod des Autors}
% TODO: Mythen des Alltags (Wein, Citroen DS).

\section{Michel Foucault: Das Gesicht im Sand}
% TODO: Archäologie des Wissens.
% TODO: Macht und Disziplin.
