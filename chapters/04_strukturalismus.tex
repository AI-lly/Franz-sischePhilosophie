\chapter{Der Tod des Subjekts: Die Ära des Strukturalismus}

\section{Einleitung: Der kalte Wind der Struktur}
In den 1960er Jahren änderte sich das intellektuelle Klima in Paris radikal.
War die Ära Sartres noch geprägt von \enquote{Hitze}, Rauch, Jazzkellern und leidenschaftlichen Appellen an die Freiheit, so brachte das neue Jahrzehnt eine \textbf{technokratische Kälte}.
Der Existentialismus hatte den Menschen (das Subjekt) auf den Thron gesetzt. Er war heroisch, tragisch und moralisch.
Doch nun traten Denker auf den Plan, die diesen Heroismus für naiv hielten. Sie trugen keine Baskenmützen, sondern Anzüge. Sie sprachen nicht von \enquote{Angst} oder \enquote{Ekel}, sondern von \enquote{Codes}, \enquote{Verwandtschaftssystemen} und \enquote{Diskursen}.
Der \textbf{Strukturalismus} war geboren.

Sein Angriff galt dem Heiligtum der westlichen Philosophie: dem \textbf{Subjekt}.
Seit Descartes galt das \enquote{Ich denke} als das unerschütterliche Zentrum der Welt. Sartre hatte dieses \enquote{Ich} ins Absolute gesteigert: Ich bin der Schöpfer meiner Werte.
Die Strukturalisten drehten den Spieß um. Ihre These war ein Schock:
\textbf{Nicht wir erschaffen die Welt. Die Strukturen erschaffen uns.}
Wir glauben, frei zu sprechen, aber wir folgen nur den Regeln der Grammatik. Wir glauben, wen wir lieben, sei unsere Wahl, aber es folgt unsichtbaren Verwandtschaftsregeln. Wir glauben, unsere Gedanken seien unsere eigenen, aber sie sind das Produkt einer historischen Epoche.
Der Slogan lautete nicht mehr \enquote{Ich denke}, sondern \enquote{Es denkt in mir} (Ça pense).
Das Subjekt ist nicht der Herr im eigenen Haus. Es ist nur ein Schnittpunkt von Linien, ein Effekt von Systemen, die es weder kontrolliert noch versteht.
Wie ein Geologe, der die Schichten unter der Landschaft untersucht, wollten die Strukturalisten die unsichtbaren Gitter (Grilles) freilegen, die unser Denken, Fühlen und Handeln determinieren. Es war der Abschied vom Humanismus und der Aufbruch in eine \enquote{Kälte der Hellsichtigkeit}.
Treibende Kraft hinter diesem Umbruch war eine neue Faszination für die Sprache. Inspiriert vom Linguisten Ferdinand de Saussure, begannen diese Denker, die Gesellschaft nicht mehr historisch zu lesen, sondern wie einen Text. Die große Entdeckung lautete: Bedeutung entsteht nicht durch die Dinge selbst oder durch das Bewusstsein des Einzelnen, sondern allein durch die Differenz der Zeichen im System. Wer die Kultur verstehen wollte, durfte nicht mehr auf die Geschichte (die Diachronie) schauen, wie es Hegel und Marx getan hatten, sondern musste die starren, zeitlosen Regeln (die Synchronie) freilegen, die im Hintergrund operieren.

Dieser Ansatz versprach endlich das, was der Philosophie lange gefehlt hatte: wissenschaftliche Exaktheit. Man wollte weg vom schwammigen "Erleben" des Existentialismus hin zu einer mathematischen Präzision der Geisteswissenschaften.

\section{Die Revolution der Sprache: Saussure als Fundament}
Eigentlich hatte der Strukturalismus keinen philosophischen Vater, sondern einen linguistischen.
Alles begann mit \textbf{Ferdinand de Saussure} (1857–1913), einem Schweizer Sprachwissenschaftler, dessen Vorlesungen posthum als \textit{Grundfragen der allgemeinen Sprachwissenschaft} (1916) erschienen.
Saussure revolutionierte unser Verständnis von Sprache – und damit vom Denken selbst.
Seine Kernidee: Sprache ist keine Liste von Namen für Dinge (Nomenklatur). Sie ist ein geschlossenes System von Zeichen.

\begin{tcolorbox}[title=Das sprachliche Zeichen]
Saussure zerlegte das \enquote{Wort} in zwei untrennbare Seiten (wie Vorder- und Rückseite eines Blattes Papier):
\begin{itemize}
    \item \textbf{Das Signifikat (Signifié):} Die Vorstellung / das Konzept (z.B. die geistige Idee eines Baumes).
    \item \textbf{Der Signifikant (Signifiant):} Das Lautbild / das geschriebene Wort (z.B. die Buchstaben B-A-U-M).
\end{itemize}
Der Clou: Die Verbindung zwischen beiden ist absolut \textbf{willkürlich} (arbiträr).
Das bedeutet: Das Lautbild \enquote{B-A-U-M} hat keinerlei innere Verbindung zu dem hölzernen Ding in der Natur.
Der Beweis ist simpel: Wäre der Name naturnotwendig mit der Sache verbunden, gäbe es weltweit nur eine einzige Sprache. Dass die Engländer \textit{tree} und die Franzosen \textit{arbre} sagen, beweist, dass das Zeichen eine reine \textbf{Konvention} ist.
\end{tcolorbox}

Wenn die Zeichen nun aber willkürlich sind (also keinen inneren Wert haben), woher wissen wir dann, was sie bedeuten?
Saussures geniale Antwort: \textbf{Bedeutung entsteht nur aus der Differenz.}
Wir erkennen ein Wort nicht an seinem Inhalt, sondern daran, dass es sich von anderen Wörtern unterscheidet.
Ein Beispiel: Das Wort \enquote{Nacht} bedeutet nichts anderes als \enquote{Nicht-Tag}.
Denken wir an das Farbspektrum. In der Natur gehen Farben fließend ineinander über. Es gibt keine Linie, wo \enquote{Blau} aufhört und \enquote{Grün} anfängt.
Die Sprache zieht diese Linien willkürlich ein. Wenn eine Sprache das Wort \enquote{Türkis} nicht kennt, dann gehört diese Nuance eben noch zu \enquote{Blau}.
Das heißt: Die Bedeutung eines Begriffs hängt untrennbar von den Grenzen zu seinen Nachbarn ab. Verschiebt sich eine Grenze, ändert sich das ganze System.

\paragraph{Die Schach- und Zug-Metapher}
Um zu verstehen, warum die \textbf{Struktur} wichtiger ist als der Inhalt, nutzte Saussure zwei berühmte Bilder:
\begin{enumerate}
    \item \textbf{Der Zug von Genf nach Paris (8:20 Uhr):} Dieser Zug ist eine feste Einheit im Fahrplan. Aber physikalisch ist er jeden Tag anders: Andere Waggons, andere Lok, anderes Personal. Trotzdem ist es \enquote{derselbe} Zug. Warum? Weil er durch seine \textbf{Position im System} (Fahrplan) definiert ist und sich von dem Zug um 9:30 Uhr unterscheidet.
    \item \textbf{Das Schachspiel:} Ob der Springer aus Holz oder Elfenbein ist, ist egal (Substanz). Wichtig ist nur, wie er ziehen darf und wie er sich vom Läufer unterscheidet (Form/Struktur).
\end{enumerate}
\subsection*{Die philosophische Konsequenz: Das Subjekt als Funktion}
Die philosophische Tragweite dieser linguistischen These war verheerend für den klassischen Humanismus.
Wenn Bedeutung nicht in den Dingen selbst liegt und auch nicht im Kopf des Individuums entsteht, sondern ausschließlich im \textit{System der Unterschiede}, dann verliert das Subjekt seine Macht.
\begin{itemize}
    \item \textbf{Die Dezentrierung:} Wir sind nicht die \enquote{Autoren} unserer Sprache. Wir treten bei der Geburt in ein riesiges, vorgefertigtes Netz aus Differenzen ein. Dieses Netz (die Sprache) war vor uns da und wird nach uns da sein.
    \item \textbf{Das Raster der Welt:} Bevor die Sprache die Welt in Begriffe wie \enquote{Fluss}, \enquote{Bach} und \enquote{Strom} unterteilt, ist das Wasser nur eine formlose Masse. Die Sprache legt ein \textbf{Raster} über die Realität. Wir sehen die Welt nicht \enquote{an sich}, sondern wir sehen sie durch die Gitterstäbe unserer Grammatik und unseres Wortschatzes.
\end{itemize}
Der Existentialist Sartre rief: \enquote{Ich gebe den Dingen ihren Sinn!}
Der Strukturalist antwortet nüchtern: \enquote{Nein. Du bedienst nur die Apparatur einer Sprache, die dich längst definiert hat.} Das \enquote{Ich} ist keine freie Schöpferkraft mehr, sondern eine \textbf{Funktion der Struktur}.

\paragraph{Der Funke springt über: Von der Sprache zur Kultur}
Hier geschah die entscheidende historische Wende.
In den 1940er und 50er Jahren erkannte ein junger Ethnologe das explosive Potenzial dieser These: \textbf{Claude Lévi-Strauss}.
Er stellte eine radikale Frage:
Wenn die Sprache ein System aus unbewussten Regeln ist, das durch binäre Oppositionen (Hell/Dunkel, Singular/Plural) funktioniert – gilt das dann nicht auch für den Rest der Kultur?
Funktionieren unsere Heiratsregeln, unsere Tischsitten und unsere Mythen nicht genau wie eine Sprache?
Lévi-Strauss wagte den Versuch, die Gesellschaft nicht mehr als Historiker zu lesen (Wer tat was?), sondern als Linguist (Welche unbewusste Grammatik steuert das Ganze?).
Damit verließ der Strukturalismus den Hörsaal der Linguisten und eroberte die Welt der Ethnologie.

\section{Claude Lévi-Strauss: Die wilden Mythen}
Der Mann, der den Strukturalismus zur Weltmacht führte, war kein Philosoph, sondern ein Ethnologe: \textbf{Claude Lévi-Strauss} (1908–2009).
Sein Buch \textit{Traurige Tropen} (1955) ist eines der großen melancholischen Meisterwerke des 20. Jahrhunderts. Es beginnt mit dem berühmten Satz: \enquote{Ich hasse Reisen und Forschungsreisende.}
Lévi-Strauss war fasziniert von den indigenen Völkern im Amazonas (wie den Nambikwara oder Bororo). Doch er suchte dort nicht nach exotischen Abenteuern, sondern nach etwas viel Tieferem: der \textbf{universalen Grammatik der Menschheit}.

Sein Ansatz, die \textbf{Strukturale Anthropologie}, war eine Revolution.
Frühere Ethnologen sammelten Masken, Tänze und Rituale wie Kuriositäten in einem Museum. Lévi-Strauss aber interessierte sich nicht für den \textit{Inhalt} der Bräuche, sondern für ihre \textit{Logik}.
Er fragte: Wenn der menschliche Geist überall auf der Welt gleich gebaut ist (dasselbe Gehirn), müssen dann nicht auch alle Kulturen denselben verborgenen Gesetzmäßigkeiten folgen?
Wie ein Chemiker, der nach dem Periodensystem der Elemente sucht, wollte Lévi-Strauss das \enquote{Periodensystem der Kultur} finden. Er glaubte, dass hinter dem bunten Chaos der Völker eine strikte, \textbf{unbewusste Algebra} am Werk ist.
Kochen, Heiraten, Tätowieren und Sagen sind keine zufälligen Launen. Sie sind Codes, die wir benutzen, um die Welt zu ordnen. Und genau wie wir sprechen lernen, ohne die Grammatikregeln explizit zu kennen, so leben wir in unserer Kultur, ohne den \enquote{Code} zu kennen, der uns steuert.

\paragraph{Die phonologische Revolution: Kultur als Kristall} Um Lévi-Strauss zu verstehen, muss man begreifen, was er aus der Linguistik importierte. 
Er war fasziniert von der \textit{Phonologie} (der Lehre von den Lauten). Ein einzelner Laut (z.,B. das \enquote{P}) hat keine Bedeutung. Er gewinnt seine Funktion erst im Kontrast zu einem anderen Laut (z.,B. \enquote{B}). 
Bedeutung entsteht durch \textbf{Differenz}. Lévi-Strauss übertrug dies auf die Kultur: Eine einzelne Maske, ein einzelnes Ritual oder eine Verwandtschaftsbezeichnung bedeutet für sich genommen nichts. Man muss sie als Teile eines Systems betrachten, in dem alles miteinander zusammenhängt wie in einem Kristallgitter. 
Er suchte nach den \enquote{unbewussten Invarianten}. Wie ein Kantianer, der nicht nach den Dingen an sich fragt, sondern nach den Bedingungen der Möglichkeit, fragte Lévi-Strauss: Welche mentalen Strukturen ermöglichen es dem Menschen überhaupt, Gesellschaft zu bilden?

\subsection{Die Elementaren Strukturen: Frauen als Worte} 
Sein erster großer Coup war die Entschlüsselung der Verwandtschaftssysteme. Die Ethnologie war zuvor ein Sammelsurium kurioser Heiratsregeln. Lévi-Strauss brachte Ordnung in das Chaos, indem er das \textbf{Inzest-Verbot} als den "Nullpunkt der Kultur" identifizierte. 
Warum ist der Inzest universal verboten? Weder aus biologischen Gründen (die Indigenen kannten keine Genetik) noch aus moralischen. Der Grund ist rein \textit{strukturell}: Das Verbot ist ein Gebot. Indem ich meine Schwester \textit{nicht} heirate, biete ich sie einem anderen Mann an. Dies erzwingt den \textbf{Tausch} (Exogamie). 
In der radikalen, oft kritisierten Lesart von Lévi-Strauss zirkulieren Frauen in archaischen Gesellschaften wie \textit{Wörter} in einer Sprache. \begin{itemize} \item Wörter werden getauscht, um \textit{Sinn} zu produzieren (Kommunikation). \item Frauen werden getauscht, um \textit{Allianzen} zu produzieren (Gesellschaft). \end{itemize} 
Die Gesellschaft ist also kein Vertrag freier Individuen (wie bei Rousseau), sondern ein Tauschsystem, das im Rücken der Akteure operiert.

\subsection{Das wilde Denken: Ingenieur vs. Bricoleur}
In seinem Werk \textit{Das wilde Denken} (1962) rehabilitierte Lévi-Strauss die geistige Welt der Indigenen. Er räumte mit dem Vorurteil auf, sogenannte \enquote{Primitive} würden nur triebhaft oder irrational denken.
Im Gegenteil: Das \enquote{wilde Denken} ist besessen von Ordnung, Klassifikation und Logik.
Es unterscheidet sich vom modernen wissenschaftlichen Denken nicht durch die Qualität der Logik, sondern durch die Art des Materials.
Lévi-Strauss prägte hierfür die berühmte Unterscheidung zwischen dem \textbf{Ingenieur} und dem \textbf{Bricoleur} (Bastler):
\begin{itemize}
    \item \textbf{Der Ingenieur (Moderne Wissenschaft):} Er arbeitet mit abstrakten Konzepten. Wenn er ein Problem hat, entwickelt er ein neues Werkzeug, das exakt dafür passt.
    \item \textbf{Der Bricoleur (Mythisches Denken):} Er arbeitet mit dem, was \enquote{zur Hand} ist (dem Bestand). Er nimmt Reste alter Mythen, Beobachtungen von Tieren und Pflanzen und setzt sie neu zusammen, um die Welt zu erklären.
\end{itemize}
Das Resultat ist eine \textbf{\enquote{Wissenschaft des Konkreten}}.
Wenn Indigene Tiere als Totems wählen (z.B. Adler-Clan vs. Bären-Clan), tun sie das nicht, weil die Tiere gut zu essen sind (Utilitarismus), sondern weil sie \textbf{gut zu denken} sind (\textit{bonnes à penser}).
Die Unterschiede in der Natur (der Adler fliegt, der Bär läuft) dienen als perfektes logisches Modell, um die sozialen Unterschiede in der Kultur (Clan A heiratet Clan B) abzubilden.

\subsection{Die Logik des Mythos: Das Rohe und das Gekochte}
Sein monumentalstes Werk sind die \textit{Mythologica} (4 Bände). Hier analysiert er über 800 Mythen der amerikanischen Ureinwohner und zeigt, dass sie alle Variationen eines einzigen, riesigen mentalen Netzwerks sind.
Mythen sind keine chaotischen Märchen. Sie sind logische Maschinen, die versuchen, die grundlegenden Widersprüche der menschlichen Existenz (Leben/Tod, Natur/Kultur) zu lösen.
Das zentrale Symbol dieser Vermittlung ist das Kochen.

\begin{tcolorbox}[title=Das Kulinarische Dreieck]
Für Lévi-Strauss markiert das Kochen den Übergang vom Tier zum Menschen. Es verwandelet die Natur in Kultur.
Er ordnet die Zustände der Nahrung in einem Dreieck an:
\begin{itemize}
    \item \textbf{Das Rohe:} Der Zustand der reinen Natur (unberührt).
    \item \textbf{Das Gekochte:} Die kulturelle Transformation des Rohen (durch Feuer).
    \item \textbf{Das Verrottete:} Die natürliche Transformation des Rohen (durch Zeit/Verfall).
\end{itemize}
Jeder Mythos der Welt lässt sich auf Variationen dieser Grundformel zurückführen.
\end{tcolorbox}

Lévi-Strauss liest diese Mythen wie eine \textbf{Orchesterpartitur}.
Man muss die einzelnen Geschichten übereinanderlegen, um die harmonische Struktur zu erkennen.
Dabei formulierte er den Satz, der den Tod des Subjekts endgültig besiegelte:
\begin{quote}
    \textit{\enquote{Wir beanspruchen also nicht, zu zeigen, wie die Menschen in den Mythen denken, sondern wie die Mythen sich in den Menschen denken, und zwar ohne dass ihnen das bewusst ist.}}
\end{quote}
Der Mensch ist nicht der Autor seiner Geschichten. Die Struktur denkt durch das Individuum hindurch. Wir sind nur die Resonanzkörper einer uralten Logik.

\section{Roland Barthes: Der Tod des Autors}
% TODO: Mythen des Alltags (Wein, Citroen DS).

\section{Michel Foucault: Das Gesicht im Sand}
% TODO: Archäologie des Wissens.
% TODO: Macht und Disziplin.
