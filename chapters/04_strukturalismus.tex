\chapter{Der Tod des Subjekts: Die Ära des Strukturalismus}

\section{Einleitung: Der kalte Wind der Struktur}
In den 1960er Jahren änderte sich das intellektuelle Klima in Paris radikal.
War die Ära Sartres noch geprägt von \enquote{Hitze}, Rauch, Jazzkellern und leidenschaftlichen Appellen an die Freiheit, so brachte das neue Jahrzehnt eine \textbf{technokratische Kälte}.
Der Existentialismus hatte den Menschen (das Subjekt) auf den Thron gesetzt. Er war heroisch, tragisch und moralisch.
Doch nun traten Denker auf den Plan, die diesen Heroismus für naiv hielten. Sie trugen keine Baskenmützen, sondern Anzüge. Sie sprachen nicht von \enquote{Angst} oder \enquote{Ekel}, sondern von \enquote{Codes}, \enquote{Verwandtschaftssystemen} und \enquote{Diskursen}.
Der \textbf{Strukturalismus} war geboren.

Sein Angriff galt dem Heiligtum der westlichen Philosophie: dem \textbf{Subjekt}.
Seit Descartes galt das \enquote{Ich denke} als das unerschütterliche Zentrum der Welt. Sartre hatte dieses \enquote{Ich} ins Absolute gesteigert: Ich bin der Schöpfer meiner Werte.
Die Strukturalisten drehten den Spieß um. Ihre These war ein Schock:
\textbf{Nicht wir erschaffen die Welt. Die Strukturen erschaffen uns.}
Wir glauben, frei zu sprechen, aber wir folgen nur den Regeln der Grammatik. Wir glauben, wen wir lieben, sei unsere Wahl, aber es folgt unsichtbaren Verwandtschaftsregeln. Wir glauben, unsere Gedanken seien unsere eigenen, aber sie sind das Produkt einer historischen Epoche.
Der Slogan lautete nicht mehr \enquote{Ich denke}, sondern \enquote{Es denkt in mir} (Ça pense).
Das Subjekt ist nicht der Herr im eigenen Haus. Es ist nur ein Schnittpunkt von Linien, ein Effekt von Systemen, die es weder kontrolliert noch versteht.
Wie ein Geologe, der die Schichten unter der Landschaft untersucht, wollten die Strukturalisten die unsichtbaren Gitter (Grilles) freilegen, die unser Denken, Fühlen und Handeln determinieren. Es war der Abschied vom Humanismus und der Aufbruch in eine \enquote{Kälte der Hellsichtigkeit}.
Treibende Kraft hinter diesem Umbruch war eine neue Faszination für die Sprache. Inspiriert vom Linguisten Ferdinand de Saussure, begannen diese Denker, die Gesellschaft nicht mehr historisch zu lesen, sondern wie einen Text. Die große Entdeckung lautete: Bedeutung entsteht nicht durch die Dinge selbst oder durch das Bewusstsein des Einzelnen, sondern allein durch die Differenz der Zeichen im System. Wer die Kultur verstehen wollte, durfte nicht mehr auf die Geschichte (die Diachronie) schauen, wie es Hegel und Marx getan hatten, sondern musste die starren, zeitlosen Regeln (die Synchronie) freilegen, die im Hintergrund operieren.

Dieser Ansatz versprach endlich das, was der Philosophie lange gefehlt hatte: wissenschaftliche Exaktheit. Man wollte weg vom schwammigen "Erleben" des Existentialismus hin zu einer mathematischen Präzision der Geisteswissenschaften.

\section{Die Revolution der Sprache: Saussure als Fundament}
Eigentlich hatte der Strukturalismus keinen philosophischen Vater, sondern einen linguistischen.
Alles begann mit \textbf{Ferdinand de Saussure} (1857–1913), einem Schweizer Sprachwissenschaftler, dessen Vorlesungen posthum als \textit{Grundfragen der allgemeinen Sprachwissenschaft} (1916) erschienen.
Saussure revolutionierte unser Verständnis von Sprache – und damit vom Denken selbst.
Seine Kernidee: Sprache ist keine Liste von Namen für Dinge (Nomenklatur). Sie ist ein geschlossenes System von Zeichen.

\begin{tcolorbox}[title=Das sprachliche Zeichen]
Saussure zerlegte das \enquote{Wort} in zwei untrennbare Seiten (wie Vorder- und Rückseite eines Blattes Papier):
\begin{itemize}
    \item \textbf{Das Signifikat (Signifié):} Die Vorstellung / das Konzept (z.B. die geistige Idee eines Baumes).
    \item \textbf{Der Signifikant (Signifiant):} Das Lautbild / das geschriebene Wort (z.B. die Buchstaben B-A-U-M).
\end{itemize}
Der Clou: Die Verbindung zwischen beiden ist absolut \textbf{willkürlich} (arbiträr).
Es gibt keinen Grund, warum das Tier \enquote{Hund} heißt und nicht \enquote{Wuff} oder \enquote{Glump}. Die Sprache ist eine Konvention.
\end{tcolorbox}

Wenn die Zeichen willkürlich sind, woher kommt dann ihre Bedeutung?
Saussures Antwort: \textbf{Aus der Differenz.}
Ein Wort hat keine Bedeutung aus sich selbst heraus, sondern nur, weil es \textit{anders} ist als die anderen Wörter im System.
\enquote{Heiß} bedeutet nur deshalb heiß, weil es nicht \enquote{kalt} oder \enquote{lauwarm} ist. In der Sprache gibt es nur Unterschiede ohne positive Terme.

\paragraph{Die Schach-Metapher}
Saussure verglich die Sprache mit einem Schachspiel.
Ob die Figur des Springers aus Holz, Elfenbein oder Glas ist, ist völlig egal (das ist die Substanz/Essenz).
Was den Springer zum Springer macht, ist einzig seine \textbf{Funktion im System}: Er zieht im L-Sprung und steht in Relation zum Bauern und zum König.
Die \textbf{Struktur} (die Spielregel) bestimmt die Identität der Figur.
Die Strukturalisten übertrugen diese Erkenntnis auf den Menschen:
Auch der Mensch hat keine \enquote{Essenz} (wie Sartre glaubte). Er ist wie eine Schachfigur. Seine Identität entsteht nur durch seine Position im sozialen Netz (Vater, Arbeiter, Bürger).
Nicht das Subjekt macht die Züge. Das System (die Sprache, die Kultur) spielt mit uns.
Claude Lévi-Strauss war der erste, der dieses linguistische Dynamit nahm und damit die ganze Kulturgeschichte sprengte.

\section{Claude Lévi-Strauss: Die wilden Mythen}
% TODO: Die Traurigen Tropen.
% TODO: Mythen denken sich selbst. Inzest-Tabu.

\section{Roland Barthes: Der Tod des Autors}
% TODO: Mythen des Alltags (Wein, Citroen DS).

\section{Michel Foucault: Das Gesicht im Sand}
% TODO: Archäologie des Wissens.
% TODO: Macht und Disziplin.
