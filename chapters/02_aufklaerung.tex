\chapter{Die Aufklärung (Les Lumières): Der Kampf um die Freiheit}

Wenn das 17. Jahrhundert das \enquote{Zeitalter der Systeme} war, in dem Descartes und Pascal einsam in ihren Studierstuben über Gott und die Seele nachdachten, so ist das 18. Jahrhundert das \enquote{Zeitalter der Kritik}. Die Philosophie verlässt den elfenbeinernen Turm der Metaphysik und betritt die lauten Salons, die Kaffeehäuser und schließlich die Straße.
Das Werkzeug, das Descartes geschmiedet hatte – die autonome Vernunft, die nichts als wahr akzeptiert, was sie nicht selbst geprüft hat –, wird nun zur Waffe. Aber sie richtet sich nicht mehr gegen Zweifel und Dämonen, sondern gegen konkrete irdische Mächte: Gegen den Absolutismus des Königs, gegen die Dogmen der Kirche und gegen die Unmündigkeit des Bürgers.

\section{Vom metaphysischen zum politischen Zweifel}
Wir haben im vorangegangenen Kapitel gesehen, wie sich die französische Philosophie in zwei Ströme teilte:
\begin{itemize}
    \item Den \textbf{Rationalismus} (Descartes), der an die Macht der Vernunft glaubt, Ordnung zu schaffen.
    \item Die \textbf{Existenzanalyse} (Pascal), die das Elend des Menschen und die Grenzen der Machbarkeit betont.
\end{itemize}
In der Aufklärung prallen diese beiden Haltungen erneut aufeinander, jedoch politisch gewendet. Auf der einen Seite steht \textbf{Voltaire}, der Erbe des rationalen Optimismus, der glaubt, dass Vernunft, Wissenschaft und Toleranz die Welt zivilisieren können. Auf der anderen Seite steht \textbf{Jean-Jacques Rousseau}, der Erbe Pascals, der spürt, dass der Fortschritt der Zivilisation den Menschen nicht glücklicher, sondern moralisch verdorbener und unfreier gemacht hat.

Diese Epoche nennt man in Frankreich \enquote{Les Lumières} (Die Lichter) – im Plural. Es ist nicht das eine, monolithische Licht einer einzigen Wahrheit, sondern es sind viele Lichter, die in die dunklen Ecken des Aberglaubens, der Justizwillkür und der religiösen Intoleranz leuchten sollen.

\begin{definitionbox}[Begriff: Der Philosophe]
Der Begriff des Philosophen wandelt sich im 18. Jahrhundert radikal. Er ist nicht mehr der weltabgewandte Grübler (wie Descartes im Ofenzimmer), sondern ein \textbf{l'homme engagé} (ein engagierter Mensch).
Der \textit{Philosophe} der Aufklärung ist ein Publizist, ein Kämpfer, ein intellektueller Unruhestifter. Er schreibt keine dicken lateinischen Wälzer mehr, sondern Romane, Theaterstücke, Flugblätter und Enzyklopädie-Artikel. Sein Ziel ist nicht mehr nur die \textit{Erkenntnis} der Welt, sondern ihre \textit{Verbesserung}.
\end{definitionbox}

Das zentrale Projekt dieser Zeit ist die \textbf{Encyclopédie} von Diderot und d'Alembert: Der Versuch, das gesamte Wissen der Menschheit zu sammeln und demokratisch zugänglich zu machen, um so die Macht der alten Eliten (Klerus und Adel), die auf Geheimwissen und Tradition basierte, zu brechen. Wissen ist nicht mehr nur Macht, Wissen ist Freiheit.

\section{Voltaire: Die Waffe des Spottes}
Wenn es einen Menschen gibt, der das 18. Jahrhundert verkörpert, dann ist es \textbf{François-Marie Arouet}, genannt \textbf{Voltaire} (1694–1778). Er war mehr als nur ein Schriftsteller; er war eine Institution, der \enquote{ungekrönte König Europas}.
Voltaire war kein systematischer Philosoph wie Kant oder Descartes, der im stillen Kämmerlein Kathedralen aus Begriffen baute. Er war ein Mann der Welt, ein Polemiker, ein \textit{Großbürger}, der wusste, wie man Reichtum anhäuft und öffentliche Meinung manipuliert.
Seine Waffe war nicht der schwere Syllogismus, sondern der \textbf{Esprit}: Der geschliffene Witz, die Ironie und der schneidende Sarkasmus. Er erkannte früh: Man stürzt Autoritäten am effektivsten, indem man sie nicht widerlegt, sondern lächerlich macht. Während Descartes den Zweifel methodisch nutzte, nutzt Voltaire das Lachen politisch.  

Voltaires Denken entspringt einer tiefen persönlichen Demütigung. Als junger, brillanter Dichter legte er sich mit dem arroganten Chevalier de Rohan an. Als Voltaire diesen intellektuell bloßstellte, ließ der Aristokrat ihn kurzerhand von seinen Dienern verprügeln und nutzte seine Verbindungen, um Voltaire in die \textbf{Bastille} werfen zu lassen.
Diese Erfahrung der absoluten Rechtlosigkeit gegenüber dem Adel prägte ihn für immer. Er wählte das Exil und ging 1726 nach \textbf{England}.

Dort erlebte er einen Kulturschock, der die französische Aufklärung zünden sollte. Er sah ein Land, in dem:
\begin{itemize}
    \item Händler und Bürger respektiert wurden (während sie in Frankreich verachtet waren),
    \item religiöse Vielfalt herrschte (Quäker, Anglikaner, Presbyterianer lebten friedlich nebeneinander),
    \item und Denker wie \textbf{John Locke} und \textbf{Isaac Newton} wie Helden verehrt wurden (statt zensiert zu werden).
\end{itemize}

Voltaire kehrte mit einer intellektuellen Bombe im Gepäck zurück: Den \textit{Lettres philosophiques} (Philosophische Briefe). In ihnen pries er die englische Freiheit, um die französische Tyrannei subtil, aber vernichtend zu kritisieren. Das Buch wurde vom Henker öffentlich verbrannt, aber der Geist war aus der Flasche. Voltaire zog sich später auf sein Gut \textbf{Ferney} an der Schweizer Grenze zurück – strategisch klug, um bei drohender Verhaftung schnell fliehen zu können – und wurde von dort aus zur moralischen Instanz Europas.

\subsection{Der Kampf gegen das Ungeheuer: \enquote{Écrasez l'infâme!}}
Doch Voltaire begnügte sich nicht damit, die englische Toleranz nur theoretisch zu loben. Von seinem sicheren Rückzugsort in Ferney aus startete er eine publizistische Offensive, die das geistige Klima Europas verändern sollte. Sein Ziel war nichts Geringeres als die Zerschlagung des religiösen Fanatismus.

Sein Schlachtruf, mit dem er hunderte seiner Briefe beendete, lautete: \textit{\enquote{Écrasez l'infâme!}} (Zermalmt das Niederträchtige/das Scheusal!).
Was genau meinte Voltaire mit diesem \enquote{Ungeheuer}? Es wäre falsch, dies als Angriff auf den Glauben an sich zu verstehen. \enquote{L'infâme} bezeichnete die unheilige Allianz aus Aberglauben, dogmatischem Starrsinn und klerikaler Macht, die Andersdenkende verfolgte. Es ist der Geist der Inquisition, der den Diskurs mit dem Scheiterhaufen beendet.

\subsubsection*{Die Affäre Calas: Der Philosoph als Anwalt}
Das dunkelste Beispiel für diesen Geist – und Voltaires hellster Moment – ist die \textbf{Affäre Calas} (1762). Jean Calas, ein rechtschaffener protestantischer Tuchhändler in Toulouse, wurde beschuldigt, seinen Sohn ermordet zu haben, um dessen Konversion zum Katholizismus zu verhindern. In einem Schauprozess, getrieben von hysterischem Volkszorn und religiösen Vorurteilen, wurde Calas zum Tode verurteilt und auf dem Rad lebendig zerbrochen.
Voltaire, damals schon 68 Jahre alt, war entsetzt. Er sah hier nicht nur einen Justizirrtum, sondern ein systemisches Versagen. Er startete eine beispiellose Kampagne: Er schrieb Briefe an die Monarchen Europas, veröffentlichte Flugblätter und verfasste den berühmten \textit{Traité sur la tolérance} (Abhandlung über die Toleranz).
Darin dekonstruierte er die Prozessakten mit der Präzision eines Kriminalisten und der Wucht eines Propheten. Er zwang den französischen König, das Urteil posthum aufzuheben.
Dies ist der Geburtsmoment des modernen Intellektuellen (\textit{l'intellectuel engagé}): Die Philosophie verlässt den akademischen Elfenbeinturm und wird zur moralischen Kontrollinstanz der Macht, die im Namen der Menschlichkeit Gerechtigkeit einfordert.

Doch dieser Kampf gegen die Kirche bedeutete keineswegs einen Kampf gegen Gott. Im Gegenteil: Voltaire bekämpfte die Priester gerade deshalb, weil sie das wahre Bild Gottes verzerrten. Dies führt uns zum Kern seiner eigenen Theologie – einer Religion der Vernunft, die ohne Dogmen auskommt.

\subsection{Theologie: Der Uhrmacher-Gott (Deismus)}
War Voltaire Atheist? Diese Frage wurde oft gestellt, und die Antwort ist ein klares Nein. Voltaire führte einen Zweifrontenkrieg: Gegen den fanatischen Klerikalismus auf der einen Seite und gegen den radikalen Materialismus (wie ihn später Diderot vertreten würde) auf der anderen Seite.
Sein theologisches Modell ist der \textbf{Deismus} (oder Theismus). Es ist der Versuch, Gott zu retten, indem man ihn von der Kirche befreit.

Voltaire unterscheidet streng zwischen:
\begin{itemize}
    \item \textbf{Die geoffenbarte Religion:} Das sind die historischen Dogmen, Rituale und Wunderberichte. Diese hält Voltaire für menschliche Erfindungen, voll von Widersprüchen und Ursache für Kriege.
    \item \textbf{Die natürliche Religion:} Das ist die rationale Einsicht, dass es eine höchste Intelligenz geben muss. Diese Einsicht ist universell; jeder vernünftige Mensch (ob Chinese, Inder oder Europäer) kann sie teilen.
\end{itemize}

\begin{intuition}[Metapher: Der Große Uhrmacher (Le Grand Horloger)]
Voltaire argumentiert nicht metaphysisch (wie Descartes), sondern empirisch-physikalisch, inspiriert von Isaac Newton.
Sein berühmtes Argument lautet: \textit{\enquote{Das Universum macht mich verlegen, und ich kann mir nicht denken, dass diese Uhr existiert und keinen Uhrmacher hat.}}
Wenn man eine perfekt funktionierende Uhr im Wald findet, schließt man zwingend auf einen Uhrmacher. Der Kosmos ist unendlich komplexer als eine Uhr; ergo muss es eine ordnende Intelligenz geben.
Doch dieser Gott ist fern. Er hat die Naturgesetze installiert und die Maschine in Gang gesetzt, aber er greift nicht mehr ein. Er wirkt keine Wunder (denn das würde bedeuten, dass er seine eigenen perfekten Gesetze korrigieren müsste). Er ist ein Gott der Vernunft, nicht des Gebets.
\end{intuition}

Neben diesem kosmologischen Argument führt Voltaire ein pragmatisch-soziales Argument an. Er fürchtete, dass ohne den Glauben an eine strafende Instanz die Moral zusammenbrechen würde. Wenn der Pöbel glaubt, es gäbe keinen Gott, dann gibt es auch keine Gerechtigkeit. Daraus resultiert sein berühmtes, oft zynisch missverstandenes Diktum:
\textit{\enquote{Si Dieu n'existait pas, il faudrait l'inventer.}} (Wenn Gott nicht existierte, müsste man ihn erfinden).
Gott ist notwendig als Garant der sozialen Ordnung. Der \enquote{Philosophen-Gott} mag fern sein, aber der \enquote{Polizei-Gott} ist nützlich.

Doch dieses optimistische Weltbild des \enquote{perfekten Uhrwerks} bekam einen Riss. Wenn Gott ein perfekter Uhrmacher ist, warum leiden dann Unschuldige? Dieses Problem der Theodizee wurde für Voltaire durch ein Ereignis zur existenziellen Krise: Das Erdbeben von Lissabon.

\subsection{Candide und das Ende des Optimismus}
Am 1. November 1755, am Allerheiligenfest, zerstörte ein gigantisches Erdbeben Lissabon und tötete Zehntausende Gläubige in den Kirchen. Dieses Ereignis erschütterte den Optimismus des 18. Jahrhunderts bis ins Mark. Wie konnte ein gütiger Uhrmacher-Gott so etwas zulassen?
Leibniz hatte gelehrt, wir lebten in der \enquote{besten aller möglichen Welten} (\textit{le meilleur des mondes possibles}), da Gott logisch nichts Schlechtes schaffen könne. Voltaire empfand diese These angesichts der Leichen von Lissabon als zynischen Hohn.

Seine literarische Antwort ist \textit{Candide oder der Optimismus} (1759), eine beißende Satire und zugleich Voltaires philosophisches Vermächtnis.
Der naive Held Candide reist durch eine Welt, die eine einzige Schlachthalle ist. Er erlebt Krieg, Syphilis, Sklaverei, Schiffbruch und das Autodafé der Inquisition. Doch sein Lehrer, der Philosoph Dr. \textbf{Pangloss} (eine Karikatur von Leibniz), hält stur an seinem Mantra fest: \enquote{Alles ist zum Besten bestellt.} Pangloss symbolisiert die \textit{metaphysische Blindheit}: Er glaubt seinem System mehr als seinen Augen. Er leugnet die Realität des Leidens, um seine Theorie zu retten.

\subsubsection*{Die Lehre des Gartens}
Am Ende der Odyssee, nach zahllosen Katastrophen, landen die Protagonisten in der Türkei. Sie sind arm, desillusioniert und streiten sich weiterhin über die beste aller möglichen Welten.
Da begegnen sie einem alten Türken, der entspannt unter einer Laube von Orangenbäumen sitzt und frische Früchte genießt.
Pangloss, neugierig und geschwätzig wie immer, fragt ihn nach dem Namen eines Muftis, der gerade in Konstantinopel erdrosselt wurde.
Der Alte zuckt mit den Schultern. Er wisse den Namen nicht, antwortet er gelassen, und es kümmere ihn auch nicht, was die Großen in der Hauptstadt treiben. Er wisse nur eines:
\enquote{Ich habe zwanzig Morgen Land, die bebaue ich mit meinen Kindern; die Arbeit hält uns drei große Übel fern: die Langeweile, das Laster und die Not.}

Diese Begegnung ist für Candide eine Offenbarung. Dieser einfache Mann, der Limonen erntet, ist glücklicher als alle Könige und Philosophen, die Candide auf seiner Weltreise getroffen hat.
Als Pangloss kurz darauf wieder anhebt, um eine grandiose metaphysische Kausalkette zu beweisen (\enquote{Denn wenn Du nicht aus dem Schloss gejagt worden wärst und nicht die Inquisition erlebt hättest, würdest Du hier keine eingelegten Zitronen essen...}), schneidet ihm Candide das Wort ab. Er hat begriffen, dass Reden sinnlos ist.
Er antwortet mit dem berühmtesten Satz der Aufklärung:
\textit{\enquote{Cela est bien dit, mais il faut cultiver notre jardin.}} (Das ist gut gesagt, aber wir müssen unseren Garten bestellen).

Was bedeutet diese Metapher? Sie ist eine radikale Absage an die theoretische Spekulation (Metaphysik) zugunsten der praktischen Tat (Arbeit).
\begin{enumerate}
    \item \textbf{Die Ablehnung der Utopie:} Wir können die Welt im Ganzen nicht retten. Die Frage nach dem \enquote{Warum} des Bösen ist unlösbar.
    \item \textbf{Die kurative Kraft der Arbeit:} Die Arbeit hält uns drei große Übel fern: \enquote{Die Langeweile, das Laster und die Not} (\textit{l'ennui, le vice et le besoin}).
    \item \textbf{Der pragmatische Humanismus:} Statt darüber zu debattieren, ob die Welt perfekt ist, sollen wir sie im Kleinen, in unserem direkten Wirkungskreis (\enquote{unserem Garten}), ein wenig besser machen. Einen Baum pflanzen ist nützlicher als einen Gottesbeweis führen.
\end{enumerate}

Mit Voltaire erreicht die Vernunft ihren pragmatischen Gipfel: Sie wird kritisch, tolerant und weltzugewandt. Doch im Schatten dieses Lichts wuchs bereits eine Gegenbewegung heran, die behauptete, genau dieser Fortschritt sei der Holzweg. Auftritt: Jean-Jacques Rousseau.

\section{Jean-Jacques Rousseau: Die Umkehrung des Fortschritts}
Wenn Voltaire der strahlende \enquote{König der Philosophen} war, so war \textbf{Jean-Jacques Rousseau} (1712–1778) ihr schlechtes Gewissen. Kein Denker der Aufklärung ist widersprüchlicher, emotionaler und folgenreicher gewesen als dieser Uhrmachersohn aus Genf.
Sieht man Voltaire als den Mann des \textit{Esprit}, der in Salons brilliert, so ist Rousseau der Mann des \textit{Gefühls}, der einsam durch die Wälder streift. Er entfremdete sich von allen: Von der Kirche (die seine Bücher verbrannte), vom Staat (der ihn verhaften wollte) und schließlich auch von den \textit{Philosophes} selbst (die er als eitle Schwätzer verachtete).

Rousseaus Denken beginnt nicht mit einem Buch, sondern mit einer Epiphanie. Es ist ein heißer Sommertag im Jahr 1749. Rousseau ist auf dem Weg zum Schloss Vincennes bei Paris, um seinen Freund Diderot zu besuchen, der dort wegen \enquote{Gotteslästerung} im Kerker sitzt. Um sich die Zeit zu vertreiben, liest Rousseau im \textit{Mercure de France} die Preisfrage der Akademie von Dijon:
\textit{\enquote{Hat die Wiederherstellung der Wissenschaften und Künste dazu beigetragen, die Sitten zu reinigen?}}

In diesem Moment bricht es über ihn herein. Er beschreibt es später als einen körperlichen Schock, einen Rausch, der ihn unter einem Baum zusammenbrechen lässt. Er sieht plötzlich eine andere Wahrheit: Nein! Die Zivilisation reinigt uns nicht, sie verdirbt uns. Der Fortschritt ist kein Aufstieg, sondern ein Fall.
Die glänzende Fassade der Pariser Gesellschaft – ihre Höflichkeit, ihr Luxus, ihre Kunst – ist nur eine Maske, hinter der sich Heuchelei, Neid und Sklaverei verbergen. \enquote{Wir haben Physiker, Geometer, Chemiker, Astronomen, Poeten, Musiker, Maler – aber wir haben keine Bürger mehr.}

Mit seiner Antwortschrift, dem \textit{Ersten Diskurs}, gewinnt er den Preis und wird über Nacht berühmt. Doch er bleibt dabei nicht stehen. Er fragt weiter: Wenn die Gesellschaft den Menschen verdorben hat, wie war der Mensch \textit{davor}? Dies führt zu seiner revolutionären Anthropologie.

\subsection{Der Naturzustand und der Ursprung der Ungleichheit}
Rousseaus \enquote{Naturzustand} ist ein missverstandenes Konzept. Es ist keine historische Beschreibung der Steinzeit, sondern ein \textbf{Gedankenexperiment}. Er fragt: Was bleibt vom Menschen übrig, wenn man alles abzieht, was die Gesellschaft ihm antrainiert hat?
Das Ergebnis ist der \textbf{Edle Wilde} (\textit{le bon sauvage}).
Im Gegensatz zu Thomas Hobbes, der diesen Urzustand als \enquote{Krieg aller gegen alle} dämonisierte, sieht Rousseau hier Frieden.
Der natürliche Mensch lebt in einer reinen Gegenwart. Er kennt keine Sorgen um die Zukunft, keinen Ehrgeiz und keine Eitelkeit. Er wird von zwei friedlichen Trieben gesteuert:
\begin{itemize}
    \item \textbf{Amour de soi (Selbstliebe):} Der gesunde Drang zu überleben (wie ein Tier, das frisst und schläft).
    \item \textbf{Pitié (Mitleid):} Eine instinktive Abneigung, andere Wesen leiden zu sehen (eine vor-rationale Empathie).
\end{itemize}

\subsubsection*{Der Sündenfall: Eisen und Weizen}
Wann endete dieses Idyll? Rousseau ist präzise: Mit der Erfindung von Metallurgie und Ackerbau. \enquote{Für den Dichter sind es Gold und Silber, aber für den Philosophen sind es \textbf{Eisen und Weizen}, die den Menschen zivilisiert und das Menschengeschlecht zugrunde gerichtet haben.}
Denn wer Ackerbau betreibt, braucht Boden. Wer Boden braucht, zieht Grenzen.
Hier setzt Rousseaus berühmteste Anklage im \textit{Zweiten Diskurs} an:
\textit{\enquote{Der erste, der ein Stück Land mit einem Zaun umgab und auf den Gedanken kam zu sagen: »Dies gehört mir«, und der Leute fand, die einfältig genug waren, ihm zu glauben, war der eigentliche Begründer der bürgerlichen Gesellschaft. Wie viele Verbrechen, Kriege, Morde, wie viel Elend und Schrecken wäre dem Menschengeschlecht erspart geblieben, wenn jemand die Pfähle herausgerissen hätte...}}

\subsubsection*{Die psychologische Katastrophe: Amour-propre}
Mit dem Eigentum kam der Vergleich. Der Mensch begann, sich nicht mehr nur selbst zu fühlen, sondern sich im Blick der anderen zu spiegeln. Aus der unschuldigen \textit{Amour de soi} wurde die giftige \textbf{Amour-propre} (Eigenliebe/Eitelkeit).
\begin{itemize}
    \item \textit{Amour de soi} ist absolut: Ich bin satt, ich bin zufrieden.
    \item \textit{Amour-propre} ist relativ: Ich bin zufrieden, nur wenn ich \textit{mehr} habe als mein Nachbar.
\end{itemize}
Dieser soziale Vergleich ist die Wurzel allen Übels. Wir leben nicht mehr in uns selbst, sondern \enquote{außer uns}, im Urteil der anderen. Wir tragen Masken, um zu gefallen. Der zivilisierte Mensch ist ein Schauspieler seiner selbst. Die Ungleichheit ist nicht natürlich, sie ist gemacht.

\subsection{Der Weg in die Freiheit: Der Gesellschaftsvertrag}
Rousseau ist kein naiver Primitivist; er fordert nicht \enquote{Zurück zur Natur} im Sinne einer Regression in die Höhle. Da der Weg zurück versperrt ist, muss die Freiheit nicht in der \textit{Wildnis}, sondern im \textit{Gesetz} neu begründet werden.
Dies geschieht in seinem politischen Hauptwerk \textit{Du Contrat Social} (1762). Es beginnt mit der berühmten Diagnose:
\textit{\enquote{Der Mensch ist frei geboren, und überall liegt er in Ketten.}}
Das Ziel ist nicht, die Ketten zu brechen (Anarchie), sondern sie zu legitimieren (Recht).

\subsubsection*{Das fundamentale Problem}
Rousseau sucht die Quadratur des Kreises: Wie findet man eine Gesellschaftsform, die jeden Einzelnen schützt, in der aber \enquote{jeder, indem er sich mit allen vereinigt, nur sich selbst gehorcht und genau so frei bleibt wie zuvor}?
Die Lösung klingt paradox: Durch die \textbf{totale Entäußerung}.
Jeder Bürger muss alle seine natürlichen Rechte restlos an die Gemeinschaft abtreten. Da dies alle tun, ist die Bedingung für alle gleich, und niemand hat ein Interesse daran, sie für andere belastend zu machen.
Der Tausch ist fundamental:
\begin{itemize}
    \item Wir verlieren die \textbf{natürliche Freiheit} (das unbegrenzte Recht auf alles, was uns reizt).
    \item Wir gewinnen die \textbf{bürgerliche Freiheit} (Eigentumsrecht) und die \textbf{sittliche Freiheit} (Gehorsam gegen das Gesetz, das man sich selbst gegeben hat).
\end{itemize}

\subsubsection*{Die Volonté générale (Der Gemeinwille)}
Das Herzstück des Staates ist nicht der Monarch, sondern der \textbf{Volonté générale}. Rousseau unterscheidet hier messerscharf:
\begin{itemize}
    \item \textbf{Volonté de tous (Wille aller):} Eine quantitative Summe von Privatinteressen (das, was heute Meinungsumfragen messen).
    \item \textbf{Volonté générale (Gemeinwille):} Ein qualitativer Konsens, der nur auf das Gemeinwohl zielt. Er entsteht, wenn die Sonderinteressen sich gegenseitig aufheben.
\end{itemize}

Daraus folgen radikale Konsequenzen:
\begin{enumerate}
    \item \textbf{Unveräußerliche Souveränität:} Das Volk darf sich nicht vertreten lassen. Ein Parlament (wie in England) lehnt Rousseau ab: \enquote{Das englische Volk glaubt frei zu sein; es täuscht sich sehr; es ist es nur während der Wahlen.} Echte Demokratie muss direkt sein.
    \item \textbf{Zwang zur Freiheit:} Da der Gemeinwille immer recht hat (er irrt nie über das Wohl des Ganzen), ist der Abweichler im Irrtum über sein eigenes Bestes. Wer sich dem Gemeinwillen verweigert, muss von der Gemeinschaft gezwungen werden zu gehorchen: \textit{\enquote{Das heißt nichts anderes, als dass man ihn zwingt, frei zu sein.}}
\end{enumerate}

\subsubsection*{Die Schattenseite: Von Rousseau zu Robespierre}
Hier zeigt sich die janusköpfige Natur Rousseaus. Einerseits ist er der Vater der modernen Demokratie und Volkssouveränität. Andererseits enthält seine Theorie den Keim des Totalitarismus. Wenn der Gemeinwille absolut ist, gibt es keinen Platz für Opposition, Parteien oder Minderheitenschutz.
Die Jakobiner (Robespierre) lasen Rousseau während der Französischen Revolution als Handbuch. Sie rechtfertigten den Terror mit genau diesem Argument: Wer gegen die Republik ist, verrät den Gemeinwillen und damit die Freiheit selbst – und muss beseitigt werden. Rousseaus Tugendrepublik wurde so zur blutigen Realität.

\subsection{Die Rettung des Individuums: Émile oder über die Erziehung}
Doch was, wenn der ideale Staat nicht realisierbar ist? Was, wenn wir in einer korrupten Gesellschaft leben müssen, ohne selbst korrupt zu werden? Hier öffnet Rousseau einen zweiten Weg: Die Pädagogik.
Wenn man die Gesellschaft nicht ändern kann, muss man zumindest den \textit{einzelnen Menschen} retten, bevor die Gesellschaft ihn deformiert.
Sein Roman \textit{Émile oder Über die Erziehung} (1762) ist daher kein Schulbuch, sondern ein philosophischer Schutzwall.

\subsubsection*{Das Prinzip der negativen Erziehung}
Rousseaus Ansatz ist revolutionär: Nicht \textit{machen}, sondern \textit{wachsen lassen}. Die traditionelle Erziehung wollte das Kind formen (wie einen Baum beschneiden). Rousseau sagt: \enquote{Alles ist gut, wie es aus den Händen des Schöpfers kommt; alles entartet unter den Händen des Menschen.}
Die Erziehung muss daher \textbf{negativ} sein: Sie besteht nicht darin, Tugend oder Wahrheit zu lehren, sondern das Kind vor Laster und Irrtum zu \textit{bewahren}.
\begin{itemize}
    \item \textbf{Zeit verlieren, um Zeit zu gewinnen:} Das Kind soll nicht frühreif sein. Es darf spielen, toben und \enquote{faul} sein. Intellektuelle Bildung vor der Reife schadet der Seele.
    \item \textbf{Dinge statt Worte:} Das Kind soll nicht durch Autorität lernen (\enquote{Weil ich es sage!}), sondern durch die \textbf{Notwendigkeit der Dinge}. Wenn es ein Fenster einschlägt, soll es frieren, nicht geschimpft werden. Émile lernt Physik und Realität, nicht Rhetorik und Gehorsam.
\end{itemize}

\subsubsection*{Vom Egoismus zur Autonomie}
Ziel ist ein Mensch, der in sich selbst ruht.
Die Erziehung begleitet die Transformation der Triebe:
\begin{enumerate}
    \item \textbf{Die Kindheit (Natur):} Stärkung der gesunden \textit{Amour de soi}. Émile ist autark wie Robinson Crusoe (das einzige Buch, das er lesen darf).
    \item \textbf{Die Jugend (Gesellschaft):} Erst mit der Pubertät, wenn Vernunft und Leidenschaft erwachen, tritt Émile in die Welt ein. Durch das \textit{Glaubensbekenntnis des savoyischen Vikars} (ein eingeschobener Text über natürliche Religion) lernt er, Gott im Herzen zu spüren, ohne Dogmen.
\end{enumerate}
Am Ende steht ein freier Mensch, der in der Gesellschaft leben kann, ohne von ihr abhängig zu sein. Er ist kein \textit{Bürger} im politischen Sinne (wie im Contrat Social), sondern ein \textit{Mensch}.

Was Voltaire durch Kritik und Rousseau durch Neugründung suchten, versuchte eine dritte Gruppe durch schiere Masse und Systematik: Die Enzyklopädisten. Sie setzten nicht auf das einsame Genie, sondern auf das kollektive Wissen.

\section{Diderot und die Encyclopédie: Wissen als Macht}
Wenn Voltaire das Schwert der Aufklärung war und Rousseau ihr Herz, dann war \textbf{Denis Diderot} (1713–1784) ihr Gehirn und ihr Motor.
Sein Lebenswerk ist kein einzelner Text, sondern ein monumentales Unternehmen: Die \textit{Encyclopédie}.

Vordergründig handelte es sich um ein Nachschlagewerk (inspiriert vom Engländer Ephraim Chambers). Doch in den Händen von Diderot und dem Mathematiker d'Alembert wurde es zu einer \textbf{\textit{machine de guerre}} (Kriegsmaschine) gegen das Ancien Régime.
Zwischen 1751 und 1772 erschienen 28 Bände (17 Text-, 11 Tafelbände) mit 72.000 Artikeln. Es war das Wikipedia des 18. Jahrhunderts, aber mit einer klaren politischen Agenda.

\subsection{Das Ziel: \enquote{Changer la façon commune de penser}}
Diderot formulierte das Ziel klar: \enquote{Alles muss geprüft, alles diskutiert, alles untersucht werden, ohne Ausnahme und ohne Rücksicht.}
Die Enzyklopädie sollte das Wissen, das bisher in den Händen der Kirche (Latein) und der Zünfte (Geheimnisse) lag, \textbf{demokratisieren}.
\begin{itemize}
    \item \textbf{Die Aufwertung des Handwerks:} Revolutionär war, dass Diderot nicht nur Theologie und Philosophie behandelte, sondern auch das Handwerk und die Technik. In präzisen Kupferstichen wurden Webstühle, Druckpressen und Bergwerke gezeigt. Die Botschaft: Die Arbeit des Bürgers und Bauern ist genauso wertvoll wie die Spekulation des Gelehrten.
    \item \textbf{Die Querverweise (Renvois):} Um die Zensur zu umgehen, nutzen die Enzyklopädisten ein geniales System von Verweisen. In einem harmlosen Artikel über \enquote{Kannibalismus} fand sich plötzlich ein Verweis auf \enquote{Abendmahl} (Eucharistie). So wurde das Dogma durch die Struktur des Buches untergraben.
\end{itemize}

\subsection{Diderots Materialismus}
Diderot selbst ging philosophisch weiter als Voltaire. Er war kein Deist, sondern ein leidenschaftlicher \textbf{Materialist}. Für ihn gab es keinen Uhrmacher-Gott. Die Welt ist Materie in ständiger Bewegung und Gärung.
In seinem visionären Dialog \textit{Le Rêve de d'Alembert} (D'Alemberts Traum) nimmt er die Evolutionstheorie vorweg: Alles ist im Fluss, Arten entstehen und vergehen, der Mensch ist nur ein temporäres Resultat der materiellen Organisation.
Es gibt keinen Geist ohne Körper, keine Seele ohne Gehirn. Diderot schreibt: \enquote{Es gibt nur eine Substanz im Universum, im Menschen und im Tier.}

Diese Gedanken waren so radikal, dass Diderot viele seiner Texte zu Lebzeiten nicht publizieren konnte. Er schrieb sie für die Schublade – oder, wie er sagte, \enquote{für die Nachwelt}.

\section{Die Konsequenz: Von der Philosophie zur Revolution}
% TODO: Wie die Ideen die Monarchie untergruben
