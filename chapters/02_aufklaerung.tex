\chapter{Die Aufklärung (Les Lumières): Der Kampf um die Freiheit}

Wenn das 17. Jahrhundert das \enquote{Zeitalter der Systeme} war, in dem Descartes und Pascal einsam in ihren Studierstuben über Gott und die Seele nachdachten, so ist das 18. Jahrhundert das \enquote{Zeitalter der Kritik}. Die Philosophie verlässt den elfenbeinernen Turm der Metaphysik und betritt die lauten Salons, die Kaffeehäuser und schließlich die Straße.
Das Werkzeug, das Descartes geschmiedet hatte – die autonome Vernunft, die nichts als wahr akzeptiert, was sie nicht selbst geprüft hat –, wird nun zur Waffe. Aber sie richtet sich nicht mehr gegen Zweifel und Dämonen, sondern gegen konkrete irdische Mächte: Gegen den Absolutismus des Königs, gegen die Dogmen der Kirche und gegen die Unmündigkeit des Bürgers.

\section{Vom metaphysischen zum politischen Zweifel}
Wir haben im vorangegangenen Kapitel gesehen, wie sich die französische Philosophie in zwei Ströme teilte:
\begin{itemize}
    \item Den \textbf{Rationalismus} (Descartes), der an die Macht der Vernunft glaubt, Ordnung zu schaffen.
    \item Die \textbf{Existenzanalyse} (Pascal), die das Elend des Menschen und die Grenzen der Machbarkeit betont.
\end{itemize}
In der Aufklärung prallen diese beiden Haltungen erneut aufeinander, jedoch politisch gewendet. Auf der einen Seite steht \textbf{Voltaire}, der Erbe des rationalen Optimismus, der glaubt, dass Vernunft, Wissenschaft und Toleranz die Welt zivilisieren können. Auf der anderen Seite steht \textbf{Jean-Jacques Rousseau}, der Erbe Pascals, der spürt, dass der Fortschritt der Zivilisation den Menschen nicht glücklicher, sondern moralisch verdorbener und unfreier gemacht hat.

Diese Epoche nennt man in Frankreich \enquote{Les Lumières} (Die Lichter) – im Plural. Es ist nicht das eine, monolithische Licht einer einzigen Wahrheit, sondern es sind viele Lichter, die in die dunklen Ecken des Aberglaubens, der Justizwillkür und der religiösen Intoleranz leuchten sollen.

\begin{definitionbox}[Begriff: Der Philosophe]
Der Begriff des Philosophen wandelt sich im 18. Jahrhundert radikal. Er ist nicht mehr der weltabgewandte Grübler (wie Descartes im Ofenzimmer), sondern ein \textbf{l'homme engagé} (ein engagierter Mensch).
Der \textit{Philosophe} der Aufklärung ist ein Publizist, ein Kämpfer, ein intellektueller Unruhestifter. Er schreibt keine dicken lateinischen Wälzer mehr, sondern Romane, Theaterstücke, Flugblätter und Enzyklopädie-Artikel. Sein Ziel ist nicht mehr nur die \textit{Erkenntnis} der Welt, sondern ihre \textit{Verbesserung}.
\end{definitionbox}

Das zentrale Projekt dieser Zeit ist die \textbf{Encyclopédie} von Diderot und d'Alembert: Der Versuch, das gesamte Wissen der Menschheit zu sammeln und demokratisch zugänglich zu machen, um so die Macht der alten Eliten (Klerus und Adel), die auf Geheimwissen und Tradition basierte, zu brechen. Wissen ist nicht mehr nur Macht, Wissen ist Freiheit.

\section{Voltaire: Die Waffe des Spottes}
Wenn es einen Menschen gibt, der das 18. Jahrhundert verkörpert, dann ist es \textbf{François-Marie Arouet}, genannt \textbf{Voltaire} (1694–1778). Er war kein systematischer Philosoph wie Kant oder Descartes; er baute keine Kathedralen aus Begriffen. Voltaire war ein Freigeist, ein Polemiker, ein \textit{Stilist}. Seine Waffe war nicht der Syllogismus, sondern der \textbf{Esprit}: Der geschliffene Witz, die Ironie und der Sarkasmus.
Er erkannte, dass man Autoritäten am effektivsten stürzt, indem man sie lächerlich macht. Während Descartes den Zweifel methodisch nutzte, nutzt Voltaire das Lachen politisch.

\subsection{Der Kampf gegen das Ungeheuer: \enquote{Écrasez l'infâme!}}
Voltaires Leben war ein einziger Kreuzzug gegen die Intoleranz. Sein Schlachtruf, mit dem er seine Briefe zu unterzeichnen pflegte, lautete: \textit{\enquote{Écrasez l'infâme!}} (Zermalmt das Niederträchtige!).
Was war dieses \enquote{Niederträchtige}? Es war nicht der Glaube an Gott an sich, sondern der \textbf{Fanatismus}, der Dogmatismus und die klerikale Macht, die im Namen der Religion Unrecht beging.

Ein Paradebeispiel für den \textit{Philosophe engagé} ist die \textbf{Affäre Calas} (1762). Jean Calas, ein Protestant, wurde in Toulouse unter falschen Indizien gerädert, weil man ihm vorwarf, seinen Sohn ermordet zu haben, um dessen Konversion zum Katholizismus zu verhindern.
Voltaire, damals schon ein alter Mann, griff ein. Er mobilisierte die öffentliche Meinung in ganz Europa, schrieb Pamphlete, nutzte seine Kontakte zu Königen und zwang den französischen Staat schlussendlich zur Rehabilitation.
Hier zeigt sich das Neue: Die Philosophie ist nicht mehr nur Theorie, sie ist eine moralische Instanz, die als Anwalt der Opfer gegen die Staatsmacht auftritt.

\subsection{Theologie: Der Uhrmacher-Gott (Deismus)}
War Voltaire Atheist? Nein. Er hielt den Atheismus für gefährlich (weil er die Moral des Volkes untergraben könnte) und den Katholizismus für lächerlich. Sein Weg war der \textbf{Deismus}.

\begin{intuition}[Metapher: Der Große Uhrmacher]
Voltaire argumentiert empirisch: \enquote{Das Universum macht mich verlegen, und ich kann mir nicht denken, dass diese Uhr existiert und keinen Uhrmacher hat.}
Gott ist für Voltaire der \textit{Grand Horloger} (Der große Uhrmacher). Er hat das komplexe Räderwerk des Kosmos konstruiert, die Naturgesetze installiert und die Maschine in Gang gesetzt.
Aber – und das ist entscheidend – danach greift er nicht mehr ein. Er wirkt keine Wunder, er hört keine Gebete, er interessiert sich nicht dafür, ob wir am Freitag Fleisch essen oder Latein sprechen. Er ist ein Gott der Vernunft, nicht des Kults.
\end{intuition}

\subsection{Candide und das Ende des Optimismus}
Voltaires philosophisches Testament ist sein kurzer Roman \textit{Candide oder der Optimismus} (1759). Darin verspottet er die Lehre von Leibniz, wir lebten in der \enquote{besten aller möglichen Welten}.
Der junge Candide reist durch eine Welt voller Erdbeben, Kriege, Inquisition und Syphilis, während sein Lehrer Pangloss stur behauptet, alles sei zum Besten bestellt.
Am Ende dieser Odyssee lehnt Candide alle großen metaphysischen Erklärungen ab. Auf die Frage, was wir tun sollen, antwortet er mit dem berühmten Satz:
\textit{\enquote{Il faut cultiver notre jardin.}} (Wir müssen unseren Garten bestellen).

Das ist keine Aufforderung zum Gärtnern, sondern eine Absage an die Utopie. Wir können die Welt nicht im Großen retten, wir können das Übel nicht theoretisch wegdiskutieren. Aber wir können im Kleinen, in unserem direkten Umfeld (\enquote{unserem Garten}), vernünftig und produktiv wirken, um das konkrete Leid zu lindern. Das ist der pragmatische Humanismus der Aufklärung.

\section{Jean-Jacques Rousseau: Die Umkehrung des Fortschritts}
% TODO: Diskurs über Ungleichheit, Gesellschaftsvertrag, Émile, Zivilisationskritik

\section{Diderot und die Encyclopédie: Wissen als Macht}
% TODO: Materialismus, Kampf gegen die Zensur, Demokratisierung des Wissens

\section{Die Konsequenz: Von der Philosophie zur Revolution}
% TODO: Wie die Ideen die Monarchie untergruben
