\chapter{Die Aufklärung (Les Lumières): Der Kampf um die Freiheit}

Wenn das 17. Jahrhundert das \enquote{Zeitalter der Systeme} war, in dem Descartes und Pascal einsam in ihren Studierstuben über Gott und die Seele nachdachten, so ist das 18. Jahrhundert das \enquote{Zeitalter der Kritik}. Die Philosophie verlässt den elfenbeinernen Turm der Metaphysik und betritt die lauten Salons, die Kaffeehäuser und schließlich die Straße.
Das Werkzeug, das Descartes geschmiedet hatte – die autonome Vernunft, die nichts als wahr akzeptiert, was sie nicht selbst geprüft hat –, wird nun zur Waffe. Aber sie richtet sich nicht mehr gegen Zweifel und Dämonen, sondern gegen konkrete irdische Mächte: Gegen den Absolutismus des Königs, gegen die Dogmen der Kirche und gegen die Unmündigkeit des Bürgers.

\section{Vom metaphysischen zum politischen Zweifel}
Wir haben im vorangegangenen Kapitel gesehen, wie sich die französische Philosophie in zwei Ströme teilte:
\begin{itemize}
    \item Den \textbf{Rationalismus} (Descartes), der an die Macht der Vernunft glaubt, Ordnung zu schaffen.
    \item Die \textbf{Existenzanalyse} (Pascal), die das Elend des Menschen und die Grenzen der Machbarkeit betont.
\end{itemize}
In der Aufklärung prallen diese beiden Haltungen erneut aufeinander, jedoch politisch gewendet. Auf der einen Seite steht \textbf{Voltaire}, der Erbe des rationalen Optimismus, der glaubt, dass Vernunft, Wissenschaft und Toleranz die Welt zivilisieren können. Auf der anderen Seite steht \textbf{Jean-Jacques Rousseau}, der Erbe Pascals, der spürt, dass der Fortschritt der Zivilisation den Menschen nicht glücklicher, sondern moralisch verdorbener und unfreier gemacht hat.

Diese Epoche nennt man in Frankreich \enquote{Les Lumières} (Die Lichter) – im Plural. Es ist nicht das eine, monolithische Licht einer einzigen Wahrheit, sondern es sind viele Lichter, die in die dunklen Ecken des Aberglaubens, der Justizwillkür und der religiösen Intoleranz leuchten sollen.

\begin{definitionbox}[Begriff: Der Philosophe]
Der Begriff des Philosophen wandelt sich im 18. Jahrhundert radikal. Er ist nicht mehr der weltabgewandte Grübler (wie Descartes im Ofenzimmer), sondern ein \textbf{l'homme engagé} (ein engagierter Mensch).
Der \textit{Philosophe} der Aufklärung ist ein Publizist, ein Kämpfer, ein intellektueller Unruhestifter. Er schreibt keine dicken lateinischen Wälzer mehr, sondern Romane, Theaterstücke, Flugblätter und Enzyklopädie-Artikel. Sein Ziel ist nicht mehr nur die \textit{Erkenntnis} der Welt, sondern ihre \textit{Verbesserung}.
\end{definitionbox}

Das zentrale Projekt dieser Zeit ist die \textbf{Encyclopédie} von Diderot und d'Alembert: Der Versuch, das gesamte Wissen der Menschheit zu sammeln und demokratisch zugänglich zu machen, um so die Macht der alten Eliten (Klerus und Adel), die auf Geheimwissen und Tradition basierte, zu brechen. Wissen ist nicht mehr nur Macht, Wissen ist Freiheit.
