\chapter{Rationalismus und Zweifel: Das Fundament}

Wir schreiben das Jahr 1619. In einem kleinen, beheizten Zimmer in Neuburg an der Donau sitzt ein Mann und beschließt, das gesamte Wissen seiner Zeit niederzureißen. Sein Name ist \textbf{René Descartes} (1596–1650). Was er in diesen kalten Winternächten beginnt, ist kein gewöhnliches philosophisches Projekt, sondern ein intellektueller Neustart. Descartes markiert den Bruch mit dem Mittelalter, der Scholastik und der Tradition. Er sucht nicht nach Wahrscheinlichkeiten oder autoritären Lehrsätzen (wie \enquote{Aristoteles hat gesagt...}), sondern nach unerschütterlicher, absoluter Gewissheit.

Um die Tragweite dieses Unterfangens zu verstehen, müssen wir uns die Situation der Zeit vor Augen führen. Die moderne Wissenschaft begann gerade erst, das geozentrische Weltbild (die Erde im Mittelpunkt) zu erschüttern (Galilei, Kopernikus). Das Vertrauen in die alten Autoritäten bröckelte. In dieses Vakuum der Ungewissheit hinein wagt Descartes den Versuch, die Philosophie so exakt zu machen wie die Mathematik.

\section{René Descartes: Der Architekt der Vernunft}

Descartes' Ansatz ist radikal, doch er beginnt nicht chaotisch. Bevor er das alte Gebäude des Wissens einreißen kann, benötigt er einen Bauplan für das neue. In seinem \textit{Diskurs über die Methode} formuliert er vier Regeln, die sicherstellen sollen, dass der Verstand sich nicht verirrt. Die wichtigste davon ist die erste: \enquote{Nichts für wahr halten, was nicht so klar und deutlich erkannt wird, dass kein Zweifel möglich ist.} 

Mit diesem Werkzeug in der Hand – der Forderung nach absoluter Evidenz – wendet er sich nun dem Bestand seines bisherigen Wissens zu. Er stellt fest, dass fast alles, was er zu wissen glaubte, auf wackeligen Beinen steht. Um ein wirklich sicheres Fundament zu finden, muss er das Gelände erst planieren.

\subsection{Der Weg in den Abgrund: Der methodische Zweifel}
Dieser Abriss erfolgt durch den \textit{methodischen Zweifel}. Er ist kein Selbstzweck, sondern ein Sieb: Alles, was auch nur den geringsten Verdacht der Unsicherheit birgt, muss fallen. Descartes führt uns dabei stufenweise tiefer in die Verunsicherung, ähnlich wie man eine Zwiebel schält, um zum Kern zu gelangen.

Er beginnt an der Oberfläche:
\begin{enumerate}
    \item \textbf{Die Sinnestäuschung:} Sehen wir nicht oft Dinge, die nicht da sind? Ein Turm sieht aus der Ferne rund aus, ist aber eckig. Wenn die Sinne uns einmal täuschen, ist es klug, ihnen nie ganz zu vertrauen.
\end{enumerate}

Doch man könnte einwenden: "Okay, meine Augen täuschen mich vielleicht über ferne Türme, aber dass ich hier sitze und schreibe, das ist doch sicher!" Hier setzt Descartes die nächste Stufe an:
\begin{enumerate}
    \setcounter{enumi}{1}
    \item \textbf{Das Traum-Argument:} Woher weiß ich, dass ich jetzt gerade nicht träume? Im Traum erscheinen uns Erlebnisse oft genauso lebhaft. Es gibt kein sicheres Kriterium, um Wachsein und Traum im Moment des Erlebens zu unterscheiden. Damit wird die gesamte Existenz der körperlichen Außenwelt fragwürdig.
\end{enumerate}

Aber selbst im Traum bleibt 2+3=5. Die Struktur der Logik scheint stabil. Doch Descartes geht noch einen Schritt weiter, um auch diese letzte Sicherheit zu erschüttern:
\begin{enumerate}
    \setcounter{enumi}{2}
    \item \textbf{Der Genius malignus (Der böse Dämon):} Was, wenn es keinen gütigen Gott gibt, sondern einen allmächtigen Betrüger-Geist, der mich selbst bei den einfachsten Rechnungen in die Irre führt?
\end{enumerate}

\begin{intuition}[Gedankenexperiment: Der böse Dämon]
Stell dir vor, deine gesamte Realität ist eine Simulation. Der Himmel, die Luft, die Erde, Farben, Gestalten – alles nur Illusionen, die ein böser Dämon dir direkt ins Bewusstsein speist. 
Wenn er allmächtig ist, scheint nichts sicher zu sein. Du hängst in einem vollkommenen Nichts, isoliert und getäuscht.
\end{intuition}

\subsection{Der Wendepunkt: Cogito, ergo sum}
Genau an diesem Punkt der totalen Verzweiflung, wo nichts mehr sicher scheint, schlägt die Strategie um. Der Zweifel stößt auf eine logische Grenze. Denn selbst wenn der Dämon mich täuscht, muss es ein \enquote{Mich} geben, das getäuscht wird. Die Täuschung setzt ein Opfer voraus. Ich kann bezweifeln, dass ich einen Körper habe, aber ich kann nicht bezweifeln, dass ich gerade zweifle.

Daraus folgt blitzartig die erste unerschütterliche Gewissheit: \textit{Cogito, ergo sum} – Ich denke, also bin ich.

Dies ist das gesuchte Fundament. Aber was ist dieses \enquote{Ich}? Da ich meinen Körper ja noch anzweifle (er könnte eine Illusion des Dämons sein), kann ich nur sicher sein, dass ich ein \textit{denkendes Ding} bin.

\begin{definitionbox}[Begriff: Der Kartesische Dualismus]
Descartes vollzieht hier eine radikale Trennung, die das abendländische Denken revolutioniert. Er spaltet die Wirklichkeit in zwei völlig unterschiedliche Substanzen:
\begin{itemize}
    \item \textbf{Res cogitans (Das denkende Ding):} Das Ich, der Geist, die Seele. Sie ist unkörperlich, unteilbar und frei. Ihr Wesen ist das Bewusstsein.
    \item \textbf{Res extensa (Das ausgedehnte Ding):} Die Materie, der Körper, die Natur. Sie ist räumlich ausgedehnt, teilbar und funktioniert rein mechanisch nach Naturgesetzen.
\end{itemize}
\end{definitionbox}

\textbf{Warum war das neu?}
Vor Descartes, im aristotelischen und scholastischen Weltbild des Mittelalters, waren Körper und Seele eine Einheit. Die Seele war die \enquote{Form} des Körpers, das Prinzip, das ihn lebendig machte. Ein Körper ohne Seele war undenkbar, eine Seele ohne Körper unvollständig (Hylomorphismus).
Descartes zerschneidet dieses Band. Für ihn ist der Körper nur noch eine \textbf{Maschine}, ein Automat aus Knochen und Fleisch, vergleichbar mit einem Uhrwerk. Die Seele ist der \enquote{Geist in der Maschine}, der sie steuert, aber wesensmäßig nichts mit ihr zu tun hat.

\textbf{Die gewaltigen Auswirkungen:}
Diese Trennung war ein Befreiungsschlag für die Naturwissenschaften. Wenn der Körper (und die Natur) nur noch tote Materie ohne Seele ist, darf man ihn untersuchen, zersägen und berechnen, ohne religiöse Tabus zu verletzen. Die Medizin und Physik konnten aufblühen.
Doch der Preis dafür ist das bis heute ungelöste \textbf{Leib-Seele-Problem}: Wenn Geist und Körper so grundverschieden sind, wie können sie interagieren? Wie kann ein immaterieller Gedanke (Wille) einen materiellen Arm bewegen? Descartes vermutete den Ort in der Zirbeldrüse – eine Lösung, die schon seine Zeitgenossen nicht überzeugte. Das \enquote{Ich} wurde zum einsamen Beobachter einer mechanischen Welt.

\subsection{Der Sieg des Verstandes: Das Wachs-Beispiel}
Hier könnte man einwenden: \enquote{Gut, ich habe einen Geist. Aber ist die materielle Welt (die Dinge, die ich anfassen kann) nicht viel realer und greifbarer?} Descartes kontert diesen Einwand brillant. Er will zeigen, dass wir selbst körperliche Dinge nicht durch die Sinne, sondern rein durch den Geist erkennen.

Dazu nimmt er das berühmte Stück Bienenwachs: Frisch aus dem Stock ist es hart, kalt, duftet nach Blumen und gibt einen Ton von sich, wenn man darauf klopft.
Nun bringen wir es ans Feuer. Es schmilzt, wird heiß, flüssig, verliert den Duft, wird durchsichtig.
Alle sinnlichen Eigenschaften – alles, was das Auge, die Nase, die Hand gemeldet haben – sind verschwunden. Aber ist es noch dasselbe Wachs? Wir wissen intuitiv: Ja.

Aber woher wissen wir das?
\begin{itemize}
    \item Die Sinne kenne nur das \enquote{Vorher} (hart) und das \enquote{Nachher} (flüssig). Sie melden zwei verschiedene Dinge.
    \item Die Einbildungskraft kann sich nicht alle unendlichen Formen vorstellen, die das Wachs annehmen könnte.
\end{itemize}
Es bleibt nur eine Erklärung: Es ist eine \textit{Einsicht des Verstandes}. Unser Geist \enquote{sieht} die Substanz jenseits der veränderlichen Hülle. Descartes beweist damit: Das, was wir an den Dingen verstehen, kommt nicht von außen (durch die Augen), sondern von innen (durch die Res cogitans). Unser Geist ist das Einzige, was wir wirklich klar erfassen; die Materie bleibt rätselhaft.

\subsection{Der Weg aus dem Gefängnis: Der Gottesbeweis}
Descartes steht nun vor einem gewaltigen Problem. Er hat bewiesen, dass er existiert (\textit{Cogito}), und dass er geistige Ideen hat (\textit{Wachs}). Aber er ist immer noch allein. Er befindet sich im \textbf{Solipsismus}: Er kann nicht beweisen, dass es außer ihm selbst noch irgendetwas gibt.
Warum? Weil der \enquote{böse Dämon} immer noch da sein könnte! Vielleicht täuscht der Dämon ihm die ganze Außenwelt nur vor. Descartes braucht einen Garanten, der sicherstellt, dass seine Wahrnehmung der Welt keine Täuschung ist. Dieser Garant kann nur ein absolut wahrhaftiger Gott sein.

Er argumentiert so (der \textit{Ideen-Beweis}):
\begin{enumerate}
    \item Ich bin ein unvollkommenes Wesen (denn ich zweifle, und Zweifeln ist ein Mangel an Wissen).
    \item Dennoch finde ich in meinem Bewusstsein die klare Idee von etwas absolut \textit{Vollkommenem} und \textit{Unendlichem} (Gott).
    \item Woher kommt diese Idee?
    \item Aus dem Nichts kann nichts kommen (Kausalprinzip).
    \item Ein unvollkommenes Wesen (Ich) kann die Idee der Vollkommenheit nicht selbst produziert haben. (Das Mindere kann nicht das Höhere hervorbringen).
    \item Also muss diese Idee von außen in mich gepflanzt worden sein – von einem Wesen, das tatsächlich vollkommen ist.
    \item Ergo: Gott existiert.
\end{enumerate}

Da Gott vollkommen ist, kann er kein Betrüger sein (Täuschung geschieht aus Schwäche oder Bosheit, also Mangel). Damit ist der \enquote{böse Dämon} besiegt. Gott garantiert, dass unsere Vernunft uns nicht prinzipiell täuscht, wenn wir sie klar und deutlich gebrauchen. Die Außenwelt ist gerettet.

\begin{intuition}[Kritik: Der Zirkelschluss des Descartes]
Dieser Beweis gilt als der schwächste Punkt in Descartes' Philosophie. Kritiker (wie Arnauld) warfen ihm sofort einen logischen Zirkel vor, den sogenannten \textbf{Cartesischen Zirkel}:
\begin{itemize}
    \item Um zu beweisen, dass Gott existiert, benutzt Descartes seine Vernunft (z.B. das Kausalprinzip).
    \item Aber um seiner Vernunft vertrauen zu können, muss er erst beweisen, dass Gott existiert (und kein Dämon ihn täuscht).
\end{itemize}
Er setzt voraus, was er erst beweisen will. Zudem bestritten spätere Empiristen, dass wir eine \enquote{angeborene Idee} von Gott haben; wir könnten sie uns einfach aus endlichen Eigenschaften (Macht, Wissen) zusammengestückelt und ins Unendliche gesteigert haben.
\end{intuition}

\section{Blaise Pascal: Die Grenzen der Vernunft}

Während Descartes mit dem Optimismus eines Architekten das Gebäude der Vernunft errichtet, bemerkt ein anderer französischer Denker bereits die Risse im Fundament. \textbf{Blaise Pascal} (1623–1662) verkörpert den Gegenpol zum Cartesianismus. Als mathematisches Wunderkind (er entwickelte die Wahrscheinlichkeitsrechnung und die erste Rechenmaschine, die \textit{Pascaline}) kannte er die Macht der Logik wie kaum ein anderer.

Doch am späten Abend des 23. November 1654, zwischen 22:30 und 0:30 Uhr, widerfuhr ihm ein Ereignis, das sein Leben in zwei Hälften teilte: Die \enquote{Feuernacht} (\textit{Nuit de feu}).
Pascal befand sich zu dieser Zeit in einer tiefen Krise. Er war weltmüde, depressiv und fühlte eine innere Leere, die weder die Mathematik noch die Gesellschaft salons ausfüllen konnten. Plötzlich, in völliger Einsamkeit, wurde er von einer mystischen Erfahrung überwältigt, die er nicht als Bild beschreibt, sondern als reine Empfindung von \textbf{Feuer}.

Es war kein intellektuelles Verstehen, sondern ein physisches und seelisches Ergriffenwerden. Zwei Stunden lang befand er sich in einem Zustand der Ekstase. Was er genau \enquote{sah}, wissen wir nicht, aber was er \textit{fühlte}, hat er auf dem berühmten \textit{Mémorial} (dem eingenähten Pergament) in stammelnden Worten festgehalten, die fast nur aus Ausrufen bestehen:

\begin{quote}
\textit{\enquote{Feuer.\\
Gott Abrahams, Gott Isaaks, Gott Jakobs, nicht der Philosophen und Gelehrten.\\
Gewissheit, Gewissheit, Empfinden, Freude, Friede.\\
Gott Jesu Christi.\\
...\\
Freude, Freude, Freude, Freudentränen.\\
Ich habe mich von ihm getrennt. Der mich lebendig machen kann (Jeremia 2,13).\\
Möge ich nicht ewig von ihm getrennt sein.}}
\end{quote}

Dieses Dokument war für Pascal mehr als rur ein Zettel; es war ein heiliges Relikt. Er nähte den schmalen Pergamentstreifen sorgfältig in das Futter seines Rockes ein (\textit{pourpoint}). Jedes Mal, wenn er die Kleidung wechselte, trennte er es heraus und nähte es in das neue Gewand wieder ein. Er trug dieses Geheimnis jahrelang buchstäblich auf dem Herzen, ohne dass jemand davon wusste. Erst nach seinem Tod entdeckte ein aufmerksamer Diener eine seltsame Verdickung im Saum seines Mantels, trennte den Stoff auf und fand das verborgene Zeugnis. So erfuhr die Welt erst posthum von der mystischen Quelle, aus der Pascal seine Philosophie schöpfte.

Die Erfahrung selbst war die einer absoluten, persönlichen Präsenz. Während der Gott der Philosophen (Descartes) nur ein abstrakter \enquote{Urheber} ist, zu dem man nicht beten kann, begegnete Pascal hier einem Gott der Liebe, der den Menschen \enquote{im Innersten ergreift}. Das \enquote{Feuer} symbolisiert dabei (wie beim biblischen Dornbusch) die brennende, aber nicht verzehrende Nähe des Göttlichen. Von diesem Moment an entsagte Pascal der Wissenschaft und widmete sein Leben der Apologetik (Verteidigung des Glaubens).

\subsection{Epistemologie: Geometrie vs. Feinsinn (L'esprit de géométrie et de finesse)}
Diese existentielle Erschütterung durch die \textit{Nuit de feu} führt Pascal zu einer radikalen Neubewertung der menschlichen Erkenntnisfähigkeit. Wenn der wahre Gott nicht durch Syllogismen gefunden werden kann, dann muss die Vernunft selbst Grenzen haben. Pascal entwickelt hierfür eine differenzierte Erkenntnistheorie, die den monolithischen Rationalismus von Descartes aufbricht. Er unterscheidet zwei fundamentale Denkmodi, die beide notwendig, aber in ihrer Anwendung streng getrennt sind:

\begin{definitionbox}[Begriff: Die zwei Ordnungen des Denkens]
\begin{itemize}
    \item \textbf{L'esprit de géométrie (Der geometrische Geist):} Dies ist der diskursive, analytische Verstand. Er verfährt langsam, hart und streng nach Regeln. Er beginnt mit klaren Definitionen und Prinzipien und leitet daraus logische Schlüsse ab. Er ist mächtig in der Wissenschaft, aber blind für das Leben. Er sieht die Teile, aber verliert das Ganze.
    \item \textbf{L'esprit de finesse (Der Feinsinn):} Dies ist die intuitive Urteilskraft. Sie erfasst ihren Gegenstand \enquote{auf einen Blick} (\textit{d'une seule vue}), ohne den langsamen Weg der Beweisführung. Sie ist notwendig in Bereichen, wo Definitionen versagen: In der Moral, der Psychologie, der Ästhetik und der religiösen Erfahrung. Die Prinzipien sind hier so fein und zahlreich, dass die grobe Logik sie nicht greifen kann.
\end{itemize}
\end{definitionbox}

Pascal wirft Descartes vor, den Menschen rein auf den \textit{esprit de géométrie} reduzieren zu wollen. Doch das Fundament unseres Wissens ist gar nicht rational. Hier führt Pascal den Begriff des \textbf{Herz} (\textit{le c\oe ur}) ein.
Das \enquote{Herz} ist bei Pascal keine Metapher für Sentiment oder Romantik, sondern ein knallharter technischer Terminus für die \textbf{intuitive Vernunft}. Er argumentiert: Die Vernunft benötigt Axiome (Raum, Zeit, Bewegung, Zahlen), um überhaupt anfangen zu können zu denken. Aber sie kann diese Axiome nicht beweisen – sie muss sie als gegeben voraussetzen.
Wer liefert diese Prinzipien? Das Herz.
\enquote{Wir erkennen die Wahrheit nicht nur mit der Vernunft, sondern auch mit dem Herzen. Auf diese letztere Weise erkennen wir die ersten Prinzipien, und vergebens versucht die Vernunft, die keinen Anteil daran hat, sie zu bekämpfen.}

Das berühmte Zitat: \enquote{\textit{Das Herz hat seine Gründe, die die Vernunft nicht kennt}} (\textit{Le c\oe ur a ses raisons que la raison ne connaît point}), ist also der Beweis für die Begrenztheit des Rationalismus. Der Rationalist ist wie ein Mann, der Beweise dafür verlangt, dass er nicht träumt – eine unmögliche Forderung, da das Gefühl der Realität (\textit{sentiment}) vor dem Beweis liegt.

Dieser epistemologische Bruch – dass wir Dinge wissen, die wir nicht beweisen können – führt direkt in die Zerrissenheit der menschlichen Natur. Wenn unser Erkenntnisvermögen bereits in zwei ungleiche Hälften gespalten ist, wie steht es dann um unser Wesen selbst? Dies leitet über zu Pascals dramatischer Anthropologie.

\subsection{Anthropologie: Elend und Größe (Misère et Grandeur)}
Wenn die Vernunft nicht alles fassen kann, was ist dann der Mensch? Pascals Antwort ist eine der berühmtesten Bestimmungen der philosophischen Anthropologie. Er sieht den Menschen als ein paradoxes Wesen, das sich keiner festen Kategorie zuordnen lässt. Wir sind \enquote{weder Engel noch Tier} (ni ange ni bête), und das Unglück will es, dass wer den Engel spielen will, zum Tier wird.

Pascal verortet den Menschen in einer ontologischen Mitte (\textit{le milieu}):

\begin{enumerate}
    \item \textbf{Die kosmische Leere:} Wir sind verloren zwischen zwei Abgründen. Dem unendlich Großen des Universums, dessen ewiges Schweigen Pascal erschreckt (\textit{Le silence éternel de ces espaces infinis m'effraie}), und dem unendlich Kleinen der mikroskopischen Welt. Wir verstehen weder den Anfang noch das Ende. Wir sind ein Nichts im Vergleich zum All, aber ein All im Vergleich zum Nichts.
    \item \textbf{Das existentielle Paradox:} Der Mensch ist ein \enquote{Monstrum}, eine Chimäre, ein Chaos. Er ist der \enquote{Richter aller Dinge} (durch seine Vernunft) und zugleich ein \enquote{Erdwurm} (durch seine Sterblichkeit); \enquote{Hort der Wahrheit} und \enquote{Kloake der Ungewissheit}.
\end{enumerate}

Dieses Spannungsfeld lässt sich nicht auflösen, es definiert uns. Pascal fasst es in den Begriffen \textit{Misère} (Elend) und \textit{Grandeur} (Größe) zusammen.
Aber – und das ist entscheidend – diese beiden bedingen einander. Wir sind nur deshalb so elend, weil wir eine Ahnung von einer verlorenen Größe haben. Ein Baum ist nicht unglücklich darüber, nicht sprechen zu können. Ein gestürzter König ist unglücklich, weil er weiß, dass er einmal König war.

\begin{intuition}[Metapher: Das denkende Schilfrohr (Le roseau pensant)]
Der Mensch ist das schwächste in der Natur, ein Schilfrohr. Ein Wassertropfen, ein Hauch genügt, um ihn zu töten. Er ist physisch völlig ausgeliefert.
Aber selbst wenn das Universum ihn vernichtete, wäre der Mensch immer noch \textit{edler} als das, was ihn tötet. Warum? Weil er \textit{weiß}, dass er stirbt. Das Universum weiß nichts von seinem Sieg.
Unsere ganze Würde besteht also im Denken. Durch den Raum verschlingt mich das Weltall wie einen Punkt; durch das Denken aber fasse ich das Weltall.
\end{intuition}

Diese Diagnose führt zu einer unerträglichen Spannung. Der Mensch kann diesen Widerspruch (sterbliches Fleisch vs. unsterblicher Geist) nicht dauerhaft aushalten. Er braucht eine Fluchtstrategie. Dies leitet über zu Pascals berühmter psychologischer Theorie: Wenn wir unser Elend nicht heilen können, versuchen wir zumindest, es zu vergessen.

\subsection{Psychologie: Die Theorie der Zerstreuung (Le Divertissement)}
Wie erträgt der Mensch dieses existentielle Spannungsfeld? Wie lebt er mit dem Wissen, ein \enquote{zum Tode verurteiltes Monstrum} zu sein? Pascals Antwort ist eine der ersten großen psychologischen Theorien der Moderne: Das \textit{Divertissement} (wörtlich: das Auseinander-Kehren, die Zerstreuung).

Der Ausgangspunkt ist die Unerträglichkeit der Ruhe. Wenn der Mensch zur Ruhe kommt, zu sich selbst findet, fällt er sofort in den \textbf{Ennui} – eine tiefe, metaphysische Langeweile, die ihm seine Nichtigkeit, seine Verlassenheit und seinen unaufhaltsamen Tod vor Augen führt.
\enquote{Nichts ist dem Menschen unerträglicher, als in vollkommener Ruhe zu sein, ohne Leidenschaften, ohne Geschäfte, ohne Zerstreuung, ohne Anwendung. Dann fühlt er sein Nichts, seine Verlassenheit, seine Unzulänglichkeit, seine Abhängigkeit, seine Ohnmacht, seine Leere.}

Um diesem Abgrund zu entkommen, baut die Gesellschaft ein gigantisches System von Ablenkungen auf.
\begin{itemize}
    \item \textbf{Der Mechanismus der Jagd:} Wir suchen nicht das Glück, sondern die Suche nach dem Glück. Pascal bringt das berühmte Beispiel des Jägers: Er verbringt den ganzen Tag in Kälte und Nässe, um einen Hasen zu jagen. Würde man ihm den Hasen geschenkt geben, würde er ihn ablehnen. Er will nicht die \textit{Beute}, er will die \textit{Jagd} – den Lärm, die Bewegung, die ihn daran hindert, an sich selbst zu denken.
    \item \textbf{Das Unglück des Königs:} Selbst ein König, der eigentlich alles hat, ist unglücklich, wenn er allein ist. Deshalb ist er stets von einem Hofstaat umgeben, der nur eine Aufgabe hat: Ihn zu unterhalten, damit er nicht merkt, dass er ein sterblicher Mensch ist. \enquote{Ein König ohne Zerstreuung ist ein Mensch voller Elend.}
\end{itemize}

Das Fazit ist bitter: Unser ganzes gesellschaftliches Leben (Amt, Krieg, Spiel, Tanz) ist ein \textbf{Fluchtmechanismus}. Wir laufen unbedacht in den Abgrund (den Tod), halten uns aber etwas vor die Augen, um ihn nicht zu sehen. Das Glück, das wir in der Zerstreuung finden, ist negativ: Es besteht nur darin, nicht an unser Unglück zu denken.
Da diese Strategie aber spätestens im Tod scheitert, drängt Pascal auf eine echte Lösung. Wenn die Flucht unmöglich ist, muss man sich der Realität stellen. Dies führt zur berühmtesten Passage der \textit{Pensées}: Der Wette.

\subsection{Theologie: Die Wette (Le Pari)}
Wenn das \textit{Divertissement} nur eine temporäre Betäubung ist, bleibt das Grundproblem bestehen: Wir sind sterblich und wissen nicht, was uns erwartet. Sobald der Lärm der Jagd verstummt, kehrt die existenzielle Angst zurück. Wir stehen allein vor dem Abgrund.
An diesem Punkt, wo die Psychologie (Ablenkung) versagt, muss die Metaphysik (Entscheidung) eintreten. Pascal versucht nun nicht, den Ungläubigen mit Bibelzitaten zu missionieren, sondern ihn dort abzuholen, wo er steht: Bei seinem rationalen Eigeninteresse.

Da Gottes Existenz rational nicht beweisbar ist (er ist ein \textit{Deus absconditus}, ein verborgener Gott), befinden wir uns in einer Situation der absoluten Ungewissheit, ähnlich einem Glücksspiel.

Wir stehen vor einem Münzwurf: \enquote{Kopf oder Zahl?} Gott ist, oder er ist nicht. Die Vernunft kann hier nicht entscheiden. Aber wir \textit{müssen} setzen. Wir können uns nicht enthalten, denn wir sind bereits \enquote{eingeschifft} (\textit{embarqué}). Wir leben, die Zeit vergeht, wir steuern auf den Tod zu. Nicht zu wählen, bedeutet automatisch, so zu leben, als gäbe es Gott nicht.

Pascal analysiert die Situation spieltheoretisch über den Erwartungswert:
\begin{itemize}
    \item \textbf{Szenario 1: Ich wette auf Gott.}
        \begin{itemize}
            \item \textit{Einsatz:} Endlich (ein tugendhaftes Leben, Verzicht auf einige \enquote{giftige Lüste}).
            \item \textit{Gewinn (falls er existiert):} Unendlich (ewige Glückseligkeit).
            \item \textit{Verlust (falls er nicht existiert):} Endlich (ich habe umsonst gebetet, aber dennoch anständig gelebt).
        \end{itemize}
    \item \textbf{Szenario 2: Ich wette gegen Gott.}
        \begin{itemize}
            \item \textit{Einsatz:} Null (ich genieße mein Leben ohne Einschränkung).
            \item \textit{Gewinn (falls er nicht existiert):} Endlich (ein paar Jahre irdisches Vergnügen).
            \item \textit{Verlust (falls er existiert):} Unendlich (Verlust des ewigen Heils, das Nichts).
        \end{itemize}
\end{itemize}

Die Mathematik ist eindeutig: \textit{\enquote{Überall da, wo das Unendliche steht und wo nicht unendlich viele Chancen des Verlustes gegen die des Gewinnes stehen, da darf man nicht anstehen, alles zu setzen.}} Selbst wenn die Wahrscheinlichkeit für Gottes Existenz winzig wäre, multipliziert mit einem unendlichen Gewinn ergibt sie einen unendlichen Erwartungswert. Die Vernunft zwingt uns also zur Wette auf Gott.

Doch Pascal ist Realist. Er weiß, dass eine rationale Einsicht noch keinen Glauben macht. \enquote{Ich sehe das ein}, sagt der Ungläubige, \enquote{aber ich kann nicht glauben. Meine Leidenschaften halten mich zurück.}
Pascals Rat ist radikal und psychologisch tiefgründig: \textit{Fake it till you make it}.
Wenn der Geist will, aber das Herz (die Maschine in uns) blockiert, muss man die Maschine mechanisch umprogrammieren. \enquote{Handelt so, als ob ihr glaubtet: Nehmt Weihwasser, lasst Messen lesen usw.} Diese Gewohnheit wird den Stolz des Verstandes brechen und den Menschen für die Gnade öffnen. Pascal nennt das drastisch: \textit{Cela vous abêtira} – das wird euch dumm machen (im Sinne von: demütig, tierhaft-automatisch, den intellektuellen Hochmut dämpfend).

\section*{Zusammenfassung des Kapitels}
Hier am Ende des ersten Kapitels sehen wir die zwei Pole, zwischen denen sich die französische Philosophie aufspannt:
\begin{enumerate}
    \item \textbf{Der Cartesianismus:} Das Vertrauen auf das autonome Subjekt, das durch Methode und Vernunft die Welt beherrscht. (Licht, Klarheit, Stärke).
    \item \textbf{Die Pascalsche Linie:} Die Einsicht in die Grenzen der Vernunft, die Tragik der menschlichen Existenz und die Notwendigkeit einer Ordnung des Herzens. (Nacht, Abgrund, Schwäche).
\end{enumerate}
Im nächsten Kapitel werden wir sehen, wie die Aufklärer (Voltaire, Rousseau) versuchen, beide Stränge zu einer politischen Sprengkraft zu verbinden.
