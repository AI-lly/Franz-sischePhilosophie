\chapter{Rationalismus und Zweifel: Das Fundament}

Wir schreiben das Jahr 1619. In einem kleinen, beheizten Zimmer in Neuburg an der Donau sitzt ein Mann und beschließt, das gesamte Wissen seiner Zeit niederzureißen. Sein Name ist \textbf{René Descartes} (1596–1650). Was er in diesen kalten Winternächten beginnt, ist kein gewöhnliches philosophisches Projekt, sondern ein intellektueller Neustart. Descartes markiert den Bruch mit dem Mittelalter, der Scholastik und der Tradition. Er sucht nicht nach Wahrscheinlichkeiten oder autoritären Lehrsätzen (wie \enquote{Aristoteles hat gesagt...}), sondern nach unerschütterlicher, absoluter Gewissheit.

Um die Tragweite dieses Unterfangens zu verstehen, müssen wir uns die Situation der Zeit vor Augen führen. Die moderne Wissenschaft begann gerade erst, das geozentrische Weltbild (die Erde im Mittelpunkt) zu erschüttern (Galilei, Kopernikus). Das Vertrauen in die alten Autoritäten bröckelte. In dieses Vakuum der Ungewissheit hinein wagt Descartes den Versuch, die Philosophie so exakt zu machen wie die Mathematik.

\section{René Descartes: Der Architekt der Vernunft}

Descartes' Ansatz ist radikal, doch er beginnt nicht chaotisch. Bevor er das alte Gebäude des Wissens einreißen kann, benötigt er einen Bauplan für das neue. In seinem \textit{Diskurs über die Methode} formuliert er vier Regeln, die sicherstellen sollen, dass der Verstand sich nicht verirrt. Die wichtigste davon ist die erste: \enquote{Nichts für wahr halten, was nicht so klar und deutlich erkannt wird, dass kein Zweifel möglich ist.} 

Mit diesem Werkzeug in der Hand – der Forderung nach absoluter Evidenz – wendet er sich nun dem Bestand seines bisherigen Wissens zu. Er stellt fest, dass fast alles, was er zu wissen glaubte, auf wackeligen Beinen steht. Um ein wirklich sicheres Fundament zu finden, muss er das Gelände erst planieren.

\subsection{Der Weg in den Abgrund: Der methodische Zweifel}
Dieser Abriss erfolgt durch den \textit{methodischen Zweifel}. Er ist kein Selbstzweck, sondern ein Sieb: Alles, was auch nur den geringsten Verdacht der Unsicherheit birgt, muss fallen. Descartes führt uns dabei stufenweise tiefer in die Verunsicherung, ähnlich wie man eine Zwiebel schält, um zum Kern zu gelangen.

Er beginnt an der Oberfläche:
\begin{enumerate}
    \item \textbf{Die Sinnestäuschung:} Sehen wir nicht oft Dinge, die nicht da sind? Ein Turm sieht aus der Ferne rund aus, ist aber eckig. Wenn die Sinne uns einmal täuschen, ist es klug, ihnen nie ganz zu vertrauen.
\end{enumerate}

Doch man könnte einwenden: "Okay, meine Augen täuschen mich vielleicht über ferne Türme, aber dass ich hier sitze und schreibe, das ist doch sicher!" Hier setzt Descartes die nächste Stufe an:
\begin{enumerate}
    \setcounter{enumi}{1}
    \item \textbf{Das Traum-Argument:} Woher weiß ich, dass ich jetzt gerade nicht träume? Im Traum erscheinen uns Erlebnisse oft genauso lebhaft. Es gibt kein sicheres Kriterium, um Wachsein und Traum im Moment des Erlebens zu unterscheiden. Damit wird die gesamte Existenz der körperlichen Außenwelt fragwürdig.
\end{enumerate}

Aber selbst im Traum bleibt 2+3=5. Die Struktur der Logik scheint stabil. Doch Descartes geht noch einen Schritt weiter, um auch diese letzte Sicherheit zu erschüttern:
\begin{enumerate}
    \setcounter{enumi}{2}
    \item \textbf{Der Genius malignus (Der böse Dämon):} Was, wenn es keinen gütigen Gott gibt, sondern einen allmächtigen Betrüger-Geist, der mich selbst bei den einfachsten Rechnungen in die Irre führt?
\end{enumerate}

\begin{intuition}[Gedankenexperiment: Der böse Dämon]
Stell dir vor, deine gesamte Realität ist eine Simulation. Der Himmel, die Luft, die Erde, Farben, Gestalten – alles nur Illusionen, die ein böser Dämon dir direkt ins Bewusstsein speist. 
Wenn er allmächtig ist, scheint nichts sicher zu sein. Du hängst in einem vollkommenen Nichts, isoliert und getäuscht.
\end{intuition}

\subsection{Der Wendepunkt: Cogito, ergo sum}
Genau an diesem Punkt der totalen Verzweiflung, wo nichts mehr sicher scheint, schlägt die Strategie um. Der Zweifel stößt auf eine logische Grenze. Denn selbst wenn der Dämon mich täuscht, muss es ein \enquote{Mich} geben, das getäuscht wird. Die Täuschung setzt ein Opfer voraus. Ich kann bezweifeln, dass ich einen Körper habe, aber ich kann nicht bezweifeln, dass ich gerade zweifle.

Daraus folgt blitzartig die erste unerschütterliche Gewissheit: \textit{Cogito, ergo sum} – Ich denke, also bin ich.

Dies ist das gesuchte Fundament. Aber was ist dieses \enquote{Ich}? Da ich meinen Körper ja noch anzweifle (er könnte eine Illusion des Dämons sein), kann ich nur sicher sein, dass ich ein \textit{denkendes Ding} bin.

\begin{definitionbox}[Begriff: Der Kartesische Dualismus]
Descartes vollzieht hier eine radikale Trennung, die das abendländische Denken revolutioniert. Er spaltet die Wirklichkeit in zwei völlig unterschiedliche Substanzen:
\begin{itemize}
    \item \textbf{Res cogitans (Das denkende Ding):} Das Ich, der Geist, die Seele. Sie ist unkörperlich, unteilbar und frei. Ihr Wesen ist das Bewusstsein.
    \item \textbf{Res extensa (Das ausgedehnte Ding):} Die Materie, der Körper, die Natur. Sie ist räumlich ausgedehnt, teilbar und funktioniert rein mechanisch nach Naturgesetzen.
\end{itemize}
\end{definitionbox}

\textbf{Warum war das neu?}
Vor Descartes, im aristotelischen und scholastischen Weltbild des Mittelalters, waren Körper und Seele eine Einheit. Die Seele war die \enquote{Form} des Körpers, das Prinzip, das ihn lebendig machte. Ein Körper ohne Seele war undenkbar, eine Seele ohne Körper unvollständig (Hylomorphismus).
Descartes zerschneidet dieses Band. Für ihn ist der Körper nur noch eine \textbf{Maschine}, ein Automat aus Knochen und Fleisch, vergleichbar mit einem Uhrwerk. Die Seele ist der \enquote{Geist in der Maschine}, der sie steuert, aber wesensmäßig nichts mit ihr zu tun hat.

\textbf{Die gewaltigen Auswirkungen:}
Diese Trennung war ein Befreiungsschlag für die Naturwissenschaften. Wenn der Körper (und die Natur) nur noch tote Materie ohne Seele ist, darf man ihn untersuchen, zersägen und berechnen, ohne religiöse Tabus zu verletzen. Die Medizin und Physik konnten aufblühen.
Doch der Preis dafür ist das bis heute ungelöste \textbf{Leib-Seele-Problem}: Wenn Geist und Körper so grundverschieden sind, wie können sie interagieren? Wie kann ein immaterieller Gedanke (Wille) einen materiellen Arm bewegen? Descartes vermutete den Ort in der Zirbeldrüse – eine Lösung, die schon seine Zeitgenossen nicht überzeugte. Das \enquote{Ich} wurde zum einsamen Beobachter einer mechanischen Welt.

\subsection{Der Sieg des Verstandes: Das Wachs-Beispiel}
Hier könnte man einwenden: \enquote{Gut, ich habe einen Geist. Aber ist die materielle Welt (die Dinge, die ich anfassen kann) nicht viel realer und greifbarer?} Descartes kontert diesen Einwand brillant. Er will zeigen, dass wir selbst körperliche Dinge nicht durch die Sinne, sondern rein durch den Geist erkennen.

Dazu nimmt er das berühmte Stück Bienenwachs: Frisch aus dem Stock ist es hart, kalt, duftet nach Blumen und gibt einen Ton von sich, wenn man darauf klopft.
Nun bringen wir es ans Feuer. Es schmilzt, wird heiß, flüssig, verliert den Duft, wird durchsichtig.
Alle sinnlichen Eigenschaften – alles, was das Auge, die Nase, die Hand gemeldet haben – sind verschwunden. Aber ist es noch dasselbe Wachs? Wir wissen intuitiv: Ja.

Aber woher wissen wir das?
\begin{itemize}
    \item Die Sinne kenne nur das \enquote{Vorher} (hart) und das \enquote{Nachher} (flüssig). Sie melden zwei verschiedene Dinge.
    \item Die Einbildungskraft kann sich nicht alle unendlichen Formen vorstellen, die das Wachs annehmen könnte.
\end{itemize}
Es bleibt nur eine Erklärung: Es ist eine \textit{Einsicht des Verstandes}. Unser Geist \enquote{sieht} die Substanz jenseits der veränderlichen Hülle. Descartes beweist damit: Das, was wir an den Dingen verstehen, kommt nicht von außen (durch die Augen), sondern von innen (durch die Res cogitans). Unser Geist ist das Einzige, was wir wirklich klar erfassen; die Materie bleibt rätselhaft.

\subsection{Der Weg aus dem Gefängnis: Der Gottesbeweis}
Descartes steht nun vor einem gewaltigen Problem. Er hat bewiesen, dass er existiert (\textit{Cogito}), und dass er geistige Ideen hat (\textit{Wachs}). Aber er ist immer noch allein. Er befindet sich im \textbf{Solipsismus}: Er kann nicht beweisen, dass es außer ihm selbst noch irgendetwas gibt.
Warum? Weil der \enquote{böse Dämon} immer noch da sein könnte! Vielleicht täuscht der Dämon ihm die ganze Außenwelt nur vor. Descartes braucht einen Garanten, der sicherstellt, dass seine Wahrnehmung der Welt keine Täuschung ist. Dieser Garant kann nur ein absolut wahrhaftiger Gott sein.

Er argumentiert so (der \textit{Ideen-Beweis}):
\begin{enumerate}
    \item Ich bin ein unvollkommenes Wesen (denn ich zweifle, und Zweifeln ist ein Mangel an Wissen).
    \item Dennoch finde ich in meinem Bewusstsein die klare Idee von etwas absolut \textit{Vollkommenem} und \textit{Unendlichem} (Gott).
    \item Woher kommt diese Idee?
    \item Aus dem Nichts kann nichts kommen (Kausalprinzip).
    \item Ein unvollkommenes Wesen (Ich) kann die Idee der Vollkommenheit nicht selbst produziert haben. (Das Mindere kann nicht das Höhere hervorbringen).
    \item Also muss diese Idee von außen in mich gepflanzt worden sein – von einem Wesen, das tatsächlich vollkommen ist.
    \item Ergo: Gott existiert.
\end{enumerate}

Da Gott vollkommen ist, kann er kein Betrüger sein (Täuschung geschieht aus Schwäche oder Bosheit, also Mangel). Damit ist der \enquote{böse Dämon} besiegt. Gott garantiert, dass unsere Vernunft uns nicht prinzipiell täuscht, wenn wir sie klar und deutlich gebrauchen. Die Außenwelt ist gerettet.

\begin{intuition}[Kritik: Der Zirkelschluss des Descartes]
Dieser Beweis gilt als der schwächste Punkt in Descartes' Philosophie. Kritiker (wie Arnauld) warfen ihm sofort einen logischen Zirkel vor, den sogenannten \textbf{Cartesischen Zirkel}:
\begin{itemize}
    \item Um zu beweisen, dass Gott existiert, benutzt Descartes seine Vernunft (z.B. das Kausalprinzip).
    \item Aber um seiner Vernunft vertrauen zu können, muss er erst beweisen, dass Gott existiert (und kein Dämon ihn täuscht).
\end{itemize}
Er setzt voraus, was er erst beweisen will. Zudem bestritten spätere Empiristen, dass wir eine \enquote{angeborene Idee} von Gott haben; wir könnten sie uns einfach aus endlichen Eigenschaften (Macht, Wissen) zusammengestückelt und ins Unendliche gesteigert haben.
\end{intuition}

\section{Blaise Pascal: Die Grenzen der Vernunft}

Während Descartes mit dem Optimismus eines Architekten das Gebäude der Vernunft errichtet, bemerkt ein anderer französischer Denker bereits die Risse im Fundament. \textbf{Blaise Pascal} (1623–1662) verkörpert den Gegenpol zum Cartesianismus. Als mathematisches Wunderkind (er entwickelte die Wahrscheinlichkeitsrechnung und die erste Rechenmaschine, die \textit{Pascaline}) kannte er die Macht der Logik wie kaum ein anderer.

Doch am späten Abend des 23. November 1654, zwischen 22:30 und 0:30 Uhr, widerfuhr ihm ein Ereignis, das sein Leben in zwei Hälften teilte: Die \enquote{Feuernacht} (\textit{Nuit de feu}).
Pascal befand sich zu dieser Zeit in einer tiefen Krise. Er war weltmüde, depressiv und fühlte eine innere Leere, die weder die Mathematik noch die Gesellschaft salons ausfüllen konnten. Plötzlich, in völliger Einsamkeit, wurde er von einer mystischen Erfahrung überwältigt, die er nicht als Bild beschreibt, sondern als reine Empfindung von \textbf{Feuer}.

Es war kein intellektuelles Verstehen, sondern ein physisches und seelisches Ergriffenwerden. Zwei Stunden lang befand er sich in einem Zustand der Ekstase. Was er genau \enquote{sah}, wissen wir nicht, aber was er \textit{fühlte}, hat er auf dem berühmten \textit{Mémorial} (dem eingenähten Pergament) in stammelnden Worten festgehalten, die fast nur aus Ausrufen bestehen:

\begin{quote}
\textit{\enquote{Feuer.\\
Gott Abrahams, Gott Isaaks, Gott Jakobs, nicht der Philosophen und Gelehrten.\\
Gewissheit, Gewissheit, Empfinden, Freude, Friede.\\
Gott Jesu Christi.\\
...\\
Freude, Freude, Freude, Freudentränen.\\
Ich habe mich von ihm getrennt. Der mich lebendig machen kann (Jeremia 2,13).\\
Möge ich nicht ewig von ihm getrennt sein.}}
\end{quote}

Die Erfahrung war die einer absoluten, persönlichen Präsenz. Während der Gott der Philosophen (Descartes) nur ein abstrakter \enquote{Urheber} ist, zu dem man nicht beten kann, begegnete Pascal hier einem Gott der Liebe, der den Menschen \enquote{im Innersten ergreift}. Das \enquote{Feuer} symbolisiert dabei (wie beim biblischen Dornbusch) die brennende, aber nicht verzehrende Nähe des Göttlichen. Von diesem Moment an entsagte Pascal der Wissenschaft und widmete sein Leben der Apologetik (Verteidigung des Glaubens).

\subsection{Epistemologie: Geometrie vs. Feinsinn}
Für Pascal ist der Rationalismus à la Descartes eine Hybris. Die Vernunft ist ein mächtiges, aber begrenztes Werkzeug. Um die Wirklichkeit vollständig zu erfassen, führt Pascal eine fundamentale Unterscheidung ein:

\begin{definitionbox}[Begriff: Die zwei Ordnungen des Denkens]
\begin{itemize}
    \item \textbf{L'esprit de géométrie (Der geometrische Geist):} Dies ist der rationale Verstand. Er arbeitet diskursiv, deduktiv und beweist Theoreme aus Prinzipien. Er ist langsam, starr, aber präzise. Dies ist das Werkzeug von Descartes.
    \item \textbf{L'esprit de finesse (Der Feinsinn):} Dies ist die intuitive Erkenntniskraft. Sie erfasst die Dinge \enquote{auf einen Blick}. Sie ist notwendig für Bereiche, die sich der mathematischen Logik entziehen: Moral, Ästhetik, Psychologie und Religion.
\end{itemize}
\end{definitionbox}

Pascal argumentiert: \enquote{Das Herz hat seine Gründe, die die Vernunft nicht kennt.} Dies ist kein sentimentaler Kitsch, sondern eine erkenntnistheoretische Aussage. Die \textit{ersten Prinzipien} (dass es Zeit, Raum, Bewegung, Zahlen gibt) können wir nicht beweisen; wir müssen sie \textit{fühlen} (sentir). Der Verstand baut auf dem auf, was das Herz (die Intuition) ihm liefert. Ein reiner Rationalist wäre unfähig zu leben, da er an den Axiomen seiner eigenen Logik zweifeln müsste.

\subsection{Anthropologie: Elend und Größe (Misère et Grandeur)}
Pascals Menschenbild ist dialektisch. Der Mensch ist weder Tier noch Engel, er ist ein \textit{Monstrum}, ein Widerspruch in sich selbst. Er ist eingespannt zwischen zwei Unendlichkeiten: Dem unendlich Großen (dem Kosmos, der ihn erdrückt) und dem unendlich Kleinen (dem mikroskopischen Nichts).

\begin{itemize}
    \item \textbf{La Misère (Das Elend):} Physisch ist der Mensch schwach, sterblich, abhängig, voller Täuschung und Eitelkeit. Er ist ein \enquote{Nichts im Vergleich zum Unendlichen}.
    \item \textbf{La Grandeur (Die Größe):} Doch in diesem Elend liegt seine Würde. Denn er hat Bewusstsein.
\end{itemize}

\begin{intuition}[Metapher: Das denkende Schilfrohr (Le roseau pensant)]
Der Mensch ist nur ein Schilfrohr, das schwächste der Natur. Aber er ist ein \textit{denkendes} Schilfrohr. Es ist nicht nötig, dass das ganze Universum sich rüstet, um ihn zu zermalmen: Ein Dunst, ein Wassertropfen reicht hin, um ihn zu töten.
Aber selbst wenn das Universum ihn vernichtete, wäre der Mensch immer noch edler als das, was ihn tötet, denn er \textit{weiß}, dass er stirbt und dass das Universum im Vorteil ist über ihn. Das Universum weiß nichts davon.
\end{intuition}

\subsection{Psychologie: Die Theorie der Zerstreuung (Le Divertissement)}
Wenn der Mensch so elend ist, wie hält er das Leben aus? Warum denken wir nicht permanent an unseren Tod? Pascals Antwort ist eine frühe psychologische Theorie der Verdrängung: Das \textit{Divertissement} (die Zerstreuung).

Wir suchen ständig Beschäftigung – das Spiel, die Jagd, den Krieg, das Amt, das Gespräch. Nicht, weil uns die Beute (beim Jagen) oder das Geld (beim Glücksspiel) wirklich glücklich macht. Wenn man uns den Hasen schenken würde, ohne dass wir ihn jagen müssten, wollten wir ihn nicht. Es geht nicht um den Besitz, sondern um die \textit{Hatz}, den Lärm, die Bewegung.
Wir tun alles, um nicht zu uns selbst zu kommen. Denn: \enquote{Das ganze Unglück der Menschen rührt allein daher, dass sie nicht ruhig in einem Zimmer bleiben können.} Die Stille konfrontiert uns mit unserer inneren Leere (Ennui) und unserer Sterblichkeit. Die Gesellschaft ist ein gigantischer Mechanismus zur kollektiven Verdrängung der existentiellen Angst.

\subsection{Theologie: Die Wette (Le Pari)}
Da die geometrische Vernunft Gott weder beweisen noch widerlegen kann (denn Gott ist keine mathematische Variable), stehen wir vor einer Entscheidung unter Unsicherheit. Wir \textit{müssen} wählen, denn wir sind bereits \enquote{eingeschifft} (embarqué) ins Leben. Agnostizismus ist unmöglich; wer nicht wählt, wählt auch (nämlich so zu leben, als gäbe es Gott nicht).

Pascal formuliert dies als Entscheidungstheorie (Erwartungswert):
\begin{itemize}
    \item \textbf{Option A: Wette auf Gott.}
        \begin{itemize}
            \item Einsatz: Endlich (Verzicht auf einige irdische Luster).
            \item Möglicher Gewinn: Unendlich (ewige Seligkeit).
        \end{itemize}
    \item \textbf{Option B: Wette gegen Gott.}
        \begin{itemize}
            \item Einsatz: Nichts (man behält seine Laster).
            \item Möglicher Gewinn: Endlich (kurzes irdisches Vergnügen).
            \item Risiko: Verlust des Unendlichen.
        \end{itemize}
\end{itemize}

Mathematisch ist die Wette auf Gott zwingend: $\text{Erwartungswert} = \infty \times \text{Wahrscheinlichkeit} - \text{Einsatz}$. Selbst bei einer sehr kleinen Wahrscheinlichkeit ist der Erwartungswert unendlich.

Doch Pascal weiß: Glaube ist keine reine Rechenoperation. Man kann sich nicht zwingen zu glauben. Sein Rat an den Ungläubigen ist daher praktisch: Er soll so handeln, \textit{als ob} er glaubte (Weihwasser nehmen, Messen besuchen). Diese Gewöhnung (\enquote{cela vous abêtira} – das wird euch dumm machen/demütig machen) bricht den Stolz des Verstandes und öffnet den Weg für die intuitive Gnade.

Hier etabliert sich der fundamentale Riss in der französischen Philosophie: Auf der einen Seite der \textit{Cartesianismus} (Licht, Vernunft, Beweis, Subjektstärke), auf der anderen die \textit{Pascalsche Linie} (Nacht, Herz, Intuition, Subjektschwäche). Beide Linien werden wir in der Aufklärung und im Existentialismus wiederfinden.
