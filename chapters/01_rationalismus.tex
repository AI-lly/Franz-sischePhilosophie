\chapter{Rationalismus und Zweifel: Das Fundament}

Wir schreiben das Jahr 1619. In einem kleinen, beheizten Zimmer in Neuburg an der Donau sitzt ein Mann und beschließt, das gesamte Wissen seiner Zeit niederzureißen. Sein Name ist \textbf{René Descartes} (1596–1650). Was er in diesen kalten Winternächten beginnt, ist kein gewöhnliches philosophisches Projekt, sondern ein intellektueller Neustart. Descartes markiert den Bruch mit dem Mittelalter, der Scholastik und der Tradition. Er sucht nicht nach Wahrscheinlichkeiten oder autoritären Lehrsätzen (wie \enquote{Aristoteles hat gesagt...}), sondern nach unerschütterlicher, absoluter Gewissheit.

Um die Tragweite dieses Unterfangens zu verstehen, müssen wir uns die Situation der Zeit vor Augen führen. Die moderne Wissenschaft begann gerade erst, das geozentrische Weltbild (die Erde im Mittelpunkt) zu erschüttern (Galilei, Kopernikus). Das Vertrauen in die alten Autoritäten bröckelte. In dieses Vakuum der Ungewissheit hinein wagt Descartes den Versuch, die Philosophie so exakt zu machen wie die Mathematik.

\section{René Descartes: Der Architekt der Vernunft}

Descartes' Ansatz ist radikal: Bevor wir etwas als \enquote{wahr} akzeptieren können, müssen wir es einer strengen Prüfung unterziehen. In seinem \textit{Diskurs über die Methode} formuliert er vier Regeln, die das wissenschaftliche Denken bis heute prägen:
\begin{enumerate}
    \item Nichts für wahr halten, was nicht so klar und deutlich erkannt wird, dass kein Zweifel möglich ist.
    \item Jedes Problem in so viele Teile zerlegen wie möglich (Analyse).
    \item Vom Einfachen zum Komplexen aufsteigen (Synthese).
    \item Vollständige Übersichten erstellen, um nichts zu vergessen.
\end{enumerate}

Doch die berühmteste Anwendung dieser Methode finden wir in seinen \textit{Meditationen über die Erste Philosophie} (1641). Hier führt er den Leser durch einen sechstägigen gedanklichen Exerzitienweg.

\subsection{Der methodische Zweifel}
Descartes beginnt mit dem Abrissbirnen-Argument: Der \textit{methodische Zweifel}. Er fragt nicht: \enquote{Was ist wahr?}, sondern: \enquote{Woran kann ich hypothetisch zweifeln?}. Wenn sich auch nur der geringste Riss im Fundament zeigt, muss das Gebäude fallen.

Er durchläuft drei Stufen des Zweifels:
\begin{enumerate}
    \item \textbf{Die Sinnestäuschung:} Sehen wir nicht oft Dinge, die nicht da sind? Ein Turm sieht aus der Ferne rund aus, ist aber eckig. Ein Stab im Wasser wirkt geknickt. Wenn die Sinne uns einmal täuschen, ist es klug, ihnen nie ganz zu vertrauen.
    \item \textbf{Das Traum-Argument:} Woher weiß ich, dass ich jetzt gerade nicht träume? Im Traum erscheinen uns Erlebnisse oft genauso lebhaft und logisch. Es gibt kein sicheres Kriterium, um Wachsein und Traum im Moment des Erlebens zu unterscheiden. Damit fällt auch die Existenz der Außenwelt (mein Körper, dieser Tisch, das Papier) in den Bereich des Zweifelhaften.
    \item \textbf{Der Genius malignus (Der böse Dämon):} Dies ist die radikalste Stufe. Selbst im Traum bleiben mathematische Wahrheiten (2+3=5) scheinbar bestehen. Aber was, wenn es keinen gütigen Gott gibt, sondern einen allmächtigen Betrüger-Geist, der mich selbst bei den einfachsten Rechnungen in die Irre führt?
\end{enumerate}

\begin{intuition}[Gedankenexperiment: Der böse Dämon]
Stell dir vor, deine gesamte Realität ist eine Simulation. Der Himmel, die Luft, die Erde, Farben, Gestalten, Töne – alles nur Illusionen, die ein böser Dämon dir direkt ins Bewusstsein speist. 
Hast du überhaupt einen Körper? Hände, Augen, Fleisch? Vielleicht bist du nur ein isoliertes Bewusstsein, dem eine Welt vorgegaukelt wird. Gibt es \textit{irgendetwas}, das dieser Dämon dir nicht vortäuschen kann? Wenn er allmächtig ist, scheint nichts sicher zu sein – nicht einmal die Mathematik.
\end{intuition}

\subsection{Das Fundament: Cogito, ergo sum}
Inmitten dieses Ozeans des Zweifels findet Descartes einen einzigen festen Punkt, einen archimedischen Punkt. Selbst wenn der Dämon mich täuscht – wer wird getäuscht? \textit{Ich}. Selbst wenn ich träume, muss da jemand sein, der träumt. Selbst wenn ich mich irre, muss ich existieren, um mich zu irren.
Ich kann an allem zweifeln, aber nicht an der Tatsache, \textit{dass} ich zweifle. Und Zweifeln ist ein Denktätigkeit.

Daraus folgt der berühmte Satz (der in den Meditationen so wörtlich gar nicht steht, aber den Kern trifft): \textit{Cogito, ergo sum} – Ich denke, also bin ich.

\begin{definitionbox}[Begriff: Res cogitans vs. Res extensa]
Descartes zieht hier eine scharfe Trennlinie.
\begin{itemize}
    \item \textbf{Res cogitans (Das denkende Ding):} Das Ich. Es ist unkörperlich, unteilbar und besteht rein aus Denken (Zweifeln, Wollen, Fühlen). Es ist dem Geist direkt zugänglich.
    \item \textbf{Res extensa (Das ausgedehnte Ding):} Die materielle Welt (Körper, Tische, Bäume). Sie ist räumlich ausgedehnt, teilbar und funktioniert rein mechanisch.
\end{itemize}
Dies ist die Geburt des kartesischen \textbf{Dualismus}. Geist und Materie sind zwei völlig verschiedene Substanzen.
\end{definitionbox}

\subsection{Das Wachs-Beispiel: Die Macht des Verstandes}
Um zu beweisen, dass der Verstand die Dinge besser erkennt als die Sinne, bringt Descartes das berühmte Beispiel eines Stücks Bienenwachs.
Frisch aus dem Stock ist es hart, kalt, duftet nach Blumen und gibt einen Ton von sich, wenn man darauf klopft.
Nun bringen wir es ans Feuer. Es schmilzt, wird heiß, flüssig, verliert den Duft, wird durchsichtig.
Alle sinnlichen Eigenschaften haben sich geändert. Aber ist es noch dasselbe Wachs? Ja.
Aber woher wissen wir das? Nicht durch die Sinne (die melden etwas völlig anderes), sondern durch den Verstand, der die Substanz jenseits der Wandlungen begreift. Wir sehen nicht mit den Augen, sondern mit dem Geist.

\subsection{Gott als Garant der Welt}
Descartes hat nun das \enquote{Ich}. Aber er ist noch immer allein im Solipsismus (nur das Ich existiert). Wie kommt er zurück zur Außenwelt? Er braucht Gott.
Descartes argumentiert: Ich bin ein unvollkommenes Wesen (da ich zweifle). Dennoch habe ich die Idee einer \textit{vollkommenen} Wesenheit (Gott) in mir. Woher kommt diese Idee? Aus dem Nichts kann nichts kommen. Ein unvollkommenes Wesen kann nicht die Ursache einer vollkommenen Idee sein. Also muss die Idee von Gott selbst eingepflanzt worden sein – wie das Markenzeichen eines Handwerkers auf seinem Werk.
Wenn Gott aber vollkommen ist, kann er kein Betrüger sein (Täuschung ist ein Mangel). Also gibt es keinen bösen Dämon. Wenn wir unsere Vernunft korrekt gebrauchen, zeigt sie uns die Wahrheit über die materielle Welt. Die Welt ist also gerettet – durch die Hintertür der Theologie.

\section{Blaise Pascal: Das Elend und die Größe des Menschen}

Während Descartes das Gebäude der Vernunft errichtet, sieht ein anderer französischer Denker schon die Risse darin. \textbf{Blaise Pascal} (1623–1662) war ein mathematisches Jahrhundertgenie (er erfand die Rechenmaschine, die Wahrscheinlichkeitsrechnung), aber nach einer mystischen Erfahrung (\enquote{Feuer. Gott Abrahams, nicht der Philosophen}) wandte er sich der Religion und der Existenzanalyse des Menschen zu.

Für Pascal ist der Rationalismus à la Descartes Hybris. Die Vernunft ist ein schwaches Werkzeug. Er unterscheidet zwei Zugänge zur Welt:
\begin{itemize}
    \item \textbf{L'esprit de géométrie (Der geometrische Geist):} Das logische, diskursive Denken, das beweist und definiert. (Descartes' Domäne).
    \item \textbf{L'esprit de finesse (Der Feinsinn):} Das intuitive Erfassen, das \enquote{Schauen}, das Gefühl. \enquote{Das Herz hat seine Gründe, die die Vernunft nicht kennt.}
\end{itemize}

\subsection{Die conditio humana: Elend und Langeweile}
Pascal zeichnet ein düsteres Bild des Menschen. Ohne Gott ist der Mensch verloren im Universum, gefangen zwischen zwei Unendlichkeiten: Dem unendlich Großen (Kosmos) und dem unendlich Kleinen (Atome). Wir verstehen weder, woher wir kommen, noch wohin wir gehen.
Um ängstliche Selbstreflexion zu vermeiden, stürzen wir uns in die \textbf{Zerstreuung} (\textit{divertissement}). Wir jagen dem Ruhm, dem Geld oder dem Vergnügen nach, nicht weil es uns glücklich macht, sondern weil es uns davon abhält, über unseren Tod und unsere Nichtigkeit nachzudenken. \enquote{Das ganze Unglück der Menschen rührt allein daher, dass sie nicht ruhig in einem Zimmer bleiben können.}

\begin{intuition}[Metapher: Das denkende Schilfrohr]
Der Mensch ist das schwächste in der Natur, ein Schilfrohr. Ein Wassertropfen, ein Hauch genügt, um ihn zu töten. Aber selbst wenn das Universum ihn vernichtet, wäre der Mensch edler als sein Mörder. Denn er \textit{weiß}, dass er stirbt. Das Universum weiß nichts davon.
Unsere ganze Würde besteht also im Denken. Wir sind elend, weil wir sterblich sind, aber wir sind groß, weil wir uns dieses Elends bewusst sind.
\end{intuition}

\subsection{Die Wette (Le Pari)}
Da die Vernunft allein Gott weder beweisen noch widerlegen kann (anders als Descartes glaubte), schlägt Pascal eine pragmatische Entscheidung vor: Die Wette.
Wir sitzen an einem Spieltisch und müssen setzen. Kopf oder Zahl? Gott existiert oder nicht? Wir können nicht \textit{nicht} spielen (wir leben ja schon).
\begin{itemize}
    \item Ich wette auf Gott: Wenn er existiert, gewinne ich alles (unendliches Glück). Wenn er nicht existiert, verliere ich nichts (nur ein paar irdische Vergnügungen).
    \item Ich wette gegen Gott: Wenn er nicht existiert, gewinne ich nichts Wesentliches. Wenn er existiert, verliere ich alles (das ewige Heil).
\end{itemize}
Rational gesehen ist die Wette auf Gott also zwingend, da der mögliche Gewinn unendlich ist, der Einsatz aber endlich.

Hier sehen wir den fundamentalen Riss in der französischen Philosophie: Auf der einen Seite der \textit{Cartesianismus} mit seinem unbedingten Glauben an Methode, Klarheit und Vernunft. Auf der anderen Seite die \textit{Pascalsche Linie}, die auf die Grenzen der Vernunft, die existenzielle Angst und die Tragik des Menschseins verweist – eine Linie, die später direkt zum Existentialismus führen wird.
