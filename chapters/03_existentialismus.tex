\chapter{Der Bruch: Die Existenz geht der Essenz voraus}

\section{Einleitung: Die verwaiste Welt}
Wenn die Aufklärung der Morgen war, an dem die Vernunft erwachte, so ist der Existentialismus die dunkelste Stunde der Nacht, in der der Mensch merkt, dass er allein ist.
Das 19. Jahrhundert hatte noch an den Fortschritt geglaubt. Hegel sah die Geschichte als Entfaltung des Weltgeistes, Marx als Weg zum Kommunismus. Doch dieser Optimismus starb in den Schützengräben von Verdun und später in den Lagern des 20. Jahrhunderts.

Nietzsches Diagnose vom \enquote{Tod Gottes} war keine bloße atheistische Provokation, sondern eine seismographische Erschütterung: Der metaphysische Boden war weggebrochen. Es gab keinen Himmel mehr, der Trost spendete, und keine objektive Moral, die Richtung gab. Georg Lukács nannte diesen Zustand \textit{transzendentale Obdachlosigkeit}.
Der Mensch findet sich in einem Universum wieder, das \textbf{schweigt}. Es ist nicht mehr der wohlgeordnete Kosmos der Antike oder das Uhrwerk Voltaires. Es ist eine gleichgültige, fremde Masse.

In diesem Vakuum muss die Philosophie neu ansetzen. Sie kann nicht mehr vom \enquote{Allgemeinen} ausgehen (dem Geist, der Klasse, der Nation), sondern muss beim \textbf{Einzelnen} beginnen. Bei seiner Angst, seiner Verlassenheit und seiner monströsen Freiheit.
Der Existentialismus dreht die 2000 Jahre alte platonische Ordnung um. Der Leitsatz lautet:
\textit{\enquote{l'existence précède l'essence}} (Die Existenz geht der Essenz voraus).
Wir sind keine fertig definierten Wesen (wie ein Stuhl oder ein Messer), die einen Zweck haben. Wir \textit{sind} erst einfach, wir tauchen auf – und dann müssen wir uns selbst erfinden.

Doch wie soll man sich in dieser neuen, chaotischen Freiheit orientieren? Die traditionelle Metaphysik hatte ausgedient; ihre großen Begriffe von \enquote{Substanz} und \enquote{Geist} wirkten angesichts der konkreten menschlichen Erfahrung hohl und blutleer. Um die Existenz in ihrer ganzen Rohheit zu erfassen, brauchte die Philosophie eine radikale neue Methode. Sie musste lernen, wieder \enquote{naiv} zu sehen – ohne den Schleier alter Theorien, direkt auf das Erlebte gerichtet. Bevor der Existentialismus die menschliche Lage deuten konnte, musste er erst lernen, sie präzise zu beschreiben. Dies leistete eine Denkrichtung, die den Blick säuberte:

\section{Die Phänomenologie: Zurück zu den Sachen selbst (Husserl)}
Bevor Sartre das Café de Flore betrat und über die Absurdität nachdachte, musste jemand das philosophische Werkzeug dafür schmieden. Dieser Jemand war ein österreichischer Mathematiker mit einem Rauschebart, der aussah wie ein biblischer Prophet: \textbf{Edmund Husserl} (1859–1938).
Sein Schlachtruf lautete: \enquote{\textit{Zurück zu den Sachen selbst!}}

Das klingt banal, war aber eine Revolution. Die Philosophie hatte sich jahrhundertelang mit Theorien \textit{über} die Welt verstaubt. Ist die Außenwelt real? Ist alles nur Einbildung?
Husserl wischte diese Fragen vom Tisch. Er sagte: Es ist egal, ob die Kaffeetasse vor mir \enquote{wirklich} existiert oder ob ich sie träume. Entscheidend ist, \textbf{dass} sie mir erscheint. Das \textit{Phänomen} ist die einzige absolute Wahrheit, die wir haben.

\subsection{Der Bewusstseinsstrahl: Die Intentionalität}
Der wichtigste Begriff, den Husserl prägte (und den Sartre später begierig aufgriff), ist die \textbf{Intentionalität}.
Die traditionelle Philosophie (seit Descartes) dachte das Bewusstsein als einen \textbf{Container} oder eine \enquote{geschlossene Schachtel}. Darin befinden sich \enquote{Vorstellungen} (Bilder der Welt). Wenn ich an einen blühenden Baum denke, habe ich ein \textit{Bild} vom Baum in meinem Kopf. Das Problem dabei: Woher weiß ich, ob der echte Baum draußen überhaupt existiert? Ich bin in meiner Schachtel gefangen (Solipsismus).

Husserl zertrümmerte diese Schachtel.
Er sagte: \enquote{Bewusstsein ist immer \textbf{Bewusstsein \textit{von} etwas}.}
Es gibt kein leeres, isoliertes Ich. Denken heißt immer \textit{an etwas} denken. Fühlen heißt \textit{etwas} fühlen. Das Bewusstsein ist keine Substanz, sondern eine reine \textbf{Relation}, ein Verhältnis.
Sartre formulierte es später radikaler: Das Bewusstsein ist ein \textbf{Scheinwerfer}, ein Pfeil, eine Explosion nach draußen. Es hat keinen \enquote{Inhalt}. Es ist leer, durchsichtig: ein reiner Wind, der zu den Dingen weht.
Das bedeutet: Wir sind nicht \textit{in} unserem Kopf, wir sind \textit{draußen} bei den Dingen. Die Welt ist nicht ein Bild in mir, sondern ich bin in der Welt. Damit war das Gespenst des Solipsismus verjagt.
Sartre war begeistert, als er davon hörte (angeblich in einer Bar, als sein Freund Raymond Aron auf sein Aprikosen-Cocktail zeigte und rief: \enquote{Siehst du, mein Lieber, wenn du Phänomenologe bist, kannst du über diesen Cocktail sprechen, und das ist Philosophie!}).

\subsection{Der Staffelstab: Von Husserl zu Heidegger}
Husserl hatte zwar die richtige Methode (Phänomenologie) gefunden, aber er blieb für seinen Meisterschüler \textbf{Martin Heidegger} (1889–1976) auf halbem Weg stehen. Denn Husserl untersuchte immer noch das \textit{Erkennen} (Wie nehme ich eine Tasse wahr?). Er blieb ein theoretischer Zuschauer.
Heidegger aber sagte: Bevor wir Zuschauer sind, sind wir Akteure. Wir sitzen nicht im Kino und betrachten das Leben; wir spielen auf der Bühne mit, ohne das Skript zu kennen.

In seinem epochalen Werk \textit{Sein und Zeit} (1927) radikalisierte er den Ansatz:
\begin{itemize}
    \item \textbf{Zuhandenheit vs. Vorhandenheit (Der Hammer):}
    Ein klassisches Beispiel erklärt den Unterschied. Wenn ein Tischler hämmert, starrt er den Hammer nicht an. Der Hammer ist für ihn kein theoretisches Objekt (\textit{Vorhanden}), über das er nachdenkt. Der Hammer ist eins mit seiner Hand, er dient einem Zweck. Er ist \textit{zuhanden}.
    Erst wenn der Hammer \textit{zerbricht}, wird er plötzlich zum Objekt. Der Tischler hält inne und starrt das kaputte Ding an.
    Heidegger meint: Die westliche Philosophie (wie Husserl) hat immer nur den \textit{zerbrochenen Hammer} angestarrt. Sie hat das theoretische Betrachten für wichtiger gehalten als den praktischen Umgang. Aber unser primärer Zugang zur Welt ist die \textbf{Sorge} (das Hantieren, Besorgen, Kümmern), nicht das Gaffen.

    \item \textbf{Das Man (Die Uneigentlichkeit):}
    Meistens sind wir gar nicht wir selbst. Wir tun, was \enquote{man} tut. Wir wählen, was \enquote{man} wählt. Wir sagen: \enquote{Man geht heute nicht mehr ohne Maske raus} oder \enquote{Man findet diesen Film gut}.
    Dieses \textbf{Man} ist eine Diktatur des Durchschnitts. Es nimmt uns die Last der Entscheidung ab. Wir gleiten in die \textbf{Verfallenheit}, in ein unauthentisches Leben.

    \item \textbf{Sein-zum-Tode und Angst:}
    Was reißt uns aus dieser bequemen Betäubung? Die \textbf{Angst}. Nicht die Furcht vor einer Spinne (die hat ein Objekt), sondern die grundlose Angst, die uns überfällt, wenn die Welt plötzlich sinnlos erscheint.
    Und vor allem: Der \textbf{Tod}. Niemand kann mir meinen Tod abnehmen. Sterben muss ich allein. \enquote{Man} stirbt nicht; \textit{Ich} sterbe.
    Das Vorlaufen zum Tode reißt das Dasein aus dem \enquote{Man} heraus und zwingt es zur \textbf{Eigentlichkeit}.
\end{itemize}

Heidegger lieferte damit das Vokabular für die Moderne: Geworfenheit, Sorge, Angst, Tod. Doch er analysierte dies (jedenfalls in \textit{Sein und Zeit}) mit der Kälte eines Geologen, der Gesteinsschichten untersucht. Er beschrieb das Dasein, aber er rief nicht zur Revolution auf. Das überließ er seinem französischen Interpreten.

\subsection{Die Politisierung: Von Heidegger zu Sartre}
Hier geschah eines der fruchtbarsten Missverständnisse der Philosophiegeschichte.
Jean-Paul Sartre verbrachte das Jahr 1933 als Stipendiat in Berlin. Während draußen die Fackelmärsche der Nazis begannen und Hitler die Macht ergriff, saß Sartre in der Bibliothek und verschlang Husserl und Heidegger.
Er war fasziniert von der Wucht der Heideggerschen Begriffe, führte aber eine entscheidende \enquote{Französisierung} durch:

\begin{enumerate}
    \item \textbf{Die Rückkehr des Subjekts:} Heidegger wollte das \enquote{Ich} (das cartesianische Subjekt) überwinden. Für ihn war das \enquote{Dasein} wichtiger als das Bewusstsein. Sartre aber, im Herzen ein Schüler Descartes', holte das \enquote{Ich} zurück. Für ihn war das \enquote{Nichts} kein mystisches Seins-Geheimnis, sondern der \textbf{Abstand}, den das freie Bewusstsein zu den Dingen hat.
    \item \textbf{Vom Schicksal zum Entwurf:} Heideggers Begriff der \enquote{Geworfenheit} klingt nach Schicksal, nach Boden, nach Schwere. Sartre akzeptierte die Geworfenheit, setzte ihr aber den \textbf{Entwurf} (\textit{le projet}) entgegen.Wir sind vielleicht ohne Grund hier, aber wir können uns selbst in die Zukunft entwerfen. Wir sind nicht das, was wir \textit{sind} (Vergangenheit), sondern das, was wir \textit{sein werden} (Zukunft).
    \item \textbf{Von der Hütte zum Boulevard:} Heideggers Philosophie atmet die Schwere der deutschen Romantik und des Waldes. Sartres Philosophie atmet die Hektik der Großstadt, das Klickern der Espressotassen und den Rauch der Gauloises. Sie ist eine Philosophie der radikalen Entwurzelung.
\end{enumerate}

Sartre nahm die Waffe, die Heidegger geschmiedet hatte, und richtete sie auf ein anderes Ziel: Die absolute Freiheit des Individuums gegen jede Autorität (sei es Gott, der Staat oder die Tradition). Aus der deutschen \enquote{Existenzphilosophie} wurde der französische \enquote{Existentialismus}.

\section{Jean-Paul Sartre: Die Verurteilung zur Freiheit}
Mitten im besetzten Paris schrieb Sartre das Buch, das diese Synthese vollzog: \textit{Das Sein und das Nichts} (1943). Es ist Heideggers Analyse, aber mit französischem Widerstandsgeist aufgeladen. Es ist bezeichnend, dass dieses Werk der absoluten Freiheit unter der deutschen Besatzung entstand. Nie waren die Franzosen freier als damals, sagte Sartre provokant, denn jeder Akt – selbst das Schweigen – war eine absolute Wahl: Kollaboration oder Widerstand.

\subsection{Die Existenz geht der Essenz voraus}
Sartres Grundthese ist ein direkter Angriff auf die christliche Metaphysik.
Um sie zu erklären, nutzt er ein handfestes Beispiel: Ein \textbf{Papiermesser}.
Bevor ein Handwerker ein Papiermesser herstellt, hat er eine Idee (eine Essenz) im Kopf: Es muss schneiden können, es muss fest sein. Hier gilt: Die Essenz kommt \textit{vor} der Existenz.
Christen stellen sich Gott als einen \enquote{übernatürlichen Handwerker} vor. Er hat die Idee des Menschen im Kopf, bevor er ihn schafft. Jeder Mensch hat also eine Bestimmung.

Sartre aber sagt: \textbf{Es gibt keinen Schöpfer.}
Also gibt es niemanden, der uns eine Idee vorschreibt. Der Mensch taucht zuerst in der Welt auf (Existenz), und erst danach definiert er sich durch seine Taten (Essenz).
Es gibt keine \enquote{menschliche Natur}. Der Mensch ist nichts anderes als das, wozu er sich macht. Er ist ein \textbf{Entwurf}.

\subsection{Die Verurteilung zur Freiheit}
Das klingt heldenhaft, ist aber eine Bürde. Wenn es keinen Gott gibt und keine Determinismus (keine Gene, keine Kindheit, die uns entschuldigen), dann sind wir \textbf{absolut verantwortlich}.
Sartre sagt: \textit{\enquote{Der Mensch ist zur Freiheit verurteilt.}}
Verurteilt, weil er sich nicht selbst erschaffen hat, und dennoch frei, weil er für alles, was er tut, verantwortlich ist.
Diese Freiheit erzeugt \textbf{Angst} (l'angoisse). Nicht Angst vor etwas Bestimmtem, sondern Angst vor den eigenen Möglichkeiten. Wie der Schwindel am Abgrund: Ich habe Angst, nicht weil ich fallen \textit{könnte}, sondern weil ich mich hinunterstürzen \textit{könnte}. Nichts hält mich davon ab, außer ich selbst.

\subsection{Mauvaise Foi: Das Schauspiel des Kellners}
Um dieser erdrückenden Angst zu entfliehen, lügen wir uns selbst an. Wir tun so, als wären wir unfrei, als wären wir durch unseren Charakter oder unsere Umstände festgelegt. Sartre nennt das \textbf{Mauvaise Foi} (Unaufrichtigkeit / schlechter Glaube).
Das berühmteste Beispiel ist der \textbf{Kellner im Café}.
Er bewegt sich ein bisschen zu schnell, seine Gesten sind ein wenig zu präzise, seine Stimme ein wenig zu eifrig. Er \textit{spielt} Kellner. Er versucht, ganz in seiner Rolle aufzugehen, wie ein Ding, um zu vergessen, dass er jederzeit das Tablet fallen lassen und gehen könnte. Er flüchtet vor seiner Freiheit in die Rolle.

\subsection{Die Hölle, das sind die anderen}
Doch wir sind nicht allein. In Sartres Theaterstück \textit{Geschlossene Gesellschaft} fällt der Satz: \textit{\enquote{L'enfer, c'est les autres}} (Die Hölle, das sind die anderen).
Warum? Nicht, weil andere Menschen nerven. Sondern wegen des \textbf{Blicks} (Le Regard).
Solange ich allein bin, bin ich Subjekt. Ich ordne die Welt um mich herum. Doch sobald ein Anderer den Raum betritt und mich ansieht, werde ich zum \textbf{Objekt} seines Blicks. Ich werde ein Element \textit{in seiner} Welt. Er stiehlt mir meine Welt.
Der Blick des Anderen fixiert mich. Für ihn bin ich \enquote{der Kellner}, \enquote{der Intellektuelle}, \enquote{der Feigling}. Er schreibt mir eine Essenz zu, gegen die ich mich wehren muss. Der Konflikt ist also der Urzustand des menschlichen Miteinanders.

\section{Albert Camus: Die Philosophie des Absurden}
Es gab eine Zeit, da waren Jean-Paul Sartre und \textbf{Albert Camus} (1913–1960) unzertrennlich. Doch philosophisch (und später politisch) trennten sie Welten.
Während Sartre ein Kind der Pariser Bourgeoisie war, wuchs Camus in der armen, heißen Sonne Algeriens auf. Sartre war klein, hässlich und ein Büchermensch. Camus war ein Torwart, ein Frauenheld und ein Mensch der Sinne.
Camus wehrte sich stets dagegen, Existentialist genannt zu werden. Seine Philosophie kreist nicht um Existenz oder Freiheit, sondern um das \textbf{Absurde}.

\subsection{Das Schweigen der Welt}
Das Absurde ist nicht die Welt selbst. Und es ist nicht der Mensch selbst.
Das Absurde entsteht aus dem \textbf{Zusammenstoß} (der Konfrontation) zwischen zwei Dingen:
\begin{itemize}
    \item Dem brennenden Verlangen des Menschen nach Sinn und Klarheit.
    \item Dem \enquote{zärtlichen Schweigen} (\textit{tendre indifférence}) des Universums.
\end{itemize}
Wir rufen in den Wald hinein: \enquote{Warum leben wir?}, und das Echo ist Stille. Dieses Schweigen ist das Absurde. Der Mensch ist ein rationales Wesen in einem irrationalen Kosmos.

\subsection{Der Mythos des Sisyphos}
In seinem berühmtesten Essay (1942) vergleicht Camus die menschliche Lage mit der griechischen Sagengestalt \textbf{Sisyphos}.
Zur Strafe muss Sisyphos einen riesigen Felsbrocken einen Berg hinaufrollen. Oben angekommen, rollt der Stein durch sein eigenes Gewicht wieder hinunter. Und Sisyphos muss von vorne beginnen. In alle Ewigkeit. Eine sinnlose, hoffnungslose Qual.
Ist das nicht tragisch? Ja. Aber Camus dreht den Mythos um.
Der entscheidende Moment ist der Abstieg. Wenn Sisyphos dem Stein hinterhergeht. In dieser Pause ist er frei. Er weiß, dass sein Kampf sinnlos ist, und \textit{dennoch} nimmt er den Stein wieder auf. Er revoltiert gegen sein Schicksal, indem er es akzeptiert.

Sein berühmter Schlusssatz ist ein Triumph des Willens über die Sinnlosigkeit:
\textit{\enquote{Der Kampf gegen Gipfel vermag ein Menschenherz auszufüllen. Wir müssen uns Sisyphos als einen glücklichen Menschen vorstellen.}}

\subsection{Die Revolte: Ich revoltiere, also sind wir}
Wenn das Leben absurd ist, ist dann Selbstmord die Lösung? Camus sagt: Nein.
Selbstmord ist Kapitulation. Die einzig würdevolle Haltung ist die \textbf{Revolte}. Man lebt \textit{trotzdem}. Man genießt die Sonne, das Meer und die Liebe im Angesicht des Todes.
Während Sartre später den gewaltsamen Umsturz (Marxismus) predigte, setzte Camus auf die \textbf{Maßhaltigkeit} (la mesure). Seine Ethik ist humanistisch: Weil wir alle das gleiche absurde Schicksal teilen, müssen wir solidarisch sein. Sein Satz gegen Descartes lautet: \textit{\enquote{Ich revoltiere, also sind wir.}}

\section{Simone de Beauvoir: Die Freiheit des Anderen}
% TODO: "Das andere Geschlecht" (Le Deuxième Sexe).
% TODO: "Man kommt nicht als Frau zur Welt, man wird es gemacht." (Immanenz vs. Transzendenz).
% TODO: Ethik der Zweideutigkeit: Freiheit bedingt die Freiheit der anderen.

\section{Konsequenz: Vom Ekel zum Engagement}
% TODO: Der Existentialismus als Humanismus.
% TODO: Das politische Engagement (Sartre in der Linken).
% TODO: Überleitung zum Strukturalismus (Levi-Strauss), der das "Subjekt" wieder auslöschen wird.
