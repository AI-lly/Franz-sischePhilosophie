\chapter{Der Bruch: Die Existenz geht der Essenz voraus}

\section{Einleitung: Die verwaiste Welt}
Wenn die Aufklärung der Morgen war, an dem die Vernunft erwachte, so ist der Existentialismus die dunkelste Stunde der Nacht, in der der Mensch merkt, dass er allein ist.
Das 19. Jahrhundert hatte noch an den Fortschritt geglaubt. Hegel sah die Geschichte als Entfaltung des Weltgeistes, Marx als Weg zum Kommunismus. Doch dieser Optimismus starb in den Schützengräben von Verdun und später in den Lagern des 20. Jahrhunderts.

Nietzsches Diagnose vom \enquote{Tod Gottes} war keine bloße atheistische Provokation, sondern eine seismographische Erschütterung: Der metaphysische Boden war weggebrochen. Es gab keinen Himmel mehr, der Trost spendete, und keine objektive Moral, die Richtung gab. Georg Lukács nannte diesen Zustand \textit{transzendentale Obdachlosigkeit}.
Der Mensch findet sich in einem Universum wieder, das \textbf{schweigt}. Es ist nicht mehr der wohlgeordnete Kosmos der Antike oder das Uhrwerk Voltaires. Es ist eine gleichgültige, fremde Masse.

In diesem Vakuum muss die Philosophie neu ansetzen. Sie kann nicht mehr vom \enquote{Allgemeinen} ausgehen (dem Geist, der Klasse, der Nation), sondern muss beim \textbf{Einzelnen} beginnen. Bei seiner Angst, seiner Verlassenheit und seiner monströsen Freiheit.
Der Existentialismus dreht die 2000 Jahre alte platonische Ordnung um. Der Leitsatz lautet:
\textit{\enquote{l'existence précède l'essence}} (Die Existenz geht der Essenz voraus).
Wir sind keine fertig definierten Wesen (wie ein Stuhl oder ein Messer), die einen Zweck haben. Wir \textit{sind} erst einfach, wir tauchen auf – und dann müssen wir uns selbst erfinden.

Doch wie soll man sich in dieser neuen, chaotischen Freiheit orientieren? Die traditionelle Metaphysik hatte ausgedient; ihre großen Begriffe von \enquote{Substanz} und \enquote{Geist} wirkten angesichts der konkreten menschlichen Erfahrung hohl und blutleer. Um die Existenz in ihrer ganzen Rohheit zu erfassen, brauchte die Philosophie eine radikale neue Methode. Sie musste lernen, wieder \enquote{naiv} zu sehen – ohne den Schleier alter Theorien, direkt auf das Erlebte gerichtet. Bevor der Existentialismus die menschliche Lage deuten konnte, musste er erst lernen, sie präzise zu beschreiben. Dies leistete eine Denkrichtung, die den Blick säuberte:

\section{Die Phänomenologie: Zurück zu den Sachen selbst (Husserl)}
Bevor Sartre das Café de Flore betrat und über die Absurdität nachdachte, musste jemand das philosophische Werkzeug dafür schmieden. Dieser Jemand war ein österreichischer Mathematiker mit einem Rauschebart, der aussah wie ein biblischer Prophet: \textbf{Edmund Husserl} (1859–1938).
Sein Schlachtruf lautete: \enquote{\textit{Zurück zu den Sachen selbst!}}

Das klingt banal, war aber eine Revolution. Die Philosophie hatte sich jahrhundertelang mit Theorien \textit{über} die Welt verstaubt. Ist die Außenwelt real? Ist alles nur Einbildung?
Husserl wischte diese Fragen vom Tisch. Er sagte: Es ist egal, ob die Kaffeetasse vor mir \enquote{wirklich} existiert oder ob ich sie träume. Entscheidend ist, \textbf{dass} sie mir erscheint. Das \textit{Phänomen} ist die einzige absolute Wahrheit, die wir haben.

\subsection{Der Bewusstseinsstrahl: Die Intentionalität}
Der wichtigste Begriff, den Husserl prägte (und den Sartre später begierig aufgriff), ist die \textbf{Intentionalität}.
Die alte Vorstellung war: Das Bewusstsein ist wie eine Schachtel oder ein Container. Darin sind \enquote{Vorstellungen} (Bilder der Welt). Wenn ich an einen Baum denke, habe ich ein Bild vom Baum in meiner Schachtel.
Husserl zertrümmerte diese Schachtel.
Er sagte: \enquote{Bewusstsein ist immer \textbf{Bewusstsein \textit{von} etwas}.}
Es gibt kein leeres Bewusstsein. Denken heißt immer \textit{an etwas} denken. Fühlen heißt \textit{etwas} fühlen.
Das Bewusstsein ist kein Container, sondern ein Scheinwerfer, ein Pfeil, eine Explosion nach draußen. Es ist leer, durchsichtig: ein reiner Wind, der zu den Dingen weht.
Sartre war begeistert, als er davon hörte (angeblich in einer Bar, als sein Freund Raymond Aron auf sein Aprikosen-Cocktail zeigte und rief: \enquote{Siehst du, mein Lieber, wenn du Phänomenologe bist, kannst du über diesen Cocktail sprechen, und das ist Philosophie!}).

\subsection{Heidegger: Vom Wissen zum Sein}
Husserls bester Schüler, \textbf{Martin Heidegger} (1889–1976), nahm diese Methode und radikalisierte sie.
Husserl wollte wissen, wie wir die Welt \textit{erkennen} (Epistemologie). Heidegger wollte wissen, was es bedeutet, \textit{zu sein} (Ontologie).
In seinem gigantischen und schwer verdaulichen Werk \textit{Sein und Zeit} (1927) prägte er Begriffe, die den Existentialismus fundierten:
\begin{itemize}
    \item \textbf{Dasein:} Der Mensch ist kein \enquote{Subjekt}, das die Welt betrachtet. Er ist immer schon \enquote{in-der-Welt-sein}. Wir stecken mittendrin.
    \item \textbf{Geworfenheit:} Wir haben uns nicht ausgesucht zu existieren. Wir sind in diese Welt \enquote{geworfen} worden, ohne Ticket und ohne Fahrplan.
    \item \textbf{Sorge:} Unsere Existenz ist kein Ruhezustand, sondern ständige Sorge um unsere Zukunft und unseren Tod.
\end{itemize}
Heidegger lieferte das Fundament. Aber es brauchte Jean-Paul Sartre, um daraus eine Philosophie zu machen, die man nicht nur in verstaubten Bibliotheken, sondern auch in verrauchten Jazzkellern leben konnte.

\section{Jean-Paul Sartre: Die Verurteilung zur Freiheit}
Während Heidegger in seiner Schwarzwaldhütte über das \enquote{Sein} brütete, saß \textbf{Jean-Paul Sartre} (1905–1980) im Café de Flore in Paris, rauchte Kette und schrieb das Buch, das eine ganze Generation prägen sollte: \textit{Das Sein und das Nichts} (1943).
Es ist bezeichnend, dass dieses Werk der absoluten Freiheit mitten in der deutschen Besatzung entstand. Nie waren die Franzosen freier als unter der Besatzung, sagte Sartre provokant. Denn jeder Akt – selbst das Schweigen – war eine absolute Wahl: Kollaboration oder Widerstand.

\subsection{Die Existenz geht der Essenz voraus}
Sartres Grundthese ist ein direkter Angriff auf die christliche Metaphysik.
Um sie zu erklären, nutzt er ein handfestes Beispiel: Ein \textbf{Papiermesser}.
Bevor ein Handwerker ein Papiermesser herstellt, hat er eine Idee (eine Essenz) im Kopf: Es muss schneiden können, es muss fest sein. Hier gilt: Die Essenz kommt \textit{vor} der Existenz.
Christen stellen sich Gott als einen \enquote{übernatürlichen Handwerker} vor. Er hat die Idee des Menschen im Kopf, bevor er ihn schafft. Jeder Mensch hat also eine Bestimmung.

Sartre aber sagt: \textbf{Es gibt keinen Schöpfer.}
Also gibt es niemanden, der uns eine Idee vorschreibt. Der Mensch taucht zuerst in der Welt auf (Existenz), und erst danach definiert er sich durch seine Taten (Essenz).
Es gibt keine \enquote{menschliche Natur}. Der Mensch ist nichts anderes als das, wozu er sich macht. Er ist ein \textbf{Entwurf}.

\subsection{Die Verurteilung zur Freiheit}
Das klingt heldenhaft, ist aber eine Bürde. Wenn es keinen Gott gibt und keine Determinismus (keine Gene, keine Kindheit, die uns entschuldigen), dann sind wir \textbf{absolut verantwortlich}.
Sartre sagt: \textit{\enquote{Der Mensch ist zur Freiheit verurteilt.}}
Verurteilt, weil er sich nicht selbst erschaffen hat, und dennoch frei, weil er für alles, was er tut, verantwortlich ist.
Diese Freiheit erzeugt \textbf{Angst} (l'angoisse). Nicht Angst vor etwas Bestimmtem, sondern Angst vor den eigenen Möglichkeiten. Wie der Schwindel am Abgrund: Ich habe Angst, nicht weil ich fallen \textit{könnte}, sondern weil ich mich hinunterstürzen \textit{könnte}. Nichts hält mich davon ab, außer ich selbst.

\subsection{Mauvaise Foi: Das Schauspiel des Kellners}
Um dieser erdrückenden Angst zu entfliehen, lügen wir uns selbst an. Wir tun so, als wären wir unfrei, als wären wir durch unseren Charakter oder unsere Umstände festgelegt. Sartre nennt das \textbf{Mauvaise Foi} (Unaufrichtigkeit / schlechter Glaube).
Das berühmteste Beispiel ist der \textbf{Kellner im Café}.
Er bewegt sich ein bisschen zu schnell, seine Gesten sind ein wenig zu präzise, seine Stimme ein wenig zu eifrig. Er \textit{spielt} Kellner. Er versucht, ganz in seiner Rolle aufzugehen, wie ein Ding, um zu vergessen, dass er jederzeit das Tablet fallen lassen und gehen könnte. Er flüchtet vor seiner Freiheit in die Rolle.

\subsection{Die Hölle, das sind die anderen}
Doch wir sind nicht allein. In Sartres Theaterstück \textit{Geschlossene Gesellschaft} fällt der Satz: \textit{\enquote{L'enfer, c'est les autres}} (Die Hölle, das sind die anderen).
Warum? Nicht, weil andere Menschen nerven. Sondern wegen des \textbf{Blicks} (Le Regard).
Solange ich allein bin, bin ich Subjekt. Ich ordne die Welt um mich herum. Doch sobald ein Anderer den Raum betritt und mich ansieht, werde ich zum \textbf{Objekt} seines Blicks. Ich werde ein Element \textit{in seiner} Welt. Er stiehlt mir meine Welt.
Der Blick des Anderen fixiert mich. Für ihn bin ich \enquote{der Kellner}, \enquote{der Intellektuelle}, \enquote{der Feigling}. Er schreibt mir eine Essenz zu, gegen die ich mich wehren muss. Der Konflikt ist also der Urzustand des menschlichen Miteinanders.

\section{Albert Camus: Die Philosophie des Absurden}
% TODO: Abgrenzung: Warum Camus kein Existentialist ist (Divergenz zu Sartre).
% TODO: Der Mythos des Sisyphos: Die Absurdität der menschlichen Lage (Sinnsuche vs. Schweigen der Welt).
% TODO: Die Revolte: Der Mensch, der "Nein" sagt. Solidarität statt Revolution.

\section{Simone de Beauvoir: Die Freiheit des Anderen}
% TODO: "Das andere Geschlecht" (Le Deuxième Sexe).
% TODO: "Man kommt nicht als Frau zur Welt, man wird es gemacht." (Immanenz vs. Transzendenz).
% TODO: Ethik der Zweideutigkeit: Freiheit bedingt die Freiheit der anderen.

\section{Konsequenz: Vom Ekel zum Engagement}
% TODO: Der Existentialismus als Humanismus.
% TODO: Das politische Engagement (Sartre in der Linken).
% TODO: Überleitung zum Strukturalismus (Levi-Strauss), der das "Subjekt" wieder auslöschen wird.
