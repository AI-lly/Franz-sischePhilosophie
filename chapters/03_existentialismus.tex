\chapter{Der Bruch: Die Existenz geht der Essenz voraus}

\section{Einleitung: Die verwaiste Welt}
Wenn die Aufklärung der Morgen war, an dem die Vernunft erwachte, so ist der Existentialismus die dunkelste Stunde der Nacht, in der der Mensch merkt, dass er allein ist.
Das 19. Jahrhundert hatte noch an den Fortschritt geglaubt. Hegel sah die Geschichte als Entfaltung des Weltgeistes, Marx als Weg zum Kommunismus. Doch dieser Optimismus starb in den Schützengräben von Verdun und später in den Lagern des 20. Jahrhunderts.

Nietzsches Diagnose vom \enquote{Tod Gottes} war keine bloße atheistische Provokation, sondern eine seismographische Erschütterung: Der metaphysische Boden war weggebrochen. Es gab keinen Himmel mehr, der Trost spendete, und keine objektive Moral, die Richtung gab. Georg Lukács nannte diesen Zustand \textit{transzendentale Obdachlosigkeit}.
Der Mensch findet sich in einem Universum wieder, das \textbf{schweigt}. Es ist nicht mehr der wohlgeordnete Kosmos der Antike oder das Uhrwerk Voltaires. Es ist eine gleichgültige, fremde Masse.

In diesem Vakuum muss die Philosophie neu ansetzen. Sie kann nicht mehr vom \enquote{Allgemeinen} ausgehen (dem Geist, der Klasse, der Nation), sondern muss beim \textbf{Einzelnen} beginnen. Bei seiner Angst, seiner Verlassenheit und seiner monströsen Freiheit.
Der Existentialismus dreht die 2000 Jahre alte platonische Ordnung um. Der Leitsatz lautet:
\textit{\enquote{l'existence précède l'essence}} (Die Existenz geht der Essenz voraus).
Wir sind keine fertig definierten Wesen (wie ein Stuhl oder ein Messer), die einen Zweck haben. Wir \textit{sind} erst einfach, wir tauchen auf – und dann müssen wir uns selbst erfinden.

Doch wie soll man sich in dieser neuen, chaotischen Freiheit orientieren? Die traditionelle Metaphysik hatte ausgedient; ihre großen Begriffe von \enquote{Substanz} und \enquote{Geist} wirkten angesichts der konkreten menschlichen Erfahrung hohl und blutleer. Um die Existenz in ihrer ganzen Rohheit zu erfassen, brauchte die Philosophie eine radikale neue Methode. Sie musste lernen, wieder \enquote{naiv} zu sehen – ohne den Schleier alter Theorien, direkt auf das Erlebte gerichtet. Bevor der Existentialismus die menschliche Lage deuten konnte, musste er erst lernen, sie präzise zu beschreiben. Dies leistete eine Denkrichtung, die den Blick säuberte:

\subsection{Die phänomenologische Methode: Epoché und Intentionalität}
Um die Welt wirklich zu verstehen, müssen wir aufhören, sie einfach nur zu \enquote{konsumieren}.
Normalerweise leben wir im Autopiloten, in der sogenannten \textbf{\enquote{natürlichen Einstellung}}. Wir glauben naiv, dass die Welt da draußen einfach existiert und unser Kopf sie wie eine Kamera abfotografiert.

\begin{tcolorbox}[title=Das Kino-Gleichnis]
Wir sind wie Kinobesucher, die sich so sehr im Film verlieren, dass sie die Leinwand und den Projektor vergessen.
Husserl verlangt nun einen radikalen Schritt zurück: Die \textbf{Epoché} (das Einklammern).
Wir sollen nicht leugnen, dass der Film (die Welt) existiert. Aber wir sollen aufhören, uns in der Handlung zu verlieren, und stattdessen auf den Projektor schauen. Wir klammern unser Urteil \enquote{Das ist echt} ein und untersuchen stattdessen unser \textbf{Erleben}.
\end{tcolorbox}

Statt zu fragen: \enquote{Existiert dieser Kaffee wirklich?}, fragen wir: \enquote{Wie erscheint mir dieser Kaffee im Bewusstsein?}
In dem Moment, in dem wir das tun – wenn wir den Blick vom Objekt auf das Bewusstsein selbst wenden – machen wir eine schockierende Entdeckung.

Die alte Philosophie (seit Descartes) dachte immer, das Bewusstsein sei ein \textbf{Container} oder eine geschlossene Schachtel, in der Bilder (\enquote{Vorstellungen}) liegen.
Doch dank der Epoché sehen wir, dass das falsch ist. Wenn wir den Projektor ansehen, finden wir keine \enquote{Dinge} in ihm. Wir finden nur einen Lichtstrahl.
Wir finden in uns... nichts. Keine Kiste, keinen kleinen Mann im Ohr. Wir finden nur eine reine \textit{Richtung}. Einen Pfeil, der nach draußen zeigt.

\begin{tcolorbox}[title=Die Kern-Definition: Intentionalität]
Husserls bahnbrechende Erkenntnis lautet: \textit{\enquote{Bewusstsein ist immer Bewusstsein \textbf{von} etwas.}}
Das Bewusstsein ist kein Ort, an dem Dinge aufbewahrt werden. Es ist eine reine Aktivität, die auf Dinge \textit{zielt}.
Das Bewusstsein ist eine \textbf{Explosion}. Es explodiert ständig aus sich heraus hin zur Welt. Es ist ein transparenter Wind, der durch uns hindurchweht zu den Dingen. Es ist leer.
\end{tcolorbox}

Und genau deshalb sind wir unmittelbar \textit{in der Welt} und nicht in unserem Kopf gefangen.
Damit löst sich auch der Solipsismus (\enquote{Bin nur ich real?}) in Luft auf. Denn ohne Welt gäbe es gar kein Bewusstsein, da es nichts gäbe, worauf es sich richten könnte. Ein Pfeil ohne Ziel ist kein Pfeil.

Als Sartre diese Theorie zum ersten Mal hörte (angeblich in einer Bar, als sein Freund Raymond Aron auf seinen Aprikosen-Cocktail zeigte und rief: \enquote{Siehst du, mein Lieber, wenn du Phänomenologe bist, kannst du über diesen Cocktail sprechen, und das ist Philosophie!}), wurde er kreidebleich vor Begeisterung. Endlich konnte man über das konkrete Leben philosophieren, ohne die Welt zu verlassen.

\subsection{Eine neue Grammatik der Wirklichkeit: Das Erbe Husserls und die Wende Heideggers}
Was hat Husserl damit eigentlich bewirkt? Er hat die Philosophie von ihrem alten Dogma befreit, dass nur Zahlen und Atome \enquote{wahr} seien.

\begin{tcolorbox}[title=Das Erbe: Die Rettung der Lebenswelt]
Husserl lehrte uns, das Subjektive \textbf{ernst zu nehmen}. Für die Phänomenologie ist der Schmerz, den ich fühle, genauso \enquote{real} und objektiv beschreibbar wie eine mathematische Formel.
Sein wichtigster Begriff ist die \textbf{Lebenswelt}: Die Welt der Wissenschaft (Atome) ist nur ein abstraktes Modell. Aber die \textit{eigentliche} Welt ist die, in der wir leben, lieben und sterben. Die Welt der Farben, Töne und Werte. Husserl gab der Philosophie ihre Augen zurück.
\end{tcolorbox}

Doch genau hier hakte sein Meisterschüler \textbf{Martin Heidegger} (1889–1976) ein.
Husserl hatte zwar das \textit{Sehen} revolutioniert, aber er blieb ein \textbf{Zuschauer}. Er untersuchte immer noch, wie wir Dinge \textit{erkennen} (z.B. eine Kaffeetasse).
Heidegger aber sagte: Bevor wir Zuschauer sind, sind wir Akteure. Wir sitzen nicht im Kino und betrachten das Leben; wir spielen auf der Bühne mit, ohne das Skript zu kennen.

In seinem epochalen Werk \textit{Sein und Zeit} (1927) radikalisierte er den Ansatz:
\begin{itemize}
    \item \textbf{Zuhandenheit vs. Vorhandenheit (Der Hammer):}
    Ein klassisches Beispiel erklärt den Unterschied. Wenn ein Tischler hämmert, starrt er den Hammer nicht an. Der Hammer ist für ihn kein theoretisches Objekt (\textit{Vorhanden}), über das er nachdenkt. Der Hammer ist eins mit seiner Hand, er dient einem Zweck. Er ist \textit{zuhanden}.
    Erst wenn der Hammer \textit{zerbricht}, wird er plötzlich zum Objekt. Der Tischler hält inne und starrt das kaputte Ding an.
    Heidegger meint: Die westliche Philosophie (wie Husserl) hat immer nur den \textit{zerbrochenen Hammer} angestarrt. Sie hat das theoretische Betrachten für wichtiger gehalten als den praktischen Umgang. Aber unser primärer Zugang zur Welt ist die \textbf{Sorge} (das Hantieren, Besorgen, Kümmern), nicht das Gaffen.

    \item \textbf{Das Man (Die Uneigentlichkeit):}
    Meistens sind wir gar nicht wir selbst. Wir tun, was \enquote{man} tut. Wir wählen, was \enquote{man} wählt. Wir sagen: \enquote{Man geht heute nicht mehr ohne Maske raus} oder \enquote{Man findet diesen Film gut}.
    Dieses \textbf{Man} ist eine Diktatur des Durchschnitts. Es nimmt uns die Last der Entscheidung ab. Wir gleiten in die \textbf{Verfallenheit}, in ein unauthentisches Leben.

    \item \textbf{Sein-zum-Tode und Angst:}
    Was reißt uns aus dieser bequemen Betäubung? Die \textbf{Angst}. Nicht die Furcht vor einer Spinne (die hat ein Objekt), sondern die grundlose Angst, die uns überfällt, wenn die Welt plötzlich sinnlos erscheint.
    Und vor allem: Der \textbf{Tod}. Niemand kann mir meinen Tod abnehmen. Sterben muss ich allein. \enquote{Man} stirbt nicht; \textit{Ich} sterbe.
    Das Vorlaufen zum Tode reißt das Dasein aus dem \enquote{Man} heraus und zwingt es zur \textbf{Eigentlichkeit}.
\end{itemize}

Heidegger lieferte damit das Vokabular für die Moderne: Geworfenheit, Sorge, Angst, Tod. Doch er analysierte dies (jedenfalls in \textit{Sein und Zeit}) mit der Kälte eines Geologen, der Gesteinsschichten untersucht. Er beschrieb das Dasein, aber er rief nicht zur Revolution auf. Das überließ er seinem französischen Interpreten.

\subsection{Die Politisierung: Von Heidegger zu Sartre}
Hier geschah eines der fruchtbarsten Missverständnisse der Philosophiegeschichte.
Jean-Paul Sartre verbrachte das Jahr 1933 als Stipendiat in Berlin. Während draußen die Fackelmärsche der Nazis begannen und Hitler die Macht ergriff, saß Sartre in der Bibliothek und verschlang Husserl und Heidegger.
Er war fasziniert von der Wucht der Heideggerschen Begriffe, führte aber eine entscheidende \enquote{Französisierung} durch:

\begin{enumerate}
    \item \textbf{Die Rückkehr des Subjekts:} Heidegger wollte das \enquote{Ich} (das cartesianische Subjekt) überwinden. Für ihn war das \enquote{Dasein} wichtiger als das Bewusstsein. Sartre aber, im Herzen ein Schüler Descartes', holte das \enquote{Ich} zurück. Für ihn war das \enquote{Nichts} kein mystisches Seins-Geheimnis, sondern der \textbf{Abstand}, den das freie Bewusstsein zu den Dingen hat.
    \item \textbf{Vom Schicksal zum Entwurf:} Heideggers Begriff der \enquote{Geworfenheit} klingt nach Schicksal, nach Boden, nach Schwere. Sartre akzeptierte die Geworfenheit, setzte ihr aber den \textbf{Entwurf} (\textit{le projet}) entgegen.Wir sind vielleicht ohne Grund hier, aber wir können uns selbst in die Zukunft entwerfen. Wir sind nicht das, was wir \textit{sind} (Vergangenheit), sondern das, was wir \textit{sein werden} (Zukunft).
    \item \textbf{Von der Hütte zum Boulevard:} Heideggers Philosophie atmet die Schwere der deutschen Romantik und des Waldes. Sartres Philosophie atmet die Hektik der Großstadt, das Klickern der Espressotassen und den Rauch der Gauloises. Sie ist eine Philosophie der radikalen Entwurzelung.
\end{enumerate}

Sartre nahm die Waffe, die Heidegger geschmiedet hatte, und richtete sie auf ein anderes Ziel: Die absolute Freiheit des Individuums gegen jede Autorität (sei es Gott, der Staat oder die Tradition). Aus der deutschen \enquote{Existenzphilosophie} wurde der französische \enquote{Existentialismus}.

\section{Jean-Paul Sartre: Die Verurteilung zur Freiheit}
Mitten im besetzten Paris, in der Kälte eines ungeheizten Zimmers, schrieb Jean-Paul Sartre das Werk, das wie ein Meteorit in die Geistesgeschichte einschlug: \textit{Das Sein und das Nichts} (1943). Es ist ein philosophischer Koloss von 700 Seiten, der den französischen Widerstandsgeist mit der deutschen Metaphysik verschmilzt.
Um dieses Werk zu verstehen, muss man die Synthese begreifen, die Sartre hier vollzieht. Er steht auf den Schultern zweier Riesen, aber er blickt in eine völlig andere Richtung.

\textbf{1. Das Erbe Husserls (Die Leere):}
Von Husserl übernimmt Sartre die \textit{Intentionalität}. Er akzeptiert, dass das Bewusstsein keine \enquote{Substanz} ist, sondern reine Ausrichtung auf die Welt. Aber Sartre radikalisiert diesen Gedanken: Wenn das Bewusstsein nur ein Wind ist, der zu den Dingen weht, dann ist es in sich selbst \textit{Nichts}. Es ist eine Leerstelle im Sein.

\textbf{2. Das Erbe Heideggers (Das In-der-Welt-sein):}
Von Heidegger übernimmt er das Vokabular der Existenz: die Faktizität, die Geworfenheit, den Entwurf. Er stimmt zu: Wir sind keine abstrakten Geister, sondern wir existieren \textit{in Situationen}.

\textbf{3. Der Bruch: Das An-sich und das Für-sich}
Hier vollzieht Sartre seine revolutionäre Wendung. Er teilt die Realität rigoros in zwei Bereiche, die sich unversöhnlich gegenüberstehen:
\begin{itemize}
    \item \textbf{Das An-sich-Sein (L’en-soi):} Das ist die Welt der Dinge. Der Stein, der Tisch, der Baum. Sie sind massiv, voll, sie ruhen in sich selbst. Sie sind einfach, was sie sind. Sie haben keine Freiheit, keine Lücke, keine Zukunft. Sie sind pure Identität.
    \item \textbf{Das Für-sich-Sein (Le pour-soi):} Das ist der Mensch (das Bewusstsein). Weil wir uns unserer selbst bewusst sind, haben wir Distanz zu uns selbst. Wir sind nie deckungsgleich mit uns wie der Stein.
\end{itemize}

\begin{tcolorbox}[title=Der Unterschied zu Heidegger]
Hier liegt der entscheidende Unterschied:
\textbf{Heidegger} sah das \enquote{Dasein} als eine \textbf{Einheit}. Für ihn sind Mensch und Welt untrennbar verwoben (\enquote{In-der-Welt-sein}). Er wollte die Trennung von Subjekt und Objekt überwinden.
\textbf{Sartre} hingegen reißt diesen Graben wieder auf. Er ist ein radikaler Dualist.
Für Sartre gibt es einen \textbf{Krieg} zwischen dem Bewusstsein (dem Nichts) und der Materie (dem Sein).
Das \enquote{Für-sich} ist kein Teil der Welt wie bei Heidegger, sondern ein Fremdkörper, der die Welt wie ein \enquote{Loch} durchlöchert. Sartre holt damit Descartes' Ich zurück, macht es aber zu einem \enquote{nichts-enden} Ich.
\end{tcolorbox}

\paragraph{Die Begründung: Woher kommt das Nichts?}
Aber warum behauptet Sartre so etwas Radikales? Ist das nicht reine Wortspielerei?
Sartres Beweis ist phänomenologisch: Er analysiert unsere Erfahrung der \textbf{Verneinung} (Negativität).
Materie ist rein positiv. Ein Stein \textit{ist}. Er ist niemals \enquote{nicht}. Er enthält kein \enquote{Nein}.
Doch der Mensch kann fragen, zweifeln und verneinen.
Sartre bringt das berühmte Beispiel von \textbf{Pierre im Café}:
Ich verabrede mich mit Pierre im Café. Ich komme zu spät, schaue in den Raum und suche ihn. Das Café ist voll mit Menschen, Tischen und Rauch. Alles ist \enquote{Sein}, alles ist voll.
Aber ich sehe schlagartig: \textit{Pierre ist nicht da.}
Ich sehe seine \textbf{Abwesenheit} fast so deutlich wie die Tische. Das Caféhintergrund tritt zurück, und im Vordergrund steht das \enquote{Nicht-Da-Sein} von Pierre.
Woher kommt dieses \enquote{Nicht}? Das Café (die Materie) produziert keine Abwesenheit. Das Café ist einfach nur voll.
\textit{Ich} bin es, der die Erwartung mitbringt und das \enquote{Nicht} in den Raum projiziert.
Weil ich Fragen stellen kann (\enquote{Ist Pierre da?}) und eine negative Antwort erhalten kann (\enquote{Nein}), muss ich ein Wesen sein, das eine Distanz zum Sein hat. Ein Wesen, das reines Sein ist (wie ein Stein), könnte niemals etwas vermissen.
Nur ein Wesen, das das \textbf{Nichts} in sich trägt, kann das Nichts in der Welt entdecken.

Sartres geniale Definition lautet daher: Der Mensch ist das Wesen, durch das das \textbf{Nichts} in die Welt kommt.
Wir sind wie ein \enquote{Wurm im Apfel des Seins}. Wir höhlen das massive Sein aus, indem wir Fragen stellen, warten, vermissen und uns Alternativen vorstellen.
Diese ontologische Lücke – dieses \enquote{Nicht-Sein} in unserem Herzen – ist genau das, was wir \textbf{Freiheit} nennen. Wir sind frei, nicht weil wir toll oder mächtig sind, sondern weil wir undeterminiert sind. Wir sind ein Mangel, der gefüllt werden muss.

Es ist bezeichnend, dass dieses Werk der absoluten Freiheit unter der deutschen Besatzung entstand. Nie waren die Franzosen freier als damals, schrieb Sartre in einem berühmten Artikel, denn die Besatzung nahm ihnen die Normalität. Jeder Akt – selbst das Kaufen einer Zeitung oder das Schweigen vor einem Offizier – war plötzlich keine Routine mehr, sondern eine absolute moralische Wahl: Kollaboration oder Widerstand. Aus der Ontologie des Nichts folgt somit die Ethik der totalen Verantwortung.

\subsection{Die Existenz geht der Essenz voraus}
Sartres Grundthese ist ein direkter Angriff auf die christliche Metaphysik.
Um sie zu erklären, nutzt er ein handfestes Beispiel: Ein \textbf{Papiermesser}.
Bevor ein Handwerker ein Papiermesser herstellt, hat er eine Idee (eine Essenz) im Kopf: Es muss schneiden können, es muss fest sein. Hier gilt: Die Essenz kommt \textit{vor} der Existenz.
Christen stellen sich Gott als einen \enquote{übernatürlichen Handwerker} vor. Er hat die Idee des Menschen im Kopf, bevor er ihn schafft. Jeder Mensch hat also eine Bestimmung.

Sartre aber sagt: \textbf{Es gibt keinen Schöpfer.}
Also gibt es niemanden, der uns eine Idee vorschreibt. Der Mensch taucht zuerst in der Welt auf (Existenz), und erst danach definiert er sich durch seine Taten (Essenz).
Es gibt keine \enquote{menschliche Natur}. Der Mensch ist nichts anderes als das, wozu er sich macht. Er ist ein \textbf{Entwurf}.

\subsection{Die Verurteilung zur Freiheit}
Das klingt heldenhaft, ist aber eine Bürde. Wenn es keinen Gott gibt und keine Determinismus (keine Gene, keine Kindheit, die uns entschuldigen), dann sind wir \textbf{absolut verantwortlich}.
Sartre sagt: \textit{\enquote{Der Mensch ist zur Freiheit verurteilt.}}
Verurteilt, weil er sich nicht selbst erschaffen hat, und dennoch frei, weil er für alles, was er tut, verantwortlich ist.
Diese Freiheit erzeugt \textbf{Angst} (l'angoisse). Nicht Angst vor etwas Bestimmtem, sondern Angst vor den eigenen Möglichkeiten. Wie der Schwindel am Abgrund: Ich habe Angst, nicht weil ich fallen \textit{könnte}, sondern weil ich mich hinunterstürzen \textit{könnte}. Nichts hält mich davon ab, außer ich selbst.

\subsection{Mauvaise Foi: Das Schauspiel des Kellners}
Um dieser erdrückenden Angst zu entfliehen, lügen wir uns selbst an. Wir tun so, als wären wir unfrei, als wären wir durch unseren Charakter oder unsere Umstände festgelegt. Sartre nennt das \textbf{Mauvaise Foi} (Unaufrichtigkeit / schlechter Glaube).
Das berühmteste Beispiel ist der \textbf{Kellner im Café}.
Er bewegt sich ein bisschen zu schnell, seine Gesten sind ein wenig zu präzise, seine Stimme ein wenig zu eifrig. Er \textit{spielt} Kellner. Er versucht, ganz in seiner Rolle aufzugehen, wie ein Ding, um zu vergessen, dass er jederzeit das Tablet fallen lassen und gehen könnte. Er flüchtet vor seiner Freiheit in die Rolle.

\subsection{Die Hölle, das sind die anderen}
Doch wir sind nicht allein. In Sartres Theaterstück \textit{Geschlossene Gesellschaft} fällt der Satz: \textit{\enquote{L'enfer, c'est les autres}} (Die Hölle, das sind die anderen).
Warum? Nicht, weil andere Menschen nerven. Sondern wegen des \textbf{Blicks} (Le Regard).
Solange ich allein bin, bin ich Subjekt. Ich ordne die Welt um mich herum. Doch sobald ein Anderer den Raum betritt und mich ansieht, werde ich zum \textbf{Objekt} seines Blicks. Ich werde ein Element \textit{in seiner} Welt. Er stiehlt mir meine Welt.
Der Blick des Anderen fixiert mich. Für ihn bin ich \enquote{der Kellner}, \enquote{der Intellektuelle}, \enquote{der Feigling}. Er schreibt mir eine Essenz zu, gegen die ich mich wehren muss. Der Konflikt ist also der Urzustand des menschlichen Miteinanders.

\section{Albert Camus: Die Philosophie des Absurden}
Es gab eine Zeit, da waren Jean-Paul Sartre und \textbf{Albert Camus} (1913–1960) unzertrennlich. Doch philosophisch (und später politisch) trennten sie Welten.
Während Sartre ein Kind der Pariser Bourgeoisie war, wuchs Camus in der armen, heißen Sonne Algeriens auf. Sartre war klein, hässlich und ein Büchermensch. Camus war ein Torwart, ein Frauenheld und ein Mensch der Sinne.
Camus wehrte sich stets dagegen, Existentialist genannt zu werden. Seine Philosophie kreist nicht um Existenz oder Freiheit, sondern um das \textbf{Absurde}.

\subsection{Das Schweigen der Welt}
Das Absurde ist nicht die Welt selbst. Und es ist nicht der Mensch selbst.
Das Absurde entsteht aus dem \textbf{Zusammenstoß} (der Konfrontation) zwischen zwei Dingen:
\begin{itemize}
    \item Dem brennenden Verlangen des Menschen nach Sinn und Klarheit.
    \item Dem \enquote{zärtlichen Schweigen} (\textit{tendre indifférence}) des Universums.
\end{itemize}
Wir rufen in den Wald hinein: \enquote{Warum leben wir?}, und das Echo ist Stille. Dieses Schweigen ist das Absurde. Der Mensch ist ein rationales Wesen in einem irrationalen Kosmos.

\subsection{Der Mythos des Sisyphos}
In seinem berühmtesten Essay (1942) vergleicht Camus die menschliche Lage mit der griechischen Sagengestalt \textbf{Sisyphos}.
Zur Strafe muss Sisyphos einen riesigen Felsbrocken einen Berg hinaufrollen. Oben angekommen, rollt der Stein durch sein eigenes Gewicht wieder hinunter. Und Sisyphos muss von vorne beginnen. In alle Ewigkeit. Eine sinnlose, hoffnungslose Qual.
Ist das nicht tragisch? Ja. Aber Camus dreht den Mythos um.
Der entscheidende Moment ist der Abstieg. Wenn Sisyphos dem Stein hinterhergeht. In dieser Pause ist er frei. Er weiß, dass sein Kampf sinnlos ist, und \textit{dennoch} nimmt er den Stein wieder auf. Er revoltiert gegen sein Schicksal, indem er es akzeptiert.

Sein berühmter Schlusssatz ist ein Triumph des Willens über die Sinnlosigkeit:
\textit{\enquote{Der Kampf gegen Gipfel vermag ein Menschenherz auszufüllen. Wir müssen uns Sisyphos als einen glücklichen Menschen vorstellen.}}

\subsection{Die Revolte: Ich revoltiere, also sind wir}
Wenn das Leben absurd ist, ist dann Selbstmord die Lösung? Camus sagt: Nein.
Selbstmord ist Kapitulation. Die einzig würdevolle Haltung ist die \textbf{Revolte}. Man lebt \textit{trotzdem}. Man genießt die Sonne, das Meer und die Liebe im Angesicht des Todes.
Während Sartre später den gewaltsamen Umsturz (Marxismus) predigte, setzte Camus auf die \textbf{Maßhaltigkeit} (la mesure). Seine Ethik ist humanistisch: Weil wir alle das gleiche absurde Schicksal teilen, müssen wir solidarisch sein. Sein Satz gegen Descartes lautet: \textit{\enquote{Ich revoltiere, also sind wir.}}

\section{Simone de Beauvoir: Die Freiheit des Anderen}
% TODO: "Das andere Geschlecht" (Le Deuxième Sexe).
% TODO: "Man kommt nicht als Frau zur Welt, man wird es gemacht." (Immanenz vs. Transzendenz).
% TODO: Ethik der Zweideutigkeit: Freiheit bedingt die Freiheit der anderen.

\section{Konsequenz: Vom Ekel zum Engagement}
% TODO: Der Existentialismus als Humanismus.
% TODO: Das politische Engagement (Sartre in der Linken).
% TODO: Überleitung zum Strukturalismus (Levi-Strauss), der das "Subjekt" wieder auslöschen wird.
