\chapter{Der Bruch: Die Existenz geht der Essenz voraus}

\section{Einleitung: Die verwaiste Welt}
Wenn die Aufklärung der Morgen war, an dem die Vernunft erwachte, so ist der Existentialismus die dunkelste Stunde der Nacht, in der der Mensch merkt, dass er allein ist.
Das 19. Jahrhundert hatte noch an den Fortschritt geglaubt. Hegel sah die Geschichte als Entfaltung des Weltgeistes, Marx als Weg zum Kommunismus. Doch dieser Optimismus starb in den Schützengräben von Verdun und später in den Lagern des 20. Jahrhunderts.

Nietzsches Diagnose vom \enquote{Tod Gottes} war keine bloße atheistische Provokation, sondern eine seismographische Erschütterung: Der metaphysische Boden war weggebrochen. Es gab keinen Himmel mehr, der Trost spendete, und keine objektive Moral, die Richtung gab. Georg Lukács nannte diesen Zustand \textit{transzendentale Obdachlosigkeit}.
Der Mensch findet sich in einem Universum wieder, das \textbf{schweigt}. Es ist nicht mehr der wohlgeordnete Kosmos der Antike oder das Uhrwerk Voltaires. Es ist eine gleichgültige, fremde Masse.

In diesem Vakuum muss die Philosophie neu ansetzen. Sie kann nicht mehr vom \enquote{Allgemeinen} ausgehen (dem Geist, der Klasse, der Nation), sondern muss beim \textbf{Einzelnen} beginnen. Bei seiner Angst, seiner Verlassenheit und seiner monströsen Freiheit.
Der Existentialismus dreht die 2000 Jahre alte platonische Ordnung um. Der Leitsatz lautet:
\textit{\enquote{l'existence précède l'essence}} (Die Existenz geht der Essenz voraus).
Wir sind keine fertig definierten Wesen (wie ein Stuhl oder ein Messer), die einen Zweck haben. Wir \textit{sind} erst einfach, wir tauchen auf – und dann müssen wir uns selbst erfinden.

\section{Die Phänomenologie: Zurück zu den Sachen selbst (Husserl)}
% TODO: Edmund Husserl und die Überwindung des Psychologismus.
% TODO: Intentionalität: Bewusstsein ist immer "Bewusstsein von etwas". Die Welt ist Phänomen.
% TODO: Martin Heidegger (kurzer Exkurs): Das Dasein und die Sorge (als Vorläufer Sartres).

\section{Jean-Paul Sartre: Die Verurteilung zur Freiheit}
% TODO: "Das Sein und das Nichts" (L'être et le néant).
% TODO: Der fundamentale Satz: "Die Existenz geht der Essenz voraus" (Beispiel Papiermesser).
% TODO: Die radikale Freiheit und die Angst (Vertigo/Schwindel).
% TODO: Mauvaise Foi (Unaufrichtigkeit): Das Beispiel des Kellners.
% TODO: Der Andere: "Die Hölle, das sind die anderen" (Der Blick/Le Regard).

\section{Albert Camus: Die Philosophie des Absurden}
% TODO: Abgrenzung: Warum Camus kein Existentialist ist (Divergenz zu Sartre).
% TODO: Der Mythos des Sisyphos: Die Absurdität der menschlichen Lage (Sinnsuche vs. Schweigen der Welt).
% TODO: Die Revolte: Der Mensch, der "Nein" sagt. Solidarität statt Revolution.

\section{Simone de Beauvoir: Die Freiheit des Anderen}
% TODO: "Das andere Geschlecht" (Le Deuxième Sexe).
% TODO: "Man kommt nicht als Frau zur Welt, man wird es gemacht." (Immanenz vs. Transzendenz).
% TODO: Ethik der Zweideutigkeit: Freiheit bedingt die Freiheit der anderen.

\section{Konsequenz: Vom Ekel zum Engagement}
% TODO: Der Existentialismus als Humanismus.
% TODO: Das politische Engagement (Sartre in der Linken).
% TODO: Überleitung zum Strukturalismus (Levi-Strauss), der das "Subjekt" wieder auslöschen wird.
