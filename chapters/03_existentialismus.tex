\chapter{Der Bruch: Die Existenz geht der Essenz voraus}

\section{Einleitung: Die verwaiste Welt}
Wenn die Aufklärung der Morgen war, an dem die Vernunft erwachte, so ist der Existentialismus die dunkelste Stunde der Nacht, in der der Mensch merkt, dass er allein ist.
Das 19. Jahrhundert hatte noch an den Fortschritt geglaubt. Hegel sah die Geschichte als Entfaltung des Weltgeistes, Marx als Weg zum Kommunismus. Doch dieser Optimismus starb in den Schützengräben von Verdun und später in den Lagern des 20. Jahrhunderts.

Nietzsches Diagnose vom \enquote{Tod Gottes} war keine bloße atheistische Provokation, sondern eine seismographische Erschütterung: Der metaphysische Boden war weggebrochen. Es gab keinen Himmel mehr, der Trost spendete, und keine objektive Moral, die Richtung gab. Georg Lukács nannte diesen Zustand \textit{transzendentale Obdachlosigkeit}.
Der Mensch findet sich in einem Universum wieder, das \textbf{schweigt}. Es ist nicht mehr der wohlgeordnete Kosmos der Antike oder das Uhrwerk Voltaires. Es ist eine gleichgültige, fremde Masse.

In diesem Vakuum muss die Philosophie neu ansetzen. Sie kann nicht mehr vom \enquote{Allgemeinen} ausgehen (dem Geist, der Klasse, der Nation), sondern muss beim \textbf{Einzelnen} beginnen. Bei seiner Angst, seiner Verlassenheit und seiner monströsen Freiheit.
Der Existentialismus dreht die 2000 Jahre alte platonische Ordnung um. Der Leitsatz lautet:
\textit{\enquote{l'existence précède l'essence}} (Die Existenz geht der Essenz voraus).
Wir sind keine fertig definierten Wesen (wie ein Stuhl oder ein Messer), die einen Zweck haben. Wir \textit{sind} erst einfach, wir tauchen auf – und dann müssen wir uns selbst erfinden.

Doch wie soll man sich in dieser neuen, chaotischen Freiheit orientieren? Die traditionelle Metaphysik hatte ausgedient; ihre großen Begriffe von \enquote{Substanz} und \enquote{Geist} wirkten angesichts der konkreten menschlichen Erfahrung hohl und blutleer. Um die Existenz in ihrer ganzen Rohheit zu erfassen, brauchte die Philosophie eine radikale neue Methode. Sie musste lernen, wieder \enquote{naiv} zu sehen – ohne den Schleier alter Theorien, direkt auf das Erlebte gerichtet. Bevor der Existentialismus die menschliche Lage deuten konnte, musste er erst lernen, sie präzise zu beschreiben. Dies leistete eine Denkrichtung, die den Blick säuberte:

\subsection{Die phänomenologische Methode: Epoché und Intentionalität}
Um die Welt wirklich zu verstehen, müssen wir aufhören, sie einfach nur zu \enquote{konsumieren}.
Normalerweise leben wir im Autopiloten, in der sogenannten \textbf{\enquote{natürlichen Einstellung}}. Wir glauben naiv, dass die Welt da draußen einfach existiert und unser Kopf sie wie eine Kamera abfotografiert.

\begin{tcolorbox}[title=Das Kino-Gleichnis]
Wir sind wie Kinobesucher, die sich so sehr im Film verlieren, dass sie die Leinwand und den Projektor vergessen.
Husserl verlangt nun einen radikalen Schritt zurück: Die \textbf{Epoché} (das Einklammern).
Wir sollen nicht leugnen, dass der Film (die Welt) existiert. Aber wir sollen aufhören, uns in der Handlung zu verlieren, und stattdessen auf den Projektor schauen. Wir klammern unser Urteil \enquote{Das ist echt} ein und untersuchen stattdessen unser \textbf{Erleben}.
\end{tcolorbox}

Statt zu fragen: \enquote{Existiert dieser Kaffee wirklich?}, fragen wir: \enquote{Wie erscheint mir dieser Kaffee im Bewusstsein?}
In dem Moment, in dem wir das tun – wenn wir den Blick vom Objekt auf das Bewusstsein selbst wenden – machen wir eine schockierende Entdeckung.

Die alte Philosophie (seit Descartes) dachte immer, das Bewusstsein sei ein \textbf{Container} oder eine geschlossene Schachtel, in der Bilder (\enquote{Vorstellungen}) liegen.
Doch dank der Epoché sehen wir, dass das falsch ist. Wenn wir den Projektor ansehen, finden wir keine \enquote{Dinge} in ihm. Wir finden nur einen Lichtstrahl.
Wir finden in uns... nichts. Keine Kiste, keinen kleinen Mann im Ohr. Wir finden nur eine reine \textit{Richtung}. Einen Pfeil, der nach draußen zeigt.

\begin{tcolorbox}[title=Die Kern-Definition: Intentionalität]
Husserls bahnbrechende Erkenntnis lautet: \textit{\enquote{Bewusstsein ist immer Bewusstsein \textbf{von} etwas.}}
Das Bewusstsein ist kein Ort, an dem Dinge aufbewahrt werden. Es ist eine reine Aktivität, die auf Dinge \textit{zielt}.
Das Bewusstsein ist eine \textbf{Explosion}. Es explodiert ständig aus sich heraus hin zur Welt. Es ist ein transparenter Wind, der durch uns hindurchweht zu den Dingen. Es ist leer.
\end{tcolorbox}

Und genau deshalb sind wir unmittelbar \textit{in der Welt} und nicht in unserem Kopf gefangen.
Damit löst sich auch der Solipsismus (\enquote{Bin nur ich real?}) in Luft auf. Denn ohne Welt gäbe es gar kein Bewusstsein, da es nichts gäbe, worauf es sich richten könnte. Ein Pfeil ohne Ziel ist kein Pfeil.

Als Sartre diese Theorie zum ersten Mal hörte (angeblich in einer Bar, als sein Freund Raymond Aron auf seinen Aprikosen-Cocktail zeigte und rief: \enquote{Siehst du, mein Lieber, wenn du Phänomenologe bist, kannst du über diesen Cocktail sprechen, und das ist Philosophie!}), wurde er kreidebleich vor Begeisterung. Endlich konnte man über das konkrete Leben philosophieren, ohne die Welt zu verlassen.

\subsection{Eine neue Grammatik der Wirklichkeit: Das Erbe Husserls und die Wende Heideggers}
Was hat Husserl damit eigentlich bewirkt? Er hat die Philosophie von ihrem alten Dogma befreit, dass nur Zahlen und Atome \enquote{wahr} seien.

\begin{tcolorbox}[title=Das Erbe: Die Rettung der Lebenswelt]
Husserl lehrte uns, das Subjektive \textbf{ernst zu nehmen}. Für die Phänomenologie ist der Schmerz, den ich fühle, genauso \enquote{real} und objektiv beschreibbar wie eine mathematische Formel.
Sein wichtigster Begriff ist die \textbf{Lebenswelt}: Die Welt der Wissenschaft (Atome) ist nur ein abstraktes Modell. Aber die \textit{eigentliche} Welt ist die, in der wir leben, lieben und sterben. Die Welt der Farben, Töne und Werte. Husserl gab der Philosophie ihre Augen zurück.
\end{tcolorbox}

Doch genau hier hakte sein Meisterschüler \textbf{Martin Heidegger} (1889–1976) ein.
Husserl hatte zwar das \textit{Sehen} revolutioniert, aber er blieb ein \textbf{Zuschauer}. Er untersuchte immer noch, wie wir Dinge \textit{erkennen} (z.B. eine Kaffeetasse).
Heidegger aber sagte: Bevor wir Zuschauer sind, sind wir Akteure. Wir sitzen nicht im Kino und betrachten das Leben; wir spielen auf der Bühne mit, ohne das Skript zu kennen.

In seinem epochalen Werk \textit{Sein und Zeit} (1927) radikalisierte er den Ansatz:
\begin{itemize}
    \item \textbf{Zuhandenheit vs. Vorhandenheit (Der Hammer):}
    Ein klassisches Beispiel erklärt den Unterschied. Wenn ein Tischler hämmert, starrt er den Hammer nicht an. Der Hammer ist für ihn kein theoretisches Objekt (\textit{Vorhanden}), über das er nachdenkt. Der Hammer ist eins mit seiner Hand, er dient einem Zweck. Er ist \textit{zuhanden}.
    Erst wenn der Hammer \textit{zerbricht}, wird er plötzlich zum Objekt. Der Tischler hält inne und starrt das kaputte Ding an.
    Heidegger meint: Die westliche Philosophie (wie Husserl) hat immer nur den \textit{zerbrochenen Hammer} angestarrt. Sie hat das theoretische Betrachten für wichtiger gehalten als den praktischen Umgang. Aber unser primärer Zugang zur Welt ist die \textbf{Sorge} (das Hantieren, Besorgen, Kümmern), nicht das Gaffen.

    \item \textbf{Das Man (Die Uneigentlichkeit):}
    Meistens sind wir gar nicht wir selbst. Wir tun, was \enquote{man} tut. Wir wählen, was \enquote{man} wählt. Wir sagen: \enquote{Man geht heute nicht mehr ohne Maske raus} oder \enquote{Man findet diesen Film gut}.
    Dieses \textbf{Man} ist eine Diktatur des Durchschnitts. Es nimmt uns die Last der Entscheidung ab. Wir gleiten in die \textbf{Verfallenheit}, in ein unauthentisches Leben.

    \item \textbf{Sein-zum-Tode und Angst:}
    Was reißt uns aus dieser bequemen Betäubung? Die \textbf{Angst}. Nicht die Furcht vor einer Spinne (die hat ein Objekt), sondern die grundlose Angst, die uns überfällt, wenn die Welt plötzlich sinnlos erscheint.
    Und vor allem: Der \textbf{Tod}. Niemand kann mir meinen Tod abnehmen. Sterben muss ich allein. \enquote{Man} stirbt nicht; \textit{Ich} sterbe.
    Das Vorlaufen zum Tode reißt das Dasein aus dem \enquote{Man} heraus und zwingt es zur \textbf{Eigentlichkeit}.
\end{itemize}

Heidegger lieferte damit das Vokabular für die Moderne: Geworfenheit, Sorge, Angst, Tod. Doch er analysierte dies (jedenfalls in \textit{Sein und Zeit}) mit der Kälte eines Geologen, der Gesteinsschichten untersucht. Er beschrieb das Dasein, aber er rief nicht zur Revolution auf. Das überließ er seinem französischen Interpreten.

\subsection{Die Politisierung: Von Heidegger zu Sartre}
Hier geschah eines der fruchtbarsten Missverständnisse der Philosophiegeschichte.
Jean-Paul Sartre verbrachte das Jahr 1933 als Stipendiat in Berlin. Während draußen die Fackelmärsche der Nazis begannen und Hitler die Macht ergriff, saß Sartre in der Bibliothek und verschlang Husserl und Heidegger.
Er war fasziniert von der Wucht der Heideggerschen Begriffe, führte aber eine entscheidende \enquote{Französisierung} durch:

\begin{enumerate}
    \item \textbf{Die Rückkehr des Subjekts:} Heidegger wollte das \enquote{Ich} (das cartesianische Subjekt) überwinden. Für ihn war das \enquote{Dasein} wichtiger als das Bewusstsein. Sartre aber, im Herzen ein Schüler Descartes', holte das \enquote{Ich} zurück. Für ihn war das \enquote{Nichts} kein mystisches Seins-Geheimnis, sondern der \textbf{Abstand}, den das freie Bewusstsein zu den Dingen hat.
    \item \textbf{Vom Schicksal zum Entwurf:} Heideggers Begriff der \enquote{Geworfenheit} klingt nach Schicksal, nach Boden, nach Schwere. Sartre akzeptierte die Geworfenheit, setzte ihr aber den \textbf{Entwurf} (\textit{le projet}) entgegen.Wir sind vielleicht ohne Grund hier, aber wir können uns selbst in die Zukunft entwerfen. Wir sind nicht das, was wir \textit{sind} (Vergangenheit), sondern das, was wir \textit{sein werden} (Zukunft).
    \item \textbf{Von der Hütte zum Boulevard:} Heideggers Philosophie atmet die Schwere der deutschen Romantik und des Waldes. Sartres Philosophie atmet die Hektik der Großstadt, das Klickern der Espressotassen und den Rauch der Gauloises. Sie ist eine Philosophie der radikalen Entwurzelung.
\end{enumerate}

Sartre nahm die Waffe, die Heidegger geschmiedet hatte, und richtete sie auf ein anderes Ziel: Die absolute Freiheit des Individuums gegen jede Autorität (sei es Gott, der Staat oder die Tradition). Aus der deutschen \enquote{Existenzphilosophie} wurde der französische \enquote{Existentialismus}.

\section{Jean-Paul Sartre: Die Verurteilung zur Freiheit}
Mitten im besetzten Paris, in der Kälte eines ungeheizten Zimmers, schrieb Jean-Paul Sartre das Werk, das wie ein Meteorit in die Geistesgeschichte einschlug: \textit{Das Sein und das Nichts} (1943). Es ist ein philosophischer Koloss von 700 Seiten, der den französischen Widerstandsgeist mit der deutschen Metaphysik verschmilzt.
Um dieses Werk zu verstehen, muss man die Synthese begreifen, die Sartre hier vollzieht. Er steht auf den Schultern zweier Riesen, aber er blickt in eine völlig andere Richtung.

\textbf{1. Das Erbe Husserls (Die Leere):}
Von Husserl übernimmt Sartre die \textit{Intentionalität}. Er akzeptiert, dass das Bewusstsein keine \enquote{Substanz} ist, sondern reine Ausrichtung auf die Welt. Aber Sartre radikalisiert diesen Gedanken: Wenn das Bewusstsein nur ein Wind ist, der zu den Dingen weht, dann ist es in sich selbst \textit{Nichts}. Es ist eine Leerstelle im Sein.

\textbf{2. Das Erbe Heideggers (Das In-der-Welt-sein):}
Von Heidegger übernimmt er das Vokabular der Existenz: die Faktizität, die Geworfenheit, den Entwurf. Er stimmt zu: Wir sind keine abstrakten Geister, sondern wir existieren \textit{in Situationen}.

\textbf{3. Der Bruch: Das An-sich und das Für-sich}
Hier vollzieht Sartre seine revolutionäre Wendung. Er teilt die Realität rigoros in zwei Bereiche, die sich unversöhnlich gegenüberstehen:
\begin{itemize}
    \item \textbf{Das An-sich-Sein (L’en-soi):} Das ist die Welt der Dinge. Der Stein, der Tisch, der Baum. Sie sind massiv, voll, sie ruhen in sich selbst. Sie sind einfach, was sie sind. Sie haben keine Freiheit, keine Lücke, keine Zukunft. Sie sind pure Identität.
    \item \textbf{Das Für-sich-Sein (Le pour-soi):} Das ist der Mensch (das Bewusstsein). Weil wir uns unserer selbst bewusst sind, haben wir Distanz zu uns selbst. Wir sind nie deckungsgleich mit uns wie der Stein.
\end{itemize}

\begin{tcolorbox}[title=Der Unterschied zu Heidegger]
Hier liegt der entscheidende Unterschied:
\textbf{Heidegger} sah das \enquote{Dasein} als eine \textbf{Einheit}. Für ihn sind Mensch und Welt untrennbar verwoben (\enquote{In-der-Welt-sein}). Er wollte die Trennung von Subjekt und Objekt überwinden.
\textbf{Sartre} hingegen reißt diesen Graben wieder auf. Er ist ein radikaler Dualist.
Für Sartre gibt es einen \textbf{Krieg} zwischen dem Bewusstsein (dem Nichts) und der Materie (dem Sein).
Das \enquote{Für-sich} ist kein Teil der Welt wie bei Heidegger, sondern ein Fremdkörper, der die Welt wie ein \enquote{Loch} durchlöchert. Sartre holt damit Descartes' Ich zurück, macht es aber zu einem \enquote{nichts-enden} Ich.
\end{tcolorbox}

\paragraph{Die Begründung: Woher kommt das Nichts?}
Aber warum behauptet Sartre so etwas Radikales? Ist das nicht reine Wortspielerei?
Sartres Beweis ist phänomenologisch: Er analysiert unsere Erfahrung der \textbf{Verneinung} (Negativität).
Stellen wir uns eine Kamera vor, die das Café fotografiert. Die Kamera registriert alle Tische, alle Gäste, den Rauch. Sie registriert nur das, was \textit{ist}. Sie kann niemals das fotografieren, was \textit{fehlt}. Für die Kamera (und für die Materie generell) ist das Sein einfach nur voll.
Doch wenn ich das Café betrete, sehe ich schlagartig: \textit{Pierre ist nicht da.}
Ich sehe seine \textbf{Abwesenheit} fast so deutlich wie die Tische.
Woher kommt dieses \enquote{Nicht}? Das Café produziert es nicht (das Café ist voll). Die Kamera sieht es nicht.
Die Schlussfolgerung ist zwingend: \textit{Ich} muss es sein, der das \enquote{Nicht} in den Raum bringt.
Damit ich überhaupt sagen kann \enquote{Der Tisch ist \textit{nicht} Pierre} oder \enquote{Ich bin \textit{nicht} der Tisch}, muss ich innerlich einen Abstand zu den Dingen haben.
Wäre ich selbst nur Materie (ein An-sich), wäre ich lückenlos mit der Welt verschmolzen. Ich klebte an den Dingen.
Dass ich mich aber geistig zurückziehen kann, Fragen stellen kann und Abwesenheit bemerken kann, beweist: Ich bin keine massive Materie. Ich trage einen \textbf{Abstand} in mir.
Und genau diesen Abstand – diese Fähigkeit, \enquote{Nein} zu sagen und auf Distanz zu gehen – nennt Sartre das \textbf{Nichts}. Es ist keine schwarze Leere, sondern die \textbf{Freiheit} des Bewusstseins, sich nicht von der Materie erdrücken zu lassen.

Sartres geniale Definition lautet daher: Der Mensch ist das Wesen, durch das das \textbf{Nichts} in die Welt kommt.
Wir sind wie ein \enquote{Wurm im Apfel des Seins}. Wir höhlen das massive Sein aus, indem wir Fragen stellen, warten, vermissen und uns Alternativen vorstellen.
Diese ontologische Lücke – dieses \enquote{Nicht-Sein} in unserem Herzen – ist genau das, was wir \textbf{Freiheit} nennen. Wir sind frei, nicht weil wir toll oder mächtig sind, sondern weil wir undeterminiert sind. Wir sind ein Mangel, der gefüllt werden muss.

Es ist bezeichnend, dass dieses Werk der absoluten Freiheit unter der deutschen Besatzung entstand. Nie waren die Franzosen freier als damals, schrieb Sartre in einem berühmten Artikel, denn die Besatzung nahm ihnen die Normalität. Jeder Akt – selbst das Kaufen einer Zeitung oder das Schweigen vor einem Offizier – war plötzlich keine Routine mehr, sondern eine absolute moralische Wahl: Kollaboration oder Widerstand. Aus der Ontologie des Nichts folgt somit die Ethik der totalen Verantwortung.

\subsection{Die Existenz geht der Essenz voraus}
Hier dockt Sartres berühmtester Slogan an. Er ist keine leere Phrase, sondern die logische Konsequenz seiner Ontologie:
Weil der Mensch ein \enquote{Nicht-Sein} (Für-sich) ist, hat er keine feste Definition. Er ist nicht \enquote{fertig} wie ein Stein.
Daraus folgt: \textbf{Die Existenz geht der Essenz voraus.}
Um diese Tragweite zu verstehen, muss man sie gegen die Tradition halten.
In der antiken und christlichen Philosophie galt das Gegenteil: Die Essenz (das Wesen) kommt zurerst.
\begin{tcolorbox}[title=Das Beispiel des Papiermessers]
Sartre erklärt das am Beispiel eines \textbf{Papiermessers}:
\begin{itemize}
    \item Ein Handwerker will ein Papiermesser bauen. Er hat das \textit{Konzept} (die Essenz) im Kopf: Es muss schneiden, es muss fest sein.
    \item Erst dann stellt er es her (Existenz).
    \item Hier gilt: \textbf{Essenz $\rightarrow$ Existenz}. Das Messer hat eine Bestimmung, bevor es überhaupt existiert.
\end{itemize}
\end{tcolorbox}
Jahrhundertelang dachte man den Menschen genauso: Gott (der übernatürliche Handwerker) hat die Idee des Menschen im Verstand, bevor er ihn erschafft. Jeder von uns hätte also eine \enquote{vorherbestimmte} Natur.

Sartres Existentialismus aber ist \textbf{konsequenter Atheismus}.
Wenn es keinen Gott gibt, gibt es niemanden, der uns im Voraus \enquote{denkt}.
Es gibt keine \enquote{menschliche Natur}, keinen Bauplan, kein Schicksal.
Der Mensch taucht zuerst in der Welt auf (er existiert), er begegnet sich, er springt in die Welt – und erst \textit{danach} definiert er sich.
Der Mensch ist am Anfang \textbf{Nichts}. Er wird erst später etwas sein, und zwar das, was er aus sich macht.

\begin{tcolorbox}[title=Die Tragweite: Das Ende der Ausreden]
Das klingt befreiend, ist aber brutal.
Wenn die Existenz der Essenz vorausgeht, gibt es \textbf{keinen Determinismus}.
Wir können nicht sagen: \enquote{Ich bin halt so, ich habe ein jähzorniges Temperament.}
Oder: \enquote{Ich kann nichts dafür, meine Kindheit war schwer.}
Sartre sagt: Du \textit{hast} kein Temperament, du \textit{wählst} es in jedem Moment neu. Ein Feigling wird nicht als Feigling geboren. Er definiert sich durch seine feigen Taten als Feigling. Aber er könnte jederzeit aufhören, einer zu sein.
Der Mensch ist nichts anderes als sein \textbf{Entwurf}. Er existiert nur in dem Maße, wie er sich verwirklicht. Er ist die Summe seiner Handlungen.
\end{tcolorbox}

\subsection{Die Verurteilung zur Freiheit}
Doch dieses Fehlen einer Essenz ist kein Grund zum Feiern. Es ist eine Katastrophe.
Wenn es keinen Gott gibt, keine menschliche Natur und keinen Determinismus, dann sind wir völlig \textbf{verlassen} (délaissement). Wir haben keine Entschuldigungen mehr. Wir stehen allein in der Kälte des Universums, ohne Geländer.

Sartre fasst dieses Drama in einem der berühmtesten Sätze der Philosophiegeschichte zusammen:
\begin{quote}
    \textit{\enquote{Der Mensch ist zur Freiheit verurteilt.}}
\end{quote}
Das Wort \textbf{verurteilt} ist hier kein stilistischer Effekt, sondern ontologische Präzision.
Wir sind verurteilt, weil wir uns nicht selbst geschaffen haben. Wir wurden in diese Welt geworfen (Faktizität), ohne gefragt zu werden.
Aber sobald wir in die Welt geworfen sind, sind wir \textbf{frei}, weil wir für alles, was wir tun, verantwortlich sind. Wir können nicht \textit{nicht} wählen. Selbst wenn ich mich entscheide, nichts zu tun, ist das eine Wahl (die Wahl der Passivität).

Diese bodenlose Freiheit löst ein fundamentales Gefühl aus: \textbf{Angst} (l'angoisse).
Hier muss man philosophisch strikt unterscheiden zwischen \textbf{Furcht} (peur) und Angst.
\begin{itemize}
    \item \textbf{Furcht} hat ein Objekt in der Welt. Ich fürchte mich vor einem knurrenden Hund, vor dem Bankrott oder vor einer explodierenden Bombe. Die Gefahr kommt von außen.
    \item \textbf{Angst} hingegen hat kein Objekt. Sie kommt von innen. Sie ist die Angst vor mir selbst, vor meiner eigenen Freiheit.
\end{itemize}

Sartre (und vor ihm Kierkegaard) illustriert dies am Beispiel des \textbf{Schwindels am Abgrund}:
Wenn ich auf einem schmalen Pfad an einer Klippe wandere, habe ich \textit{Furcht}, auszurutschen und zu sterben. Diese Furcht ist rational; sie lässt mich vorsichtig gehen. Ich vertraue dem Geländer, es schützt mich vor der Schwerkraft.
Aber dann kommt ein anderes Gefühl. Ich erkenne plötzlich: Das Geländer hält mich zwar physisch zurück, aber \textit{nichts} zwingt mich, dort und nicht im Abgrund zu sein.
Ich \textit{könnte} mich hinunterstürzen.
Kein Naturgesetz, kein Gott und kein Instinkt verhindert, dass ich springe. Nur mein eigener Wille hält mich zurück. Aber mein Wille ist instabil, ich muss ihn in jeder Sekunde neu fassen.
Dieser Schwindel ist die \textbf{Angst}: Das schwindelerregende Bewusstsein, dass ich der alleinige Urheber meiner Zukunft bin und dass nichts mich vor mir selbst schützt.

Diese Freiheit bedeutet eine \textbf{absolute Verantwortung}.
Wenn ich wähle, wähle ich nicht nur für mich allein. Ich erschaffe durch meine Taten ein Bild des Menschen.
Wenn ich heirate und Kinder bekomme, behaupte ich implizit: \enquote{Monogamie ist der richtige Weg für den Menschen.} Ich werde zum Gesetzgeber der Menschheit.
Jede meiner Handlungen ist eine Botschaft an die Welt: \enquote{So soll der Mensch sein.}
Diese Last – die Verantwortung für die ganze Welt auf den eigenen Schultern zu tragen, ohne göttliches Mandat – ist es, was die Existenz so schwer (und heroisch) macht.

\subsection{Mauvaise Foi: Die Flucht in die Unaufrichtigkeit}
Diese Angst vor der absoluten Freiheit ist oft unerträglich. Deshalb verbringen wir unser Leben damit, vor ihr zu fliehen.
Wir suchen nach Wegen, uns selbst zu belügen, um uns einzureden, wir seien \textit{nicht} frei. Wir wollen glauben, wir seien feste Dinge (Charaktere, Rollen), die determiniert sind.
Sartre nennt diese fundamentale Selbsttäuschung \textbf{Mauvaise Foi} (schlechter Glaube / Unaufrichtigkeit).

Sein berühmtestes Beispiel ist eine phänomenologische Meisterleistung: \textbf{Der Kellner im Café}.
Beobachten wir ihn. Seine Bewegungen sind lebhaft und drängend, ein wenig \textit{zu} präzise, ein wenig \textit{zu} schnell. Er kommt auf die Gäste zu mit einem Schritt, der ein wenig \textit{zu} eifrig ist. Er verneigt sich mit einer Stimme, die ein wenig \textit{zu} besorgt ist.
Seine Augen drücken ein Interesse aus, das künstlich wirkt.
Was macht er da? Er \textit{spielt}. Er spielt das Spiel, ein Kellner zu sein.
Warum diese Komödie?
Der Kellner versucht, sich selbst in ein \textbf{Ding} zu verwandeln. Er möchte \enquote{Kellner-sein} so wie ein Stein \enquote{Stein-ist}. Er möchte mit seiner Funktion verschmelzen, um die schwindelerregende Freiheit loszuwerden, jederzeit etwas anderes sein zu können.
Er flüchtet aus dem unruhigen \textbf{Für-sich} (Freiheit) in die Sicherheit des \textbf{An-sich} (Sache).
Er sagt sich: \enquote{Ich habe keine Wahl, ich muss früh aufstehen, ich muss freundlich sein, ich bin nun mal Kellner.}
Das ist die Lüge. Denn er \textit{weiß}, dass er kein Kellner-Ding ist. Er weiß, dass er jeden Morgen neu entscheidet, aufzustehen und diese Rolle zu spielen. Die Mauvaise Foi ist die Kunst, die eigene Freiheit zu kennen und sie gleichzeitig vor sich selbst zu verleugnen.

\subsection{Die Hölle, das sind die anderen}
Dies ist das vielleicht am meisten missverstandene Zitat der Philosophiegeschichte.
In Sartres Stück \textit{Geschlossene Gesellschaft} sagt Garcin: \textit{\enquote{L'enfer, c'est les autres.}} (Die Hölle, das sind die anderen).
Das bedeutet nicht, dass andere Menschen nerven oder dass wir Misanthropen sein sollten.
Es ist eine ontologische Aussage über den \textbf{Blick} (Le Regard).

Um das zu verstehen, nutzen wir Sartres Beispiel des \textbf{Spanners am Schlüsselloch}:
Stellen wir uns jemanden vor, der aus Eifersucht durch ein Schlüsselloch in ein Zimmer späht.
\begin{itemize}
    \item \textbf{Phase 1 (Das Subjekt):} Solange er allein ist, ist er reines Bewusstsein. Er sieht die Szene im Zimmer, aber er sieht sich selbst nicht. Er ist der unsichtbare Mittelpunkt der Welt. Die Welt gehört ihm. Er ist frei.
    \item \textbf{Phase 2 (Der Blick):} Plötzlich hört er Schritte im Flur. Jemand steht hinter ihm.
    \item \textbf{Phase 3 (Das Objekt):} Schlagartig gefriert er. Er wird \textit{gesehen}.
\end{itemize}
In diesem Moment geschieht etwas Metaphysisches: Der Spanner wird vom Subjekt zum Objekt.
Der Andere sieht ihn nicht als \enquote{freies Bewusstsein}, sondern als \enquote{einen Spanner}. Er nagelt ihn auf eine Definition fest.
Der Andere stiehlt mir meine Welt. Die Dinge im Flur ordnen sich nicht mehr um den Spanner, sondern um den Neuankömmling.
Sartre nennt das den \textbf{Abfluss}: Wenn ein Anderer den Raum betritt, ist es, als würde ein Abfluss im Boden entstehen, durch den meine ganze Welt abfließt. Sie ordnet sich neu – um \textit{ihn}.

Das Gefühl, das diesen Wandel begleitet, ist \textbf{Scham}.
Scham ist das Eingeständnis: \enquote{Ich bin wirklich dieser Gegenstand, den der Andere ansieht.}
Deshalb ist die Hölle die anderen: Weil wir uns selbst nie vollständig sehen können. Wir brauchen den Anderen, um zu wissen, wer wir sind. Aber der Andere macht uns zum Ding. Wir sind seiner Freiheit ausgeliefert. Wir sind Geiseln seines Urteils.
Der ewige Kampf darum, wieder Subjekt zu werden, indem man den Anderen zum Objekt macht, ist das Drama menschlicher Beziehungen.

\paragraph{Zusammenfassung: Sartres Drama der Freiheit}
Damit schließt sich der Kreis von Sartres Philosophie. Sie ist ein Drama in drei Akten:
Erstens die \textbf{Ontologie}: Wir sind das Nichts, das Loch im Sein, und deshalb radikal frei.
Zweitens die \textbf{Ethik}: Diese Freiheit verurteilt uns zur totalen Verantwortung, vor der wir oft in die Unaufrichtigkeit (Mauvaise Foi) fliehen.
Drittens die \textbf{Sozialphilosophie}: Der Andere ist die Grenze meiner Freiheit. Die \enquote{Hölle} ist keine moralische Verurteilung, sondern eine ontologische Tatsache: Der Andere hält den Schlüssel zu meinem Sein. Ich brauche ihn, um zu existieren, aber er stiehlt mir meine Welt. Wir sind dazu verdammt, frei zu sein, aber wir sind auch dazu verdammt, für andere ein Objekt zu sein.

\section{Albert Camus: Die Philosophie des Absurden}
Während Sartre die Philosophie in dunkle, deutsche Tiefe zieht, bringt \textbf{Albert Camus} (1913–1960) das mediterrane Licht in den Existentialismus.
Die beiden waren kurzzeitig Gefährten, doch ihre Temperamente könnten nicht unterschiedlicher sein:
Sartre war ein Kind der Bibliothek, ein hässlicher Intellektueller, der über Hegel brütete.
Camus war ein Kind der Straße, ein Torwart, ein Frauenheld, aufgewachsen in der armen, aber sonnendurchfluteten Welt Algiers.
Camus lehnte das Etikett \enquote{Existentialist} stets ab. Er baute kein komplexes ontologisches System wie Sartre. Sein Denken ist direkter, sinnlicher und tragischer. Sein Thema ist nicht die Freiheit (wie bei Sartre), sondern die \textbf{Konfrontation}.

\begin{tcolorbox}[title=Der Konflikt: Warum es zum Bruch kam]
Warum zerbrach diese Freundschaft? Weil ihre Philosophien politisch unvereinbar waren:
\begin{itemize}
    \item \textbf{Sartre (Die Geschichte):} Für Sartre ist der Mensch reine Freiheit. Er \textit{macht} sich selbst. Daraus folgt politisch: Wir können die Gesellschaft neu erschaffen (Kommunismus). Wer die Welt ändern will, muss sich die \enquote{Hände schmutzig machen}. Gewalt kann notwendig sein, um die Freiheit zu erkämpfen. Sartre stellte die \textbf{Geschichte} über die Moral.
    \item \textbf{Camus (Die Natur):} Für Camus gibt es eine unveränderliche menschliche Natur und eine Grenze (das Absurde, den Tod). Wir können das Paradies nicht auf Erden erzwingen. Er lehnte jede Diktatur ab (auch die stalinistische). Sein Credo: Man darf keine echten Menschen heute für eine theoretische Zukunft morgen opfern. Camus stellte die \textbf{Moral} über die Geschichte.
\end{itemize}
1952 kam es zum Eklat: Sartre warf Camus vor, ein naiver Moralist zu sein, der die politische Realität ignoriert. Camus sah in Sartre einen Intellektuellen, der im Namen der Freiheit bereit war, Tyrannei zu entschuldigen.
\end{tcolorbox}

Camus beginnt nicht mit dem \enquote{Nichts}, sondern mit einem Gefühl, das uns unvermittelt treffen kann: dem \textbf{Absurden}.
Was ist das Absurde?
Es ist nicht die Welt selbst (die Welt ist einfach da). Und es ist nicht der Mensch (der Mensch ist tragisch).
Das Absurde ist das Band, das beide verknüpft. Es entsteht aus dem \textbf{Zusammenstoß} zwischen zwei unvereinbaren Kräften:
\begin{itemize}
    \item Dem \textbf{Menschen}, der leidenschaftlich nach Sinn, Einheit und Klarheit verlangt (\enquote{Nostalgie}).
    \item Der \textbf{Welt}, die darauf mit einem frostigen, \enquote{zärtlichen Schweigen} (\textit{tendre indifférence}) antwortet.
\end{itemize}
Der Mensch schreit: \enquote{Warum?} und das Universum sagt gar nichts. Dieser Skandal ist das Absurde.

Camus beschreibt, wie dieses Gefühl uns im Alltag überfällt. Es ist wie das Erwachen eines Schauspielers, der seinen Text vergessen hat.
Man denke an einen Mann, der in einer Telefonzelle wild gestikuliert. Man hört ihn nicht, man sieht nur seine grotesken Bewegungen. Er wirkt wie eine mechanische Puppe.
Plötzlich fragen wir uns: \enquote{Warum macht der das? Was soll das ganze Theater?}
In diesem Moment bröckelt die Kulisse. Die \enquote{dichte Fremdheit} der Welt schlägt uns entgegen. Ein Stein ist plötzlich nur ein Stein, fremd und unzugänglich. Das Gesicht im Spiegel wird uns fremd.
Das Absurde ist der Moment, in dem die Gewohnheit stirbt und die Frage nach dem Sinn nackt vor uns steht.

\subsection{Der Mythos des Sisyphos}
Wenn das Leben keinen vorgegebenen Sinn hat und die Welt schweigt, drängt sich eine brutale Frage auf. Camus stellt sie im ersten Satz seines Buches:
\begin{quote}
    \textit{\enquote{Es gibt nur ein wirklich ernstes philosophisches Problem: den Selbstmord.}}
\end{quote}
Wenn das Leben absurd ist, warum beenden wir es nicht?
Camus' Antwort ist ein entschiedenes \textbf{Nein} zum Selbstmord. Der Selbstmord ist eine Kapitulation. Er löst das Absurde auf, indem er das Bewusstsein vernichtet.
Camus fordert das Gegenteil: Wir müssen das Absurde am Leben erhalten. Wir müssen revoltieren.

Sein Held dieser Revolte ist \textbf{Sisyphos}.
Von den Göttern dazu verurteilt, einen Felsbrocken einen Berg hinaufzurollen, nur damit dieser oben wieder hinunterrollt – in alle Ewigkeit. Eine sinnlose, hoffnungslose Arbeit.
Camus beschreibt die körperliche Anstrengung: das Gesicht gegen den Stein gepresst, die Wange im Staub, die angespannten Muskeln.
Doch Camus interessiert sich für einen ganz bestimmten Moment: \textbf{die Stunde des Bewusstseins}.
Es ist der Moment, in dem der Stein hinuntergerollt ist und Sisyphos ihm langsam, schweren Schrittes, folgt.
In diesem Augenblick weiß er, dass seine Qual kein Ende hat.
Eigentlich müsste er verzweifeln. Aber genau hier triumphiert er.
Er erkennt: Dieser Stein ist \textit{sein} Stein. Dieses Schicksal ist \textit{sein} Schicksal.
Indem er sein Schicksal annimmt und es verachtet (\enquote{Es gibt kein Schicksal, das durch Verachtung nicht überwunden werden kann}), macht er sich zum Herrn seiner Tage. Er leugnet die Götter und hebt den Stein wieder auf.
Seine Freiheit liegt nicht darin, dass er aufhören kann (das kann er nicht), sondern darin, dass er weitermacht, \textit{obwohl} es sinnlos ist. Das ist die \textbf{Revolte}.

Sein berühmter Schlusssatz ist kein naiver Optimismus, sondern ein heroischer Trotz:
\textit{\enquote{Der Kampf gegen Gipfel vermag ein Menschenherz auszufüllen. Wir müssen uns Sisyphos als einen glücklichen Menschen vorstellen.}}

Doch diese Revolte bleibt bei Camus nicht einsam. Sie weitet sich zur Ethik der Solidarität.
Hier zieht Camus den endgültigen Trennstrich zu Sartre.
Während Sartre in der \textbf{Revolution} den gewaltsamen Umsturz der Gesellschaft suchte (und dafür stalinistische Lager in Kauf nahm), setzte Camus auf die \textbf{Revolte}.
Der Unterschied ist fundamental:
Die Revolution opfert die lebenden Menschen für eine theoretische Idee in der Zukunft. Die Revolte aber ist der Aufstand für die Würde des Menschen im \textit{Hier und Jetzt}.
Camus' Antwort auf Descartes lautet deshalb nicht \enquote{Ich denke, also bin ich}, sondern:
\begin{quote}
    \textit{\enquote{Ich revoltiere, also sind wir.}}
\end{quote}
Weil wir alle das gleiche absurde Schicksal teilen (wir rollen alle unseren Stein), sind wir Brüder im Leid.
Sartre wollte Geschichte schreiben, selbst wenn er dafür über Leichen gehen musste.
Camus wollte verhindern, dass die Welt auseinanderbricht. Er predigte die \textbf{Maßhaltigkeit} (\textit{la mesure}) – das Wissen, dass der Mensch Grenzen hat und dass keine Ideologie wertvoller ist als ein Menschenleben.
Sartre ist der Philosoph der Freiheit und der Zukunft. Camus ist der Philosoph der Gerechtigkeit und der Gegenwart – des Sonnenlichts, das auf den Stein fällt.

\section{Simone de Beauvoir: Die Freiheit des Anderen}
Lange Zeit wurde \textbf{Simone de Beauvoir} (1908–1986) nur als \enquote{die Gefährtin} von Sartre wahrgenommen. Ein fataler Irrtum.
Während Sartre die abstrakte Ontologie lieferte (Nichts, Sein, Freiheit), wandte Beauvoir diese Werkzeuge auf die konkrete, blutige Realität an.
Ihr Hauptwerk \textit{Das andere Geschlecht} (Le Deuxième Sexe, 1949) ist nicht nur ein feministisches Manifest, sondern eine existentialistische Analyse der Unterdrückung.

Beauvoir übernimmt Sartres Dualismus von \textbf{Transzendenz} (Hinausgreifen, Entwerfen, Freiheit) und \textbf{Immanenz} (Stagnation, Verdinglichung).
Jedes menschliche Bewusstsein strebt nach Transzendenz. Es will in die Welt hinauswirken.
Das historische Verbrechen des Patriarchats besteht laut Beauvoir darin, dass der Mann sich selbst als das \textbf{Subjekt} (das Absolute) setzt und die Frau zum \textbf{Anderen} (dem Objekt) erklärt.
Die Frau wird in die \textbf{Immanenz} gezwungen:
Sie soll nicht handeln, entdecken oder erschaffen (Transzendenz), sondern sie soll \textit{sein}. Sie soll warten, bewahren, gefallen. Sie wird dazu verdammt, ein An-sich zu bleiben, während der Mann sich als Für-sich entfaltet.

Daraus ergibt sich ihr berühmtester Satz, der die Gender-Studies begründete:
\begin{quote}
    \textit{\enquote{Man kommt nicht als Frau zur Welt, man wird es gemacht.}}
\end{quote}
Dieser Satz ist die direkte Anwendung von \enquote{Existenz geht der Essenz voraus}.
Es gibt keine \enquote{weibliche Natur} oder eine \enquote{ewig weibliche Seele}. Biologie ist kein Schicksal.
Was wir als \enquote{Frau-Sein} bezeichnen (Passivität, Koketterie, Mütterlichkeit), ist das Ergebnis einer lebenslangen Dressur durch die Zivilisation. Mädchen werden dazu erzogen, ihre Transzendenz aufzugeben und sich zum Objekt für den Mann zu machen.

Beauvoirs Philosophie mündet in eine \textbf{Ethik der Zweideutigkeit}:
Anders als der frühe Sartre, der die Freiheit als einsamen Akt sah, erkannte Beauvoir: \enquote{Sich selbst frei wollen, heißt auch, die anderen frei wollen.}
Ich kann nicht wirklich frei sein, wenn ich von Unfreien umgeben bin. Die Befreiung der Frau ist daher keine reine Frauenfrage, sondern die Bedingung für die Freiheit des Mannes selbst. Solange er den Anderen unterdrückt, bleibt er selbst in seiner Rolle als Herr gefangen.

\section{Konsequenz: Vom Ekel zum Engagement}
Was bleibt vom Existentialismus? Er begann 1938 mit einem Gefühl des \textbf{Ekels} (La Nausée) vor der Sinnlosigkeit der Existenz. Aber er endete nicht in der Depression. Er endete in der Tat.
Wenn es keinen Gott gibt und keinen Plan, dann sind wir die einzigen Gesetzgeber. Wir sind verurteilt, den Sinn selbst zu erschaffen.
Daraus leitet Sartre den Imperativ des \textbf{Engagements} ab.
Der Philosoph darf nicht mehr im Elfenbeinturm sitzen. Er muss sich einmischen. Wer schweigt, wählt auch (er wählt die Komplizenschaft). Sartre verkaufte Zeitungen auf der Straße, protestierte gegen Kriege und nutzte seine Prominenz als politische Waffe.
Der Existentialismus ist, wie Sartre 1946 betonte, ein \textbf{Humanismus}: Er gibt dem Menschen seine Würde zurück, indem er ihm sagt: \enquote{Du bist nichts anderes als dein Leben. Dein Schicksal liegt in deinen Händen.}

\subsection*{Ausblick: Der Tod des Subjekts}
Doch jedes Pendel schlägt zurück.
Sartre hatte das \textbf{Subjekt} (das \enquote{Ich}) zum absoluten König gemacht. Das \enquote{Für-sich} war mächtiger als jede Gesellschaft, jede Psychologie, jede Geschichte.
In den 1950er und 60er Jahren begann sich der Wind in Paris zu drehen. Eine neue Generation von Denkern fragte:
\textit{Sind wir wirklich so frei? Oder werden wir nicht vielmehr von unsichtbaren Strukturen gesteuert, die wir gar nicht bemerken?}
Spricht wirklich das \enquote{Ich}, oder spricht die \textbf{Sprache} durch uns? Denkt das Subjekt, oder denkt das \textbf{Unbewusste} in ihm?
Am Horizont zog ein neues theoretisches Gewitter auf, das den König stürzen sollte. Claude Lévi-Strauss studierte die Mythen der Ureinwohner, Jacques Lacan sezierte das Unbewusste, Michel Foucault analysierte die Macht.
Sie alle teilten ein Ziel: Den Existentialismus zu beerdigen und das \enquote{Subjekt} aufzulösen.
Nach der Ära der Freiheit begann die Ära der \textbf{Struktur}.
